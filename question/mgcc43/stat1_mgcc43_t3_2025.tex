\documentclass[12pt]{article}
\usepackage{geometry}
\usepackage{amsfonts}
\usepackage{float}
\usepackage{amsmath}

\geometry{
legalpaper, total={177.8mm, 290mm},left=5mm, right=5mm,
top=7mm, bottom=12mm,
}

\usepackage{letltxmacro}

\LetLtxMacro\Oldfootnote\footnote

\newcommand{\EnableFootNotes}{%
  \LetLtxMacro\footnote\Oldfootnote%
}

\newcommand{\DisableFootNotes}{%
  \renewcommand{\footnote}[2][]{\relax}
}

\DisableFootNotes

\begin{document}

\iffalse
\begin{table}[]
\begin{tabular}{llcllcll}
 &  & \textbf{MYMENSINGH GIRLS’  CADET COLLEGE} &  &                                            & \multicolumn{1}{l}{}            &                        &                        \\
 &  & WHAT EXAMINATION - 2025                   &  &                                            &                                 &                        &                        \\
 &  & CLASS: WHAT                               &  &                                            &                                 &                        &                        \\
 &  & STATISTICS (CREATIVE)                     &  &                                            &                                 &                        &                        \\
 &  & WHAT PAPER                                &  &                                            & \multicolumn{1}{r}{}            &                        &                        \\ \cline{7-8} 
 &  & [According to the Syllabus of 2025]       &  & \multicolumn{1}{r}{\textbf{Subject Code:}} & \multicolumn{1}{l|}{\textbf{1}} & \multicolumn{1}{l|}{3} & \multicolumn{1}{l|}{0} \\ \cline{7-8} 
 &  & TIME – 2 hours \& 35 minutes              &  &                                            & \multicolumn{1}{r}{}            &                        &                        \\
 &  & FULL MARKS – 50                           &  &                                            & \multicolumn{1}{r}{\textbf{}}   &                        &                       
\end{tabular}
\end{table}
\fi

\begin{table}[]
\begin{tabular}{llllllllcllcl}
\textit{Canopus} &  &  &  &  &  &  &  & \textbf{MYMENSINGH GIRLS’ CADET COLLEGE} &                                             &                                 & \multicolumn{1}{l}{}            &                        \\
                 &  &  &  &  &  &  &  & THIRD TERM-END EXAMINATION - 2025          &                                             &                                 &                                 &                        \\
                 &  &  &  &  &  &  &  & CLASS: XI                               &                                             &                                 &                                 &                        \\
                 &  &  &  &  &  &  &  & STATISTICS (CREATIVE)                    &                                             &                                 &                                 &                        \\
                 &  &  &  &  &  &  &  & FIRST PAPER                             &                                             &                                 & \multicolumn{1}{r}{}            &                        \\ \cline{11-13} 
                 &  &  &  &  &  &  &  & [According to the Syllabus of 2026]      & \multicolumn{1}{r|}{\textbf{Subject Code:}} & \multicolumn{1}{l|}{\textbf{1}} & \multicolumn{1}{l|}{\textbf{2}} & \multicolumn{1}{l|}{9} \\ \cline{11-13} 
                 &  &  &  &  &  &  &  & TIME – 2 hours \& 35 minutes             &                                             &                                 & \multicolumn{1}{r}{}            &                        \\
                 &  &  &  &  &  &  &  & FULL MARKS – 50                          &                       
\end{tabular}
\end{table}

\vspace{-1cm}
\hrule

\begin{center}
[\textbf{N.B.} – The figures of the right margin indicate full marks. Read the stems carefully and answer the associated questions. Answer any \textbf{FIVE} questions taking at least two questions from each group]\\
\end{center}

\begin{center}
\textbf{Group  - A}
\end{center}
\begin{enumerate}
\item
\textbf{The weekly sales (in hundreds of dollars) for a retail store over six weeks are recorded as 12, 15, 18, 20, 22, and 25 (denoted by \( y \)). The store manager claimed that the square of total sales exceeds the total of squared sales.}

\begin{enumerate}
\item Is the type of blood group (A, B, AB, O) a qualitative or 
quantitative variable?\hfill 1

\item Differentiate between ratio and interval scale. \hfill 2
    \item
    Calculate $\displaystyle \sum_{i=1}^6 (y_i - 3y_i)^2$ using the provided data. \hfill 3
    
    \item
    Determine whether the manager’s claim is correct based on the data. \hfill 4
\end{enumerate}

  \item
  \textbf{Income and expenditure (both in thousands) of some individuals in four successive months are collected:}
 
\begin{table}[h]
 \begin{center}
\begin{tabular}{l|l|l|l|l}

Income (x)  & 20 & 30 & 25 & 10 \\ \hline
Expenditure (y) & 15  & 27  & 18 & 5 \\ 
\end{tabular}
\end{center}
\end{table}


  \begin{enumerate}
    \item
	What is a discrete variable? \hfill 1
    \item
    	Can fractional numbers be discrete? Explain briefly.  \hfill 2
    	     \item
     	Find $\displaystyle \sum_{i=1}^4 (x_i)(y_i-3)$.
    \item
    	Are, in the stem, $\displaystyle \sum_{i=1}^{n}
    	\sum_{i=1}^{n} x_iy_j = \sum_{i=1}^{n} x_iy_i?$ Vindicate. \hfill 4
 \hfill 4
  \end{enumerate}
  
    \item
  \textbf{The ages of 20 participants in a fitness program were recorded and found to be as follows:}
  \begin{center}
  25, 30, 28, 35, 40, 38, 26, 32, 36, 31 \\
  27, 33, 29, 41, 42, 37, 34, 39, 43, 45 \\
  \end{center}

  
  \begin{enumerate}
  \item What does the width of the bins represent in a histogram? \hfill 1
    \item Relate histogram and stem and leaf plot \hfill 2
    \item  
	 Create a frequency distribution and interpret. \hfill 3
    \item
	Create a Histogram from the data and explain. If the no. of 
	classes were fewer, how would the \\ pattern of the distribution shift? \hfill 4
  \end{enumerate}

  \begin{center}
\textbf{Group  - B}
\end{center}

    \item
	  \textbf{A passer-by walks 3 hours at 5 km per hour (kph), another 3 hours 
	  at 4 kph, and another 3 hours at 3 kph.} 
  
  \begin{enumerate}
    \item
	When is harmonic mean suitable? \hfill 1
    \item
	Which mean could we use for the given data and why? \hfill 2
    \item  
	Find the average speed of the passer-by usingt he proper method. \hfill 3
    \item
	Find the correct and suitable average speed using another method and mathematically show they are equivalent. \hfill 4
  \end{enumerate}

\item  
  \textbf{The following table presents the distribution of monthly salaries (in thousand BDT) of employees in two different departments of a company.}  

\begin{table}[H]
\centering
\begin{tabular}{ccc}
\hline
Salary Range (in 1000 BDT) & \begin{tabular}[c]{@{}c@{}}Number of Employees\\ Department - X\end{tabular} & \begin{tabular}[c]{@{}c@{}}Number of Employees\\ Department - Y\end{tabular} \\ \hline
20-25 & 8 & 6 \\ 
25-30 & 14 & 12 \\ 
30-35 & 19 & 21 \\ 
35-40 & 24 & 26 \\ 
40-45 & 15 & 10 \\ \hline
\end{tabular}
\end{table}

  \begin{enumerate}
  \item what is the relationship among AM, GM, and HM? \hfill 1
      \item If $\bar{X} = 3 , \text{ and } n = 10$, what is $\sum X_i$? \hfill 2
    \item  
	Determine the arithmetic mean salary of employees in Department - X.  \hfill 3  
    \item  
	Compute the combined mean salary. Is it higher than the arithmetic mean of Department - Y? Justify your answer with a statistical explanation.  \hfill 4  
\end{enumerate}  

  
      \item
  \textbf{Temperatures of two cold regions for five days are as below:}

    \begin{center}

    City A: 2, 1, -1, 0, 3

    City B: 3, 0, -2, 2, 3
    
    \end{center}
  \begin{enumerate}
    \item
	What is dispersion? \hfill 1
    \item
	Illustrate the necessity of dispersion with an example. \hfill 2
    \item  
	Find the variance of temperature of city A.  \hfill 3
    \item
	Which city has more consistent temperature? Analyze.\hfill 4
\end{enumerate}

\end{enumerate}
 \vspace{2.5cm}

\begin{center}

  \vfill
  --AAM--
\end{center}


\newpage
\setcounter{page}{1}

\begin{table}[]
\begin{tabular}{llllllllcllcl}
\textit{Canopus} &  &  &  &  &  &  &  & \textbf{MYMENSINGH GIRLS’ CADET COLLEGE} &                                             &                                 & \multicolumn{1}{l}{}            &                        \\
                 &  &  &  &  &  &  &  & THIRD TERM-END EXAMINATION - 2025          &                                             &                                 &                                 &                        \\
                 &  &  &  &  &  &  &  & CLASS: XI                               &                                             &                                 &                                 &                        \\
                 &  &  &  &  &  &  &  & STATISTICS (CREATIVE)                    &                                             &                                 &                                 &                        \\
                 &  &  &  &  &  &  &  & FIRST PAPER                             &                                             &                                 & \multicolumn{1}{r}{}            &                        \\ \cline{11-13} 
                 &  &  &  &  &  &  &  & [According to the Syllabus of 2026]      & \multicolumn{1}{r|}{\textbf{Subject Code:}} & \multicolumn{1}{l|}{\textbf{1}} & \multicolumn{1}{l|}{\textbf{2}} & \multicolumn{1}{l|}{9} \\ \cline{11-13} 
                 &  &  &  &  &  &  &  & TIME – 2 hours \& 35 minutes             &                                             &                                 & \multicolumn{1}{r}{}            &                        \\
                 &  &  &  &  &  &  &  & FULL MARKS – 50                          &                       
\end{tabular}
\end{table}


\hrule

\begin{center}
[\textbf{N.B.} – The figures of the right margin indicate full marks. Read the stems carefully and answer the associated questions. Answer any \textbf{FIVE} questions taking at least two questions from each group]\\
\end{center}

\begin{center}
\textbf{Group  - A}
\end{center}
\begin{enumerate}

 \item
	  \textbf{Below are some information, with two variables x and y:} 
  

  \begin{table}[h]
\centering
\begin{tabular}{c|c|c|c|c}
$x_i$ & 3 & 4 & 1 & 0 \\ \hline
$y_i$ & 1 & 5 & 0 & 2 \\
\end{tabular}
\end{table}

  
  \begin{enumerate}
    \item
	What is a qualitative variable? \hfill 1
    \item
	Find $\displaystyle \sum_{i=1}^{4}x_i^2$ \hfill 2
    \item  
	Prove that $\displaystyle \sum_{i=1}^{4} (x_i+y_i) = \sum_{i=1}^{4}x_i + \sum_{i=1}^{4}y_i $ \hfill 3
    \item
	Find the value of $\displaystyle \sum_{i=1}^{4} x_iy_i-\sum_{i=1}^{4} x_i+4$ \hfill 4

  \end{enumerate}
  
    \item
\textbf{A health researcher recorded the daily steps (in thousands) of five individuals over a week:}

\begin{center}
$y_1 = 8,\ y_2 = 5,\ y_3 = 12,\ y_4 = 10,\ y_5 = 7$
\end{center}

\begin{enumerate}
\item What is ordinal data? \hfill 1

\item Explain change of origin and scale with an example. \hfill 2

    \item
    Compute $\displaystyle \sum_{i=1}^5 (y_i - 10)^2$. \hfill 3
    
    \item
    Evaluate $\displaystyle \sum_{i=1}^5 (2y_i^2 - 3y_i + 4)$ and comment on how shifting the origin would affect this sum. \hfill 4
\end{enumerate}

     \item
	  \textbf{The following table tracks the number of individuals who sleep 
	  within specific hourly intervals. }
	  
\begin{table}[h]
  \centering
\begin{tabular}{c|c|c|c|c|c|c|c}
Hours of Sleep (per night) & 4-5   & 5-6   & 6-7   & 7-8   & 8-9   & 9-10  & 10+   \\ \hline
Number of Individuals      & 12    & 20    & 25    & 30    & 18    & 8     & 7     
\end{tabular}
\end{table}


  \begin{enumerate}
  \item What is the purpose of a frequency distribution? \hfill 1
    \item Relate histogram and stem and leaf plot \hfill 2
    \item  
	Draw an Ogive from the data provided and explain. \hfill 3
    \item
	Write five useful insights about the data combining information from 
	the Ogive and the table. \hfill 4
  \end{enumerate}  
  
  

\begin{center}
\textbf{Group  - B}
\end{center}
  
    \item
  \textbf{In the test examination, marks of 11 students in statistics are: 90, 92, 93, 49, 44, 88, 80, 58, 83, 71, 76.}
  \begin{enumerate}
    \item
	What is central tendency? \hfill 1
    \item
	When is median better than arithmetic mean? Explain with an example. \hfill 2
    \item  
	Find the 3rd the quartile and $61^{st}$ percentile from the data and explain.  \hfill 3
    \item
	Do quantiles depend on change of origin and scale. Prove using two examples.\hfill 4
\end{enumerate}
  
   \item
	  \textbf{A student walks 3 hours at 5 km per hour (kph), 4 hours at 4 kph, and 2 hours at 3 kph.} 
  
  \begin{enumerate}
    \item
	When is harmonic mean suitable? \hfill 1
    \item
	Which mean could we use for the given data and why? \hfill 2
    \item  
	Find the average speed using weighted harmonic mean. \hfill 3
    \item
	Find the correct and suitable average speed using another method and mathematically show they are equivalent. \hfill 4
  \end{enumerate}
  
\item
\textbf{Rainfall measurements (in mm) of two cities for six days are as below:}

\begin{table}[h]
\centering
\begin{tabular}{c|c|c|c|c|c|c}
Day & 1 & 2 & 3 & 4 & 5 & 6 \\
\hline
City X & 15 & 12 & 10 & 18 & 14 & 11 \\ \hline
City Y & 20 & 8 & 25 & 5 & 22 & 10 \\
\end{tabular}
\end{table}

\begin{enumerate}
    \item
    What is a measure of dispersion? \hfill 1
    
    \item
    Explain with an example why measuring dispersion is important in data analysis. \hfill 2
    
    \item
    Calculate the variance of rainfall for City X. \hfill 3
    
    \item
    Which city shows more consistent rainfall? Justify. \hfill 4
\end{enumerate}
  
\end{enumerate}

 \vspace{2.5cm}

\begin{center}
  \vfill
  --AAM--
\end{center}

\end{document}