\documentclass[12pt]{article}
\usepackage{geometry}
\usepackage{amsfonts}
\usepackage{float}

% HIde Footnotes

\usepackage{letltxmacro}

\LetLtxMacro\Oldfootnote\footnote

\newcommand{\EnableFootNotes}{%
  \LetLtxMacro\footnote\Oldfootnote%
}

\newcommand{\DisableFootNotes}{%
  \renewcommand{\footnote}[2][]{\relax}
}
% HIde Footnotes

\geometry{
legalpaper, total={177.8mm, 290mm},left=20mm,
top=7mm, bottom=27mm,
}

\begin{document}

\begin{table}[]
\begin{tabular}{llllllcllrlll}
\textit{Coefficient} &  &  &  &  &  & \textbf{SYLHET CADET COLLEGE}       &  &                       & \multicolumn{1}{l}{}                        &                        &                        &                        \\ \cline{10-13} 
                 &  &  &  &  &  & PROGRESS TEST EXAMINATION - 2025             &  & \multicolumn{1}{l|}{} & \multicolumn{1}{c|}{Ques Setter}            & \multicolumn{3}{l|}{}                                                    \\ \cline{10-13} 
                 &  &  &  &  &  & CLASS: XII                          &  & \multicolumn{1}{l|}{} & \multicolumn{1}{c|}{Moderator}              & \multicolumn{3}{l|}{}                                                    \\ \cline{10-13} 
                 &  &  &  &  &  & STATISTICS (CREATIVE)               &  & \multicolumn{1}{l|}{} & \multicolumn{1}{c|}{VP}                     & \multicolumn{3}{l|}{}                                                    \\ \cline{10-13} 
                 &  &  &  &  &  & FIRST PAPER                        &  &                       &                                             &                        &                        &                        \\ \cline{11-13} 
                 &  &  &  &  &  & [According to the Syllabus of 2025] &  &                       & \multicolumn{1}{r|}{\textbf{Subject Code:}} & \multicolumn{1}{l|}{1} & \multicolumn{1}{l|}{2} & \multicolumn{1}{l|}{9} \\ \cline{11-13} 
                 &  &  &  &  &  & TIME – 2 hours \& 35 minutes        &  &                       &                                             &                        &                        &                        \\
                 &  &  &  &  &  & FULL MARKS – 50                     &  &                       & \textbf{}                                   &                        &                        &                       
\end{tabular}
\end{table}

\hrule

\DisableFootNotes

\begin{center}
[\textbf{N.B.} – The figures of the right margin indicate full marks. Read the stems carefully and answer the associated questions. Answer any \textbf{FIVE} questions taking at least two questions from each group]\\
\end{center}

\begin{center}
\textbf{Group  - A}
\end{center}

  \begin{enumerate}
  
  \item
\textbf{The monthly sales (in thousands of units) recorded by a store over 
five months are 120, 135, 150, 160, and 175. The store manager 
stated that the square of the total sales is greater than the total of 
the squared sales.}

\begin{enumerate}
    \item  What is change of origin? \hfill 1
    \item 
    If the scores of five students in a test are 78, 85, 92, 88, 95, 
  find $\displaystyle \sum_{i=1}^5 (x_i^2 - 2x_i + 3)$ \hfill 2
   \item Calculate $\displaystyle \sum_{i=1}^5 (x_i - 1.5x_i)^2$ using the 
    provided data. \hfill 3
    \item
    Assess whether the manager’s statement is accurate based on the data. \hfill 4
\end{enumerate}

  \item
\textbf{The monthly sales and expenses (in thousand BDT) of five retail stores are given below:}

\begin{table}[h]
\centering
\begin{tabular}{c|ccccc}
Store & A & B & C & D & E \\ \hline
Sales (X) & 50 & 65 & 40 & 70 & 55 \\ \hline
Expenses (Y) & 30 & 45 & 25 & 50 & 35
\end{tabular}
\end{table}

\begin{enumerate}
\item What is univariate data? \hfill 1
\item Differentiate between discrete and continuous variable. \hfill 2
    \item 
    Calculate $\displaystyle \sum_{i=1}^5 (x_i + y_i)$ \hfill 3
    \item 
    Verify whether the statement $\displaystyle \sum_{i=1}^5 (3x_i - 2y_i) = 3 \sum_{i=1}^5 x_i - 2 \sum_{i=1}^5 y_i$ holds true. \hfill 4
\end{enumerate}


\item
\textbf{Average marks of two sections A \& B in a statistics exam are 68 and 74 respectively. The overall average mark of both sections combined is 70. Section A has 25 students.}

\begin{enumerate}
    \item  
    How many students are there in section B? \hfill 3
    \item
    Later, it was found that a student in section B was wrongly marked 80 instead of 90. Find the corrected average of section B and the new combined average. \hfill 4
\end{enumerate}

  
%  \newpage
  
   \item
	  \textbf{A meteorologist records the monthly rainfall (in mm) in different regions over a year as shown below:}

\begin{table}[h]
\centering
\begin{tabular}{c|c|c|c|c|c}
\textbf{Rainfall (mm)} & 0-50 & 50-100 & 100-150 & 150-200 & 200-250 \\ \hline
\textbf{Frequency}     & 6    & 10     & 14      & 8       & 7       
\end{tabular}
\end{table}

  
  \begin{enumerate}
    \item
	Which class contains the Mode? \hfill 1
    \item
	Find $\Delta_1$ and $\Delta_2$, where the symbols represent their usual meanings \hfill 2
    \item  
	Find the Mode using the Direct formula. \hfill 3
    \item
	Find the Mode using histogram and compare with direct method. \\ Which one do you think is more accurate? \hfill 4
  \end{enumerate}


\begin{center}
\textbf{Group  - B}
\end{center}

  \item
\textbf{A data set represents the test scores of 10 students in a recent exam, recorded as follows:}

\begin{center}
56, 62, 68, 71, 65, 59, 74, 67, 70, 63
\end{center}

\begin{enumerate}
 \item How many types of moments are there?  \hfill 1
  \item 	Derive the value of thew first central moment. \footnote{$\frac{\sum(x_i-\bar x}{n} = \bar x - \bar x) = 0$} \hfill 2
\item
Calculate the first four moments about 3. \hfill 3
\item
Compute the variance and kurtosis of the data using converted central moments. 
Explain what the kurtosis indicates about the distribution. \hfill 4
\end{enumerate}

  \item  
  \textbf{A financial analyst is studying the annual returns (in percentage) of a set of investment portfolios. The following table summarizes the data.}  

\begin{table}[h]
\centering
\begin{tabular}{c|cccccc}
Annual Return (\%) & -5 to 0 & 1-5 & 6-10 & 11-15 & 16-20 & 21-25 \\ \hline
Frequency & 3 & 5 & 7 & 6 & 4 & 3
\end{tabular}
\end{table}

  \begin{enumerate}
   \item What is negative skewness?\hfill 1
    \item In a right-skewed distribution, how are Mean, Median, and Mode related?  \hfill 2
    \item  
	Compute the skewness of the data and interpret the nature of the investment returns.  \hfill 3  
    \item  
	Determine the kurtosis and explain.  \hfill 4  
  \end{enumerate}  
  
  \item
\textbf{The monthly sales (in units) of laptops at a major electronics store are recorded below:} 

\begin{table}[h]
\centering
\begin{tabular}{c|cccccccc}
\textbf{Month} & Jan & Feb & Mar & Apr & May & Jun & Jul & Aug \\ \hline
\textbf{Sales} & 120 & 150 & 130 & 170 & 160 & 180 & 200 & 210
\end{tabular}
\end{table}

\begin{enumerate}
 \item Give an example of irregular variation? \hfill 1
 \item What are the limitations of semi-average method? \hfill 2
  \item  
  Compute the trend using a three-monthly moving average method. \hfill 3
  \item
  Illustrate the trend graphically and estimate the expected laptop sales for September. \hfill 4
\end{enumerate}

\item
\textbf{The classification of published statistics is essential for organizing data into meaningful categories. This helps in better understanding and utilization of the data for research and policy-making.}

\begin{enumerate}
 \item What is non-official statistics?  \hfill 1
  \item What are the limitations of official statistics? \hfill 2
    \item  
    Explain the classification system used for published statistics in Bangladesh. \hfill 3
    \item
    Evaluate the effectiveness of this classification system in meeting the needs of researchers and policymakers. \hfill 4
\end{enumerate}
  
\end{enumerate}

 \vspace{2.5cm}

%\begin{center}
%“It is easy to lie with statistics; it is easier to lie without them.” - Frederick Mosteller
%\end{center}

\end{document}