\documentclass{article}
\usepackage{geometry}
\usepackage{amsfonts}

\geometry{
legalpaper, total={177.8mm, 290mm},left=20mm,
top=7mm, bottom=27mm,
}

\begin{document}

\begin{table}[h]
\centering
\begin{tabular}{lllll}
\textbf{\large SYLHET CADET COLLEGE} &  &  &  &  \\ \cline{4-5} 
FIRST TERM-END EXAMINATION - 2024 &  & \multicolumn{1}{l|}{} & 
\multicolumn{1}{l|}{Set} & \multicolumn{1}{l|}{A} \\ \cline{4-5} 
CLASS: XII &  &  &  &  \\ \cline{3-5} 
STATISTICS (CREATIVE)& \multicolumn{1}{l|}{\textbf{Subject Code:}} & 
\multicolumn{1}{l|}{1} & \multicolumn{1}{l|}{2} & \multicolumn{1}{l|}{9} \\ \cline{3-5} 
 FIRST PAPER &  &  &  &  \\
TIME – 2 hours \& 35 minutes &  &  &  &  \\
FULL MARKS – 50 &  &  &  & 
\end{tabular}
\end{table}
%  \normalfont\normalsize
 % 11.45a.m.~--~1.45p.m.

\hrule

\begin{center}
[\textbf{N.B.} – The figures of the right margin indicate full marks. Read 
the stems carefully and answer the associated questions. Answer any
\textbf{FIVE} questions taking at least two questions from each group]\\
\end{center}

\begin{center}
\textbf{Group  - A}

\end{center}
  \begin{enumerate}

  \item
\textbf{A software developer tracked the response times (in milliseconds) 
of a web application under test during peak usage hours. An unexpected 
delay of 1.2 ms was added to each recorded response time due to server
lag. The recorded times are as follows:}

\begin{center}
45, 48, 52, 50, 47, 53, 60, 55, 58, 49
\end{center}

\begin{enumerate}
\item What is nominal scale of measurement?

\item 
If the ages of a group of people are 22, 25, 28, 30, and 35, 
find $\displaystyle \sum_{i=1}^5 (x_i^2 + 3x_i)$ \hfill 2
    \item  
    Calculate $\displaystyle \sum_{i=1}^{10} (X_i - 25)$ from the stem. \hfill 3
    \item
    Find the sum of the original response times before the lag was added.
    \hfill 4
\end{enumerate}

  \item
\textbf{Concentrations of a chemical solution (in mol/L) were recorded over 
several trials as follows:} 

\begin{table}[h]
\centering
\begin{tabular}{c|ccccc}
\textbf{Concentration (mol/L)} & 0.1-0.2 & 0.2-0.3 & 0.3-0.4 & 0.4-0.5 & 0.5-0.6
\\ \hline
\textbf{Frequency}             & 4       & 7       & 5       & 6       & 3       
\end{tabular}
\end{table}

\begin{enumerate}
\item If $u_i = x_i + y_i$, what us $\bar x$ in terms of $u$? \hfill 1
\item Find Arithmetic Mean: $14,18,22, \cdots 70$ \hfill 2
    \item  
    Compute the Arithmetic Mean of the distribution given using 
    the Short-cut method. \hfill 3
    \item
    Compute the Arithmetic Mean using a different assumed mean (A). \\
    Do both methods yield the same result? \hfill 4
\end{enumerate}


\item
\textbf{A botanist measures the heights (in cm) of plants from a sample as 
shown below:}

\begin{table}[h]
\centering
\begin{tabular}{c|c}
\textbf{Height (cm)} & \textbf{Frequency} \\ \hline
20-30                & 6                  \\ \hline
30-40                & 10                 \\ \hline
40-50                & 12                 \\ \hline
50-60                & 8                  \\ \hline
60-70                & 4                  
\end{tabular}
\end{table}

\begin{enumerate}
\item Is Median affected by outliers?
\item Does Median depend on origin and scale? Prove.
    \item  
    Find the median height of the plants and interpret. \hfill 3
    \item
    Determine the first (Q1) and third (Q3) quartiles of the plant heights. \\
    Explain the significance of these quartiles in understanding the data 
    distribution. \hfill 4
\end{enumerate}

  \item
\textbf{A psychologist is studying the stress levels (measured on a scale 
of 1 to 50) experienced by five participants during a specific task. 
The observed values are:}
\begin{center}
$x_1 = 32, x_2 = 28, x_3 = 40, x_4 = 35, x_5 = 22$
\end{center}
\begin{enumerate}
\item Is the brand of a smartphone (e.g., Apple, Samsung, etc.) a 
qualitative or quantitative variable?\hfill 1
\item After expansion, what does $\displaystyle \sum_{i=1}^n 
\left( ax_i-b \right)$ become?
    \item
    Compute the value of $\displaystyle \sum_{i=1}^5 (x_i - 30)^2$ from the 
    stem. \hfill 3
    \item
    Calculate $\displaystyle \sum_{i=1}^5 (2x_i^2 - 5x_i + 4)$ using both a 
    direct approach and by splitting the summation terms. \hfill 4
\end{enumerate}
 
  \begin{center}
\textbf{Group  - B}
\end{center}
  
  \item
\textbf{A study was conducted to track the daily water consumption (in liters) 
of 10 individuals over a week. The recorded values are as follows:}

\begin{center}
2.5, 3.1, 2.8, 3.5, 2.7, 3.3, 2.6, 3.0, 2.9, 3.4
\end{center}

\begin{enumerate}

  \item What is central moment? \hfill 1
 \item Can moments be negative? Analyze. \hfill 2
\item
Determine the variance of the data set. \hfill 3
\item
Assess whether the data distribution appears to be symmetric with the help
two different methods. \hfill 4
\end{enumerate}

  \item
\textbf{A data set represents the test scores of 10 students in a recent 
exam, recorded as follows:}

\begin{center}
56, 62, 68, 71, 65, 59, 74, 67, 70, 63
\end{center}

    \begin{enumerate}
    \item How many types of kurtosis are there? \hfill 1
    \item What is the pattern of in a left-skewed didstribution? \hfill 2
\item
Calculate the first four moments about 3. \hfill 3
\item
Compute the variance and kurtosis of the data using converted central moments. 
Explain what the kurtosis indicates about the distribution. \hfill 4
\end{enumerate}

\item
\textbf{The quarterly production data (in tons) for a factory is given below:}

\begin{table}[h]
\centering
\begin{tabular}{cccccccc}
Quarter    & Q1     & Q2     & Q3     & Q4     & Q5     & Q6     & Q7     \\ \hline
Production & 150    & 160    & 155    & 145    & 170    & 165    & 180    \\
\end{tabular}
\end{table}

    \begin{enumerate}
     \item Give an example of irregular variation. \hfill 1
     \item Mention the methods of measuring the trend. \hfill 2
\item
Calculate the trend using the moving average method for a 3-quarter 
period. \hfill 3
\item
Plot the trend line and predict the production for Q8 using two methods 
and compare. \hfill 4
\end{enumerate}

\item
\textbf{Government organizations rely on statistical data to create policies 
and allocate resources efficiently. In some countries, statistical data on 
health, education, and economic performance is used by policymakers to make 
decisions. However, the misuse of data can lead to ineffective or 
harmful policies.}

\begin{enumerate}
    \item What does BBS stand for? \hfill 1
    \item Differentiate between official and non-official statistics. \hfill 2
    \item  
    Define the scope of official statistics in policy-making and the role 
    they play in resource allocation. \hfill 3
    \item
    Discuss the possible consequences of misusing statistical data in 
    policymaking, providing an example from a real-world situation. \hfill 4
\end{enumerate}
  
\end{enumerate}
\end{document}