\documentclass{article}
\usepackage{geometry}
\usepackage{amsfonts}

\geometry{
legalpaper, total={177.8mm, 290mm},left=20mm,
top=13mm, bottom=15mm,
}

\begin{document}

\begin{table}[h]
\centering
\begin{tabular}{lllll}
\textbf{\large SYLHET CADET COLLEGE} &  &  &  &  \\ \cline{4-5} 
YEAR-FINAL EXAMINATION - 2024 &  & \multicolumn{1}{l|}{} & \multicolumn{1}{l|}{Set} & \multicolumn{1}{l|}{A} \\ \cline{4-5} 
CLASS: XI &  &  &  &  \\ \cline{3-5} 
STATISTICS (CREATIVE)& \multicolumn{1}{l|}{\textbf{Subject Code:}} & \multicolumn{1}{l|}{1} & \multicolumn{1}{l|}{2} & \multicolumn{1}{l|}{9} \\ \cline{3-5} 
 FIRST PAPER &  &  &  &  \\
TIME – 2 hours \& 20 minutes &  &  &  &  \\
FULL MARKS – 50 &  &  &  & 
\end{tabular}
\end{table}
%  \normalfont\normalsize
 % 11.45a.m.~--~1.45p.m.

\hrule

\begin{center}
[\textbf{N.B.} – The figures of the right margin indicate full marks. Read the stems carefully and answer the associated questions. Answer any \textbf{FIVE} questions taking at least two from each group.]\\

\end{center}

  \begin{center}
  \textbf{Group--A}
  \end{center}
    \begin{enumerate}
    
       \item
	  \textbf{The capital and profit (in million BDT) of some Bangladeshi industries are bgiven below:}
	  
	  \begin{table}[h]
	  \centering
\begin{tabular}{c|ccccc}
Industry & 1 & 2 & 3 & 4 & 5 \\ \hline
Capital (X) & 20 & 15 & 26 & 31 & 18 \\ \hline
Profit (Y) & 15 & 10 & 17 & 25 & 10
\end{tabular}
\end{table}
  
  \begin{enumerate}
    \item
	What is finite population? \hfill 1
    \item
	What are the functions of statistics? \hfill 2
    \item  
	Find the value of $\displaystyle \sum_{i=1}^5 \sum_{j=1}^5 (x_i - y_j)$ \hfill 3
    \item
	Analyze the statement theoretically and empirically:  $\displaystyle \sum_{i=1}^5 (4x_i-6y_j) = 4 \sum_{i=1}^5 x_i - 6 \sum_{i=1}^5 y_j $ \hfill 4
  \end{enumerate}
  
  
 \item
	  \textbf{Marks obtained by five students in statisics out of 15 were 4, 6, 10, 12, and 15. The examiner said, the square of the sum of the marks is greater than the sum of the squares of the marks.} 
  
  \begin{enumerate}
    \item
	What is finite population? \hfill 1
    \item
	Explain quantitative variable with an example. \hfill 2
    \item  
	In the light of the available data, find $\displaystyle \sum_{i=1}^5 (x_i-2x_i)^2$ \hfill 3
    \item
	Verify the comment of the examiner. \hfill 4
  \end{enumerate}
  
     \item
	  \textbf{A passer-by walks 3 hours at 5 km per hour (kph), another 3 hours at 4 kph, and another 3 hours at 3 kph.} 
  
  \begin{enumerate}
    \item
	When is harmonic mean suitable? \hfill 1
    \item
	Which mean could we use for the given data and why? \hfill 2
    \item  
	Find the average speed of the passer-by usingt he proper method. \hfill 3
    \item
	Find the correct and suitable average speed using another method and mathematically show they are equivalent. \hfill 4
  \end{enumerate}

    \item
  \textbf{In the test examination, marks of 11 students in statistics are: 90, 92, 93, 49, 44, 88, 80, 58, 83, 71, 76.}
  \begin{enumerate}
    \item
	What is central tendency? \hfill 1
    \item
	When is median better than arithmetic mean? Explain with an example. \hfill 2
    \item  
	Find the 3rd the quartile and $61^{st}$ percentile from the data and explain.  \hfill 3
    \item
	Do quantiles depend on change of origin and scale. Prove using two examples.\hfill 4
\end{enumerate}


  

    \begin{center}
  \textbf{Group--B}
  \end{center}
  
     \item
	  \textbf{There has been an increase in average lifetime of people of Bangladesh. To get more insight on this, a research was conducted, in which ages of retired government employees were recorded. A sample of 10 people is given below:}
	  
	  \begin{center}
	  75, 62, 63, 72, 66, 76, 59, 77, 70, 79
	  \end{center}
    \begin{enumerate}
    \item
	What is the 2nd central moment? \hfill 1
    \item
	Show that the first central moment is zero. \hfill 2
    \item  
	Find the variance of the data. \hfill 3
    \item
	Are the data symmetric? Justify. \hfill 4
  \end{enumerate}
  
       \item
	  \textbf{Bangladesh foreign debt has been increasing rapidly in recent years. The Bangladesh bank provides the follwoing data.}
	  
	  \begin{table}[h]
	  \centering
\begin{tabular}{c|c|c|c|c|c|c|c|c|c}
Fiscal Year & 2015-16 & 2016-17 & 2017-18 & 2018-19 & 2019-20 & 2020-21 & 2021-22 & 2022-23 & 2023-24 \\ \hline
Debt & 41.17 & 45.81 & 56.01 & 62.63 & 68.55 & 81.62 & 95.45 & 98.94 & $\sim$130.00
\end{tabular}
\end{table}
  
  \begin{enumerate}
    \item
	Name the components of time series. \hfill 1
    \item
	What are linear and non-linear trends? \hfill 2
    \item  
	Find 3-yearly moving avergae from the data and plot. \hfill 3
    \item
	Whaich components of time series may underlie the data? Analyze. \hfill 4
  \end{enumerate}
  
  
  
     \item
	  \textbf{Every country has one or more agencies to deal with statistics of the country for proper management of its assets and population. Bangladesh Bureau of Statistics (BBS) serves as the centralized official bureau in Bangladesh for collecting and disseminating statistics in Bangladesh. USA has several such agencies, like Census Bureau or Bureau of Labor Statistics.} 
  
  \begin{enumerate}
    \item
	What is data? \hfill 1
    \item
	How is statistics important in planning?\hfill 2
    \item  
	Differentiate between official and non-official statistics. \hfill 3
    \item
	Elucidate the classification of published statistics in Bangladesh.  \hfill 4
  \end{enumerate}

\end{enumerate}

 \vspace{2.5cm}

\begin{center}

\end{center}

\end{document}