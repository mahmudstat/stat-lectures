\documentclass[12pt]{exam}
%\documentclass[11pt,a4paper]{exam}
\usepackage{amsmath,amsthm,amsfonts,amssymb,dsfont}
\usepackage{float}
\usepackage{ifthen}
\usepackage{array}
\usepackage{geometry}
\geometry{
legalpaper, total={177.8mm, 290mm},left=20mm,
top=7mm, bottom=27mm,
}
\usepackage{enumerate}% http://ctan.org/pkg/enumerate
\usepackage{multicol}
\usepackage{hhline}
\usepackage[table]{xcolor}


% Accumulate the answers. Unmodified from Phil Hirschorn's answer
% https://tex.stackexchange.com/questions/15350/showing-solutions-of-the-questions-separately/15353
\newbox\allanswers
\setbox\allanswers=\vbox{}

\newenvironment{answer}
{%
    \global\setbox\allanswers=\vbox\bgroup
    \unvbox\allanswers
}%
{%
    \bigbreak
    \egroup
}

\newcommand{\showallanswers}{\par\unvbox\allanswers}
% End Phil's answer


% Is there a better way?
\newcommand*{\getanswer}[5]{%
    \ifthenelse{\equal{#5}{a}}
    {\begin{answer}\thequestion. (a)~#1\end{answer}}
    {\ifthenelse{\equal{#5}{b}}
        {\begin{answer}\thequestion. (b)~#2\end{answer}}
        {\ifthenelse{\equal{#5}{c}}
            {\begin{answer}\thequestion. (c)~#3\end{answer}}
            {\ifthenelse{\equal{#5}{d}}
                {\begin{answer}\thequestion. (d)~#4\end{answer}}
                {\begin{answer}\textbf{\thequestion. (#5)~Invalid answer choice.}\end{answer}}}}}
}

\setlength\parindent{0pt}
%usage \choice{ }{ }{ }{ }
%(A)(B)(C)(D)
\newcommand{\fourch}[5]{
    \par
    \begin{tabular}{*{4}{@{}p{0.23\textwidth}}}
        (a)~#1 & (b)~#2 & (c)~#3 & (d)~#4
    \end{tabular}
    \getanswer{#1}{#2}{#3}{#4}{#5}
}

%(A)(B)
%(C)(D)
\newcommand{\twoch}[5]{
    \par
    \begin{tabular}{*{2}{@{}p{0.46\textwidth}}}
        (a)~#1 & (b)~#2
    \end{tabular}
    \par
    \begin{tabular}{*{2}{@{}p{0.46\textwidth}}}
        (c)~#3 & (d)~#4
    \end{tabular}
    \getanswer{#1}{#2}{#3}{#4}{#5}
}

%(A)
%(B)
%(C)
%(D)
\newcommand{\onech}[5]{
    \par
    (a)~#1 \par (b)~#2 \par (c)~#3 \par (d)~#4
    \getanswer{#1}{#2}{#3}{#4}{#5}
}

\newlength\widthcha
\newlength\widthchb
\newlength\widthchc
\newlength\widthchd
\newlength\widthch
\newlength\tabmaxwidth

\setlength\tabmaxwidth{0.96\textwidth}
\newlength\fourthtabwidth
\setlength\fourthtabwidth{0.25\textwidth}
\newlength\halftabwidth
\setlength\halftabwidth{0.5\textwidth}

\newcommand{\choice}[5]{%
\settowidth\widthcha{AM.#1}\setlength{\widthch}{\widthcha}%
\settowidth\widthchb{BM.#2}%
\ifdim\widthch<\widthchb\relax\setlength{\widthch}{\widthchb}\fi%
    \settowidth\widthchb{CM.#3}%
\ifdim\widthch<\widthchb\relax\setlength{\widthch}{\widthchb}\fi%
    \settowidth\widthchb{DM.#4}%
\ifdim\widthch<\widthchb\relax\setlength{\widthch}{\widthchb}\fi%

% These if statements were bypassing the \onech option.
% \ifdim\widthch<\fourthtabwidth
%     \fourch{#1}{#2}{#3}{#4}{#5}
% \else\ifdim\widthch<\halftabwidth
% \ifdim\widthch>\fourthtabwidth
%     \twoch{#1}{#2}{#3}{#4}{#5}
% \else
%      \onech{#1}{#2}{#3}{#4}{#5}
%  \fi\fi\fi}

% Allows for the \onech option.
\ifdim\widthch>\halftabwidth
    \onech{#1}{#2}{#3}{#4}{#5}
\else\ifdim\widthch<\halftabwidth
\ifdim\widthch>\fourthtabwidth
    \twoch{#1}{#2}{#3}{#4}{#5}
\else
    \fourch{#1}{#2}{#3}{#4}{#5}
\fi\fi\fi}


\begin{document}

\begin{table}[]
\begin{tabular}{lllllllllrlll}
Bernoulli &  &  &  &  &  & \textbf{SYLHET CADET COLLEGE}       &  &                       & \multicolumn{1}{l}{}                        &                        &                        &                        \\ \cline{10-13} 
       &  &  &  &  &  & PROGRESS TEST EXAMINATION - 2025             &  & \multicolumn{1}{l|}{} & \multicolumn{1}{c|}{Ques Setter}            & \multicolumn{3}{l|}{}                                                    \\ \cline{10-13} 
       &  &  &  &  &  & CLASS: 000                          &  & \multicolumn{1}{l|}{} & \multicolumn{1}{c|}{Moderator}              & \multicolumn{3}{l|}{}                                                    \\ \cline{10-13} 
       &  &  &  &  &  & MULTIPLE CHOICE QUESTIONS           &  & \multicolumn{1}{l|}{} & \multicolumn{1}{c|}{VP}                     & \multicolumn{3}{l|}{}                                                    \\ \cline{10-13} 
       &  &  &  &  &  & STATISTICS                          &  &                       &                                             &                        &                        &                        \\ \cline{11-13} 
       &  &  &  &  &  & SECOND PAPER                        &  &                       & \multicolumn{1}{r|}{\textbf{Subject Code:}} & \multicolumn{1}{l|}{1} & \multicolumn{1}{l|}{3} & \multicolumn{1}{l|}{0} \\ \cline{11-13} 
       &  &  &  &  &  & [According to the Syllabus of 2025] &  &                       &                                             &                        &                        &                        \\ \cline{12-12}
       &  &  &  &  &  & TIME – 25 minutes                   &  &                       & \textbf{Set:}                               & \multicolumn{1}{l|}{}  & \multicolumn{1}{l|}{C} &                        \\ \cline{12-12}
       &  &  &  &  &  & FULL MARKS – 25                     &  &                       & \multicolumn{1}{l}{}                        &                        &                        &                       
\end{tabular}
\end{table}



%  \normalfont\normalsize
 % 11.45a.m.~--~1.45p.m.
\hrule

\begin{center}
[N.B. – Answer all the questions. Each question carries ONE mark. Block fully, with a black ball- point pen, the circle of the letter that stands for the correct/best answer in the “Answer sheet” for the Multiple Choice Questions Examination.]\\

  
  \textbf{Candidates are asked not to leave any mark or spot on the question paper.}
\end{center}
\begin{questions}

\question \textbf{E(4x+2Y) = ?}
\choice{E(X) - E(Y)}{4E(X) + 2E(Y)}{2E(X) + 4E(Y)}{$E(X) \times E(Y)$}{b}

% Situation Set Starts
\textbf{Answer the next THREE questions based on the following information}

\begin{table}[h]
\centering
\begin{tabular}{c|c|c|c}
X     & 1           & 2           & 3           \\ \hline
P(x)  & $\frac{1}{6}$ & $\frac{1}{2}$ & $\frac{1}{3}$
\end{tabular}
\end{table}


\question \textbf{What is the value of $E(X)$?}
\choice{$2.00$}{$2.17$}{$2.33$}{$2.50$}{b}

\question \textbf{What is the value of $E(X^2)$?}
\choice{$5.17$}{$4.83$}{$5.00$}{$5.33$}{a}

\question \textbf{What is $V(3X)$?}
\choice{$9.67$}{$11.33$}{$12.67$}{$4.25$}{d}
% Situation Set Ends

\question \textbf{If $E(X^2) = 45$ and $V(X) = 21$, what is $E(X)$?}  
\choice{$4 \sqrt{3}$}{$2 \sqrt{6}$}{$6 \sqrt{2}$}{$7 \sqrt{2}$}{b}

\question \textbf{What is the Standard Deviation of Binomial Distribution?}
\choice{np}{npq}{nq}{$\sqrt{npq}$}{d}

\question \textbf{In a binomial distribution with $p = 0.5$ and $P(2) = 0.1093$, what is $n$?}  
\choice{15}{1}{8}{12}{c}

\question \textbf{Consider a binomial experiment. Which of the following statements is/are true?}  

i. Each trial results in exactly one of two possible outcomes. \\  
ii. The expected value is always greater than the variance. \\  
iii. The probability mass function of a binomial distribution can be computed using the binomial formula.  

\textbf{Which one is correct?}  

\choice{i and ii}{i and iii}{ii and iii}{i, ii and iii}{d}  

% Situation Set Starts
\textbf{Answer the next two questions based on the following information}

\begin{center}
The mean of a Binomila distribution is 40 and standard deviation 6. 
\end{center}

\question \textbf{What is the value of $n$?}
\choice{200}{300}{400}{500}{c}

\question \textbf{What is the value of $1-q$?}
\choice{0.5}{0.2}{0.3}{0.1}{d}

\question \textbf{What is the value of $P(X\le 40)$?}
\choice{0.52}{0.54}{0.45}{0.91}{b}
% Situation Set Ends

\question \textbf{Which one is true of the parameter (m) of Poisson Distribution?}
\choice{$m=0$}{$m<0$}{$m>0$}{$m=1$}{c}

\question \textbf{For a Poisson variate $X$, if $P(2) = P(3)$, what is the variance?}  
\choice{3}{4}{5}{6}{a}

\question \textbf{A number is randomly chosen from a list of 10 consecutive positive
integers. What is the probability that the number selected is greater than the
average (arithmetic mean) of all 10 integers?}
\choice{$\frac 13$}{$\frac 34$}{$\frac 4{10}$}{$\frac 12$}{d}

\question \textbf{Let $S = \{1, 2, 3, \dots, 10\}$. Which of the following pairs of events are disjoint?}

\choice{$A$: Multiples of 3, $B$: Multiples of 5}{%
$A$: Prime numbers, $B$: Even numbers greater than 2}{%
$A$: Numbers less than 4, $B$: Numbers greater than 6}{%
None of the above}{d}

\question \textbf{The probability of rain is $\frac 16$ for any given day next week. What is the probability that it will rain on both Monday and Tuesday?}
\choice{$\frac 16$}{$\frac 1{36}$}{$\frac 56$}{$\frac 1{17}$}{a}

\question \textbf{If \( P(A) = 0.2 \), \( P(B) = 0.3 \), and \( P(A \cup B) = 0.4 \), what is \( P(A \cap B) \)?}  
\choice{$0.9$}{$0.2$}{$0.3$}{$0.1$}{d}

\question \textbf{If two fair coins are tossed together, what is the probability of getting at least one head?}  
\choice{$\frac12$}{$\frac13$}{$\frac34$}{$\frac14$}{c}  

\question \textbf{A die is thrown thrice and the number of times a 6 appears is denoted by $X$. How many possible values can $X$ take?}
\choice{1}{2}{3}{4}{d}

% Multiple Completion Starts
\question \textbf{For a continuous random variable \( X \) with PDF \( f(x) = k(2 - x) \) defined on \( 0 \leq x \leq 2 \):}

i. The value of \( k \) is 1. \\
ii. The cumulative distribution function \( F(x) = x - \frac{x^2}{4} \) for \( 0 \leq x \leq 2 \). \\
iii. \( P(1 < X < 2) = \frac{3}{8} \)

\textbf{Which one is correct?}

\choice{i}{i and ii}{ii}{i, ii and iii}{c}
% Multiple Completion Ends

\textbf{Answer the next three questions based on the following information}

\begin{table}[H]
\centering
\begin{tabular}{c|c|c|c|c}
X & 0 & 1 & 2 & 3 \\ \hline
P(X) & $\frac 14$ & m & $\frac 13$ & $\frac 16$
\end{tabular}
\end{table}

\question \textbf{What is the value of m?}
\choice{$\frac 1 3$}{$\frac 5 {12}$}{$\frac 1 4$}{$\frac 1 6$}{c}

\question \textbf{Find $F(2)$.}
\choice{$\frac 1 2$}{$\frac 3 4$}{$\frac 5 6$}{$\frac 2 3$}{c}

\question \textbf{What is $P(X > 1)$?}
\choice{$\frac 1 2$}{$\frac 5 {12}$}{$\frac 1 3$}{$\frac 7 {12}$}{a}

\end{questions}

 \vspace{2.5cm}

\begin{center}
 “Quote” \\ -- Author
\end{center}

\pagebreak
%\newpage  %Uncomment to put on new age
\bigskip

\begin{multicols}{3}
[
Answer Key
]
\showallanswers % Phil Hirschorn
\end{multicols}


\end{document}