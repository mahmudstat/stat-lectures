\documentclass{exam}
%\documentclass[11pt,a4paper]{exam}
\usepackage{amsmath,amsthm,amsfonts,amssymb,dsfont}
\usepackage{ifthen}
\usepackage{geometry}
\geometry{
legalpaper, total={177.8mm, 290mm},left=20mm,
top=7mm, bottom=27mm,
}
\usepackage{enumerate}% http://ctan.org/pkg/enumerate
\usepackage{multicol}
\usepackage{hhline}
\usepackage[table]{xcolor}


% Accumulate the answers. Unmodified from Phil Hirschorn's answer
% https://tex.stackexchange.com/questions/15350/showing-solutions-of-the-questions-separately/15353
\newbox\allanswers
\setbox\allanswers=\vbox{}

\newenvironment{answer}
{%
    \global\setbox\allanswers=\vbox\bgroup
    \unvbox\allanswers
}%
{%
    \bigbreak
    \egroup
}

\newcommand{\showallanswers}{\par\unvbox\allanswers}
% End Phil's answer


% Is there a better way?
\newcommand*{\getanswer}[5]{%
    \ifthenelse{\equal{#5}{a}}
    {\begin{answer}\thequestion. (a)~#1\end{answer}}
    {\ifthenelse{\equal{#5}{b}}
        {\begin{answer}\thequestion. (b)~#2\end{answer}}
        {\ifthenelse{\equal{#5}{c}}
            {\begin{answer}\thequestion. (c)~#3\end{answer}}
            {\ifthenelse{\equal{#5}{d}}
                {\begin{answer}\thequestion. (d)~#4\end{answer}}
                {\begin{answer}\textbf{\thequestion. (#5)~Invalid answer choice.}\end{answer}}}}}
}

\setlength\parindent{0pt}
%usage \choice{ }{ }{ }{ }
%(A)(B)(C)(D)
\newcommand{\fourch}[5]{
    \par
    \begin{tabular}{*{4}{@{}p{0.23\textwidth}}}
        (a)~#1 & (b)~#2 & (c)~#3 & (d)~#4
    \end{tabular}
    \getanswer{#1}{#2}{#3}{#4}{#5}
}

%(A)(B)
%(C)(D)
\newcommand{\twoch}[5]{
    \par
    \begin{tabular}{*{2}{@{}p{0.46\textwidth}}}
        (a)~#1 & (b)~#2
    \end{tabular}
    \par
    \begin{tabular}{*{2}{@{}p{0.46\textwidth}}}
        (c)~#3 & (d)~#4
    \end{tabular}
    \getanswer{#1}{#2}{#3}{#4}{#5}
}

%(A)
%(B)
%(C)
%(D)
\newcommand{\onech}[5]{
    \par
    (a)~#1 \par (b)~#2 \par (c)~#3 \par (d)~#4
    \getanswer{#1}{#2}{#3}{#4}{#5}
}

\newlength\widthcha
\newlength\widthchb
\newlength\widthchc
\newlength\widthchd
\newlength\widthch
\newlength\tabmaxwidth

\setlength\tabmaxwidth{0.96\textwidth}
\newlength\fourthtabwidth
\setlength\fourthtabwidth{0.25\textwidth}
\newlength\halftabwidth
\setlength\halftabwidth{0.5\textwidth}

\newcommand{\choice}[5]{%
\settowidth\widthcha{AM.#1}\setlength{\widthch}{\widthcha}%
\settowidth\widthchb{BM.#2}%
\ifdim\widthch<\widthchb\relax\setlength{\widthch}{\widthchb}\fi%
    \settowidth\widthchb{CM.#3}%
\ifdim\widthch<\widthchb\relax\setlength{\widthch}{\widthchb}\fi%
    \settowidth\widthchb{DM.#4}%
\ifdim\widthch<\widthchb\relax\setlength{\widthch}{\widthchb}\fi%

% These if statements were bypassing the \onech option.
% \ifdim\widthch<\fourthtabwidth
%     \fourch{#1}{#2}{#3}{#4}{#5}
% \else\ifdim\widthch<\halftabwidth
% \ifdim\widthch>\fourthtabwidth
%     \twoch{#1}{#2}{#3}{#4}{#5}
% \else
%      \onech{#1}{#2}{#3}{#4}{#5}
%  \fi\fi\fi}

% Allows for the \onech option.
\ifdim\widthch>\halftabwidth
    \onech{#1}{#2}{#3}{#4}{#5}
\else\ifdim\widthch<\halftabwidth
\ifdim\widthch>\fourthtabwidth
    \twoch{#1}{#2}{#3}{#4}{#5}
\else
    \fourch{#1}{#2}{#3}{#4}{#5}
\fi\fi\fi}


\begin{document}

\begin{table}[h]
\centering
\begin{tabular}{lllll}
\textbf{\large SYLHET CADET COLLEGE} &  &  &  &  \\ \cline{4-5} 
PRETEST EXAMINATION - 2024 &  & \multicolumn{1}{l|}{} & \multicolumn{1}{l|}{Set} & \multicolumn{1}{l|}{A} \\ \cline{4-5} 
CLASS: XII &  &  &  &  \\ \cline{3-5} 
MULTIPLE CHOICE QUESTIONS & \multicolumn{1}{l|}{\textbf{Subject Code:}} & \multicolumn{1}{l|}{1} & \multicolumn{1}{l|}{3} & \multicolumn{1}{l|}{0} \\ \cline{3-5} 
STATISTICS SECOND PAPER &  &  &  &  \\
TIME – 25 minutes &  &  &  &  \\
FULL MARKS – 25 &  &  &  & 
\end{tabular}
\end{table}
%  \normalfont\normalsize
 % 11.45a.m.~--~1.45p.m.
\hrule

\begin{center}
[N.B. – Answer all the questions. Each question carries ONE mark. Block fully, 
with a black ball- point pen, the circle of the letter that stands for 
the correct/best answer in the “Answer sheet” for the Multiple Choice 
Questions.]\\

  
  \textbf{Candidates are asked not to leave any mark or spot on the question paper.}
\end{center}
\begin{questions}


\question \textbf{A die is rolled twice. How many possible outcomes are there?}
\choice{6}{12}{36}{18}{c}

\textbf{Answer the next TWO questions based on the following information.}

An urn contains 5 red, 7 blue, and 8 green balls.

\question \textbf{What is the probability that the ball drawn is red?}
\choice{0.26}{0.25}{0.2}{0.4}{a}

\question \textbf{P(The ball drawn is not blue)--}
\choice{$\frac{13}{20}$}{$0.5$}{$\frac{7}{20}$}{$\frac{8}{20}$}{a}

\question \textbf{The conditions for a cumulative distribution function (CDF) are--}

i. $F(x)$ is non-decreasing.

ii. $0 \le F(x) \le 1$

iii. $\displaystyle \lim_{x \to \infty} F(x) = 1$

\textbf{Which one is correct?}

\choice{i and ii}{ii and iii}{i and iii}{i, ii, and iii}{d}

\question \textbf{Which one is not a discrete random variable?}
\choice{Summation two die throw outcome}{Weight}{Number of heads in five coin tosses}{Released version number of a software}{d}

\question \textbf{$f(x) = x^2; 0 < X < 4$; What is $F(4)$?}  
\choice{16}{0}{4}{1}{d}

\textbf{Answer the next three questions based on the following information}

\begin{center}
The probability function of random variable $x$ is given below:

\( P(x) = \frac{2x + 1}{k}; x = 1, 2, 3, 4 \)
\end{center}

\question \textbf{What is the value of $k$?}
\choice{18}{25}{12}{24}{d}

\question \textbf{What is $E(X)$?}
\choice{1.75}{2.92}{3.25}{2.25}{b}

\question \textbf{What is $V(X)$?}
\choice{1.05}{3.0}{1.5}{1.25}{a}

\question \textbf{The characteristics of binomial distribution--}

i. $E(X) > V(X)$ \\
ii. $E(X) = V(X)$ \\
iii. $E(X) = np$

\textbf{Which one is correct?}

\choice{i and ii}{i and iii}{ii and iii}{i, ii and iii}{b}

\question \textbf{The parameter of a Poisson Distribution is 5. What is its mean?}
\choice{2}{5}{2.24}{25}{b}

\question \textbf{When does Binomial Distribution tend to Poisson Distribution?}
\choice{$n \rightarrow \infty, p \rightarrow 0$ \& $np$ is finite}{$n \rightarrow \infty, p \rightarrow 0$ \& $np$ is infinite}{$n \rightarrow \infty, p \rightarrow \infty$ \& $np$ is finite}{$n \rightarrow 0, p \rightarrow \infty$ \& $np$ is infinite}{a}

\question \textbf{A City has a dependency ratio of 0.52. If its working-age population (15-64) is 50,000, what is the total number of dependents (0-14 and 65+)?}
\choice{15,600}{20,000}{26,000}{30,000}{c}

\textbf{Answer the next TWO questions using the following information}

$P(C) = \frac 2 5, P(D) = \frac 3 4 \space \& \space P(C \cup D) = \frac{9}{10}$

\question \textbf{$P(C \cap D) = ?$}
\choice{$\frac{1}{10}$}{$\frac{1}{4}$}{$\frac{7}{20}$}{$\frac{4}{5}$}{b}

\question \textbf{$P(C \cap \bar D)=?$}
\choice{$\frac{1}{10}$}{$\frac{2}{5}$}{$\frac{2}{20}$}{$\frac{3}{10}$}{c}

\question \textbf{What is the minimum value of variance a random variable?}
\choice{$-\infty$}{1}{0}{-1}{c}

\question \textbf{If $E(X) = -0.5$, then $E(1-2X) = $?}
\choice{0}{-1}{2}{1}{c}

\question \textbf{What is the Standard Deviation of Binomial Distribution?}
\choice{np}{npq}{nq}{$\sqrt{npq}$}{d}

\question \textbf{The population of a city is 500,000, and the number of 
live births recorded in a year is 8,000. What is the Crude Birth Rate (CBR)?}
\choice{12 per 1,000}{16 per 1,000}{20 per 1,000}{22 per 1,000}{b}

\question \textbf{In a Binomial distribution, how are mean and variance related?}
\choice{$Mean > Variance$}{$Mean < Variance$}{$Mean = Variance$}{$Mean =2 \times Variance$}{a}

\textbf{Answer the next two questions based on the following information}

For a Poisson variate X, P(2) = P(5).

\question \textbf{What is standard deviation?}
\choice{1.978}{1.998}{1.989}{1.889}{a}

\question \textbf{What is the value of P(2)?}
\choice{0.25}{0.14}{0.15}{0.02}{c}

\question \textbf{If \( P(2) \) in a Poisson distribution with parameter 
\(\lambda\) equals 0.2240, what is the parameter \(\lambda\)?}
\choice{2.4551}{1.2515}{1.2115}{2.5112}{b}

\textbf{Answer the following 2 questions based on the information given below.}  

\begin{table}[h]
\centering
\begin{tabular}{|c|c|c|}
\hline
\textbf{City} & \textbf{Population (in thousands)} & \textbf{Area (in km\(^2\)} \\ \hline
Gamma         & 1200                              & 400                        \\ \hline
Delta         & 800                               & 320                        \\ \hline
\end{tabular}
\end{table}  

\question \textbf{What is the population density of City Delta?}  
\choice{2 people/km\(^2\)}{4 people/km\(^2\)}{2.5 people/km\(^2\)}{2.2 people/km\(^2\)}{b}  

\question \textbf{Which city is less densely populated?}  
\choice{Gamma}{Delta}{Both are equal}{Cannot be determined}{b}  

\end{questions}

 \vspace{2.5cm}

\begin{center}
"It is a capital mistake to theorize before one has data." -- Sir Arthur Conan Doyle
\end{center}

\pagebreak
%\newpage  %Uncomment to put on new age
\bigskip

\begin{multicols}{3}
[
Answer Key
]
\showallanswers % Phil Hirschorn
\end{multicols}


\end{document}