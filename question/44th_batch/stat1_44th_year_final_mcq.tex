\documentclass{exam}
%\documentclass[11pt,a4paper]{exam}
\usepackage{amsmath,amsthm,amsfonts,amssymb,dsfont}
\usepackage{ifthen}
\usepackage[legalpaper, total={177.8mm, 290mm}]{geometry}
\usepackage{enumerate}% http://ctan.org/pkg/enumerate
\usepackage{multicol}
\usepackage{hhline}
\usepackage[table]{xcolor}


% Accumulate the answers. Unmodified from Phil Hirschorn's answer
% https://tex.stackexchange.com/questions/15350/showing-solutions-of-the-questions-separately/15353
\newbox\allanswers
\setbox\allanswers=\vbox{}

\newenvironment{answer}
{%
    \global\setbox\allanswers=\vbox\bgroup
    \unvbox\allanswers
}%
{%
    \bigbreak
    \egroup
}

\newcommand{\showallanswers}{\par\unvbox\allanswers}
% End Phil's answer


% Is there a better way?
\newcommand*{\getanswer}[5]{%
    \ifthenelse{\equal{#5}{a}}
    {\begin{answer}\thequestion. (a)~#1\end{answer}}
    {\ifthenelse{\equal{#5}{b}}
        {\begin{answer}\thequestion. (b)~#2\end{answer}}
        {\ifthenelse{\equal{#5}{c}}
            {\begin{answer}\thequestion. (c)~#3\end{answer}}
            {\ifthenelse{\equal{#5}{d}}
                {\begin{answer}\thequestion. (d)~#4\end{answer}}
                {\begin{answer}\textbf{\thequestion. (#5)~Invalid answer choice.}\end{answer}}}}}
}

\setlength\parindent{0pt}
%usage \choice{ }{ }{ }{ }
%(A)(B)(C)(D)
\newcommand{\fourch}[5]{
    \par
    \begin{tabular}{*{4}{@{}p{0.23\textwidth}}}
        (a)~#1 & (b)~#2 & (c)~#3 & (d)~#4
    \end{tabular}
    \getanswer{#1}{#2}{#3}{#4}{#5}
}

%(A)(B)
%(C)(D)
\newcommand{\twoch}[5]{
    \par
    \begin{tabular}{*{2}{@{}p{0.46\textwidth}}}
        (a)~#1 & (b)~#2
    \end{tabular}
    \par
    \begin{tabular}{*{2}{@{}p{0.46\textwidth}}}
        (c)~#3 & (d)~#4
    \end{tabular}
    \getanswer{#1}{#2}{#3}{#4}{#5}
}

%(A)
%(B)
%(C)
%(D)
\newcommand{\onech}[5]{
    \par
    (a)~#1 \par (b)~#2 \par (c)~#3 \par (d)~#4
    \getanswer{#1}{#2}{#3}{#4}{#5}
}

\newlength\widthcha
\newlength\widthchb
\newlength\widthchc
\newlength\widthchd
\newlength\widthch
\newlength\tabmaxwidth

\setlength\tabmaxwidth{0.96\textwidth}
\newlength\fourthtabwidth
\setlength\fourthtabwidth{0.25\textwidth}
\newlength\halftabwidth
\setlength\halftabwidth{0.5\textwidth}

\newcommand{\choice}[5]{%
\settowidth\widthcha{AM.#1}\setlength{\widthch}{\widthcha}%
\settowidth\widthchb{BM.#2}%
\ifdim\widthch<\widthchb\relax\setlength{\widthch}{\widthchb}\fi%
    \settowidth\widthchb{CM.#3}%
\ifdim\widthch<\widthchb\relax\setlength{\widthch}{\widthchb}\fi%
    \settowidth\widthchb{DM.#4}%
\ifdim\widthch<\widthchb\relax\setlength{\widthch}{\widthchb}\fi%

% These if statements were bypassing the \onech option.
% \ifdim\widthch<\fourthtabwidth
%     \fourch{#1}{#2}{#3}{#4}{#5}
% \else\ifdim\widthch<\halftabwidth
% \ifdim\widthch>\fourthtabwidth
%     \twoch{#1}{#2}{#3}{#4}{#5}
% \else
%      \onech{#1}{#2}{#3}{#4}{#5}
%  \fi\fi\fi}

% Allows for the \onech option.
\ifdim\widthch>\halftabwidth
    \onech{#1}{#2}{#3}{#4}{#5}
\else\ifdim\widthch<\halftabwidth
\ifdim\widthch>\fourthtabwidth
    \twoch{#1}{#2}{#3}{#4}{#5}
\else
    \fourch{#1}{#2}{#3}{#4}{#5}
\fi\fi\fi}


\begin{document}

\begin{table}[h]
\centering
\begin{tabular}{lllll}
\textbf{\large SYLHET CADET COLLEGE} &  &  &  &  \\ \cline{4-5} 
FIRST TERM-END EXAMINATION - 2023 &  & \multicolumn{1}{l|}{} & \multicolumn{1}{l|}{Set} & \multicolumn{1}{l|}{A} \\ \cline{4-5} 
CLASS: XI &  &  &  &  \\ \cline{3-5} 
MULTIPLE CHOICE QUESTIONS & \multicolumn{1}{l|}{\textbf{Subject Code:}} & \multicolumn{1}{l|}{1} & \multicolumn{1}{l|}{2} & \multicolumn{1}{l|}{9} \\ \cline{3-5} 
STATISTICS FIRST PAPER &  &  &  &  \\
TIME – 20 minutes &  &  &  &  \\
FULL MARKS – 20 &  &  &  & 
\end{tabular}
\end{table}
%  \normalfont\normalsize
 % 11.45a.m.~--~1.45p.m.
\hrule

\begin{center}
[N.B. – Answer all the questions. Each question carries ONE mark. Block fully, with a black ball- point pen, the circle of the letter that stands for the correct/best answer in the “Answer sheet” for the Multiple Choice Questions Examination.]\\

  
  \textbf{Candidates are asked not to leave any mark or spot on the question paper.}
\end{center}
\begin{questions}

\question \textbf{If $f_i = 3, 5, 7$ and $x_i = 2, 4, 7$; ; what is the value of $\displaystyle \sum_{i=1}^3 f_ix_i^2$?}
\choice{450}{350}{345}{435}{d}

\question \textbf{Which is not a function of statistics?}
\choice{Data collection}{Data organization}{Analysis}{Database creation}{d}

\question \textbf{If $\sum (x_i-k)=0$, what is the value of k?}
\choice{$n$}{$\bar x$}{$x$}{$n \bar x$}{b}

\question \textbf{Which one is an example of an infinite population?}
\choice{Students of Dhaka University}{Cadets of SCC}{Minor planets in the solar system}{Red blood cells in a person's body}{d}

\question \textbf{If a rate is defined as $\displaystyle R = \frac cd$, where c is constant, then which measure is perfect?}
\choice {Weighted arithmetic mean}{Harmonic mean}{Quadratic mean}{Weighted geometric mean}{b}

\question \textbf{Capital and profit belong to a variable which is--}

i. Bivariate \\
ii. Quantitative \\
iii. Qualitative

\textbf{Which one is correct?}

\choice{i and ii}{i and iii}{ii and iii}{i, ii and iii}{a}

\question \textbf{Which formula is correct for harmonic mean?}
\choice{$\dfrac{n}{\sum_{i=1}^n \dfrac{f_i}{x_i}}$}{$\dfrac{f_i}{\sum_{i=1}^n \dfrac{f_i}{x_i}}$}{$\dfrac{\sum f_i}{\sum_{i=1}^n \dfrac{f_i}{x_i}}$}{$\dfrac{\sum f_i}{\sum_{i=1}^n \dfrac{1}{x_i}}$}{a}

\question \textbf{Which of the following may be used to determine mode?}
\choice{Histogram}{Frequency Curve}{Ogive}{Frequency Polygon}{a}

\question \textbf{Which is considered statistics?}
\choice{Jaman obtained 75 in statistics}{Shafiq lives at Road no. 5}{Mean monthly income in a city is 60,000 taka}{Width of a book is 10 cm}{c}

\question \textbf{Time Series has how many components?}
\choice{2}{3}{4}{5}{c}

\textbf{Answer the next three questions as per the following information.}

\begin{center}
42 44 59 64 70 72 74 91 94 are 9 values.
\end{center}

\question \textbf{What is the 50th percentile?}
\choice {64}{70}{72}{71}{b}

\question \textbf{Below which value lie 70 percent values?}
\choice {42}{44}{59}{74}{d}

\question \textbf{Above which value lie 30\% observations?}
\choice{3rd Quartile}{Median}{30th Percentile}{70th percentile}{d}

\question \textbf{Limitations of published statistics in Bangladesh are --}

i. Wrong data collection method \\
ii. Insufficient data \\
iii. Lack of proper training

\textbf{Which one is correct?}

\choice{i and ii}{i and iii}{ii and iii}{i, ii and iii}{d}

\question \textbf{An equation is: y = 5x + 9. If $\bar x = 20, \bar y = ?$}
\choice{100}{209}{109}{29}{c}

\question \textbf{The standard deviation of a mesokurtik distribution is 2. What is the value of the 4th central moment?}
\choice{4}{8}{16}{48}{d}

\question \textbf{$\beta_2 = \sqrt 9$ implies data are--}
\choice{Leptokurtic}{Platykurtic}{Mesokurtic}{Symmetric}{c}

\question \textbf{What is the second central moments of first 10 natural numbers?}
\choice{9.90}{9.09}{8.25}{5.67}{c}

\textbf{Answer the next THREE questions based on the following information}

\begin{table}[h]
\begin{tabular}{c|cccccccc}
Year & 2016 & 2017 & 2018 & 2019 & 2020 & 2021 & 2022 & 2023 \\ \hline
USD Exchange Rate & 78.35 & 79.49 & 82.87 & 83.26 & 84.60 & 84.37 & 85.80 & 106.70
\end{tabular}
\caption{\label{usdrate}Source--Investing.com}
\end{table}

\question \textbf{What is the second value of semi-average method?}
\choice{85.40}{90.37}{91.73}{89.78}{b}

\question \textbf{What kind of a trend do the data have?}
\choice{Upward}{Downward}{Both upward \& downward}{No trend}{a}

\question \textbf{Which component of time series is visible in the later part of the data?}
\choice{Seasonal Variation}{General Trend}{Irregular Variation}{Cyclic Variation}{c}

\question \textbf{First moment around zero is --}
\choice{0}{1}{-1}{Arithmetic Mean}{d}

\question \textbf{Which moment is equal to zero?}
\choice{First raw moment around 1}{Second central moment}{First central moment}{Second raw moment around 0}{c}

\question \textbf{Which measure of trend is subjective?}
\choice{Semi-average method}{Graphical method}{Moving average method}{None of the above}{b}

\question \textbf{Which is a type of trend?}

i. Linear trend \\
ii. Non-linear trend \\
iii. Cyclic trend

\textbf{Which one is correct?}

\choice{i and ii}{i and iii}{ii and iii}{i, ii and iii}{a}

%\question \textbf{To complete the song, the last answer should be
%\choice{a}{b}{c}{d}{e} % Invalid answer choice

\end{questions}

 \vspace{2.5cm}

\begin{center}
An approximate answer to the right problem is worth a good deal more than an exact answer to an approximate problem. – John Tukey.
\end{center}

\pagebreak
%\newpage  %Uncomment to put on new age
\bigskip

\begin{multicols}{3}
[
Answer Key
]
\showallanswers % Phil Hirschorn
\end{multicols}


\end{document}