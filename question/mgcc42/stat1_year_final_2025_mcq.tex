\documentclass[12pt]{exam}
%\documentclass[11pt,a4paper]{exam}
\usepackage{amsmath,amsthm,amsfonts,amssymb,dsfont}
\usepackage{float}
\usepackage{ifthen}
\usepackage{array}
\usepackage{geometry}
\geometry{
legalpaper, total={177.8mm, 290mm},left=10mm, right=10mm,
top=7mm, bottom=10mm,
}
\usepackage{enumerate}% http://ctan.org/pkg/enumerate
\usepackage{multicol}
\usepackage{hhline}
\usepackage[table]{xcolor}


% Accumulate the answers. Unmodified from Phil Hirschorn's answer
% https://tex.stackexchange.com/questions/15350/showing-solutions-of-the-questions-separately/15353
\newbox\allanswers
\setbox\allanswers=\vbox{}

\newenvironment{answer}
{%
    \global\setbox\allanswers=\vbox\bgroup
    \unvbox\allanswers
}%
{%
    \bigbreak
    \egroup
}

\newcommand{\showallanswers}{\par\unvbox\allanswers}
% End Phil's answer


% Is there a better way?
\newcommand*{\getanswer}[5]{%
    \ifthenelse{\equal{#5}{a}}
    {\begin{answer}\thequestion. (a)~#1\end{answer}}
    {\ifthenelse{\equal{#5}{b}}
        {\begin{answer}\thequestion. (b)~#2\end{answer}}
        {\ifthenelse{\equal{#5}{c}}
            {\begin{answer}\thequestion. (c)~#3\end{answer}}
            {\ifthenelse{\equal{#5}{d}}
                {\begin{answer}\thequestion. (d)~#4\end{answer}}
                {\begin{answer}\textbf{\thequestion. (#5)~Invalid answer choice.}\end{answer}}}}}
}

\setlength\parindent{0pt}
%usage \choice{ }{ }{ }{ }
%(A)(B)(C)(D)
\newcommand{\fourch}[5]{
    \par
    \begin{tabular}{*{4}{@{}p{0.23\textwidth}}}
        (a)~#1 & (b)~#2 & (c)~#3 & (d)~#4
    \end{tabular}
    \getanswer{#1}{#2}{#3}{#4}{#5}
}

%(A)(B)
%(C)(D)
\newcommand{\twoch}[5]{
    \par
    \begin{tabular}{*{2}{@{}p{0.46\textwidth}}}
        (a)~#1 & (b)~#2
    \end{tabular}
    \par
    \begin{tabular}{*{2}{@{}p{0.46\textwidth}}}
        (c)~#3 & (d)~#4
    \end{tabular}
    \getanswer{#1}{#2}{#3}{#4}{#5}
}

%(A)
%(B)
%(C)
%(D)
\newcommand{\onech}[5]{
    \par
    (a)~#1 \par (b)~#2 \par (c)~#3 \par (d)~#4
    \getanswer{#1}{#2}{#3}{#4}{#5}
}

\newlength\widthcha
\newlength\widthchb
\newlength\widthchc
\newlength\widthchd
\newlength\widthch
\newlength\tabmaxwidth

\setlength\tabmaxwidth{0.96\textwidth}
\newlength\fourthtabwidth
\setlength\fourthtabwidth{0.25\textwidth}
\newlength\halftabwidth
\setlength\halftabwidth{0.5\textwidth}

\newcommand{\choice}[5]{%
\settowidth\widthcha{AM.#1}\setlength{\widthch}{\widthcha}%
\settowidth\widthchb{BM.#2}%
\ifdim\widthch<\widthchb\relax\setlength{\widthch}{\widthchb}\fi%
    \settowidth\widthchb{CM.#3}%
\ifdim\widthch<\widthchb\relax\setlength{\widthch}{\widthchb}\fi%
    \settowidth\widthchb{DM.#4}%
\ifdim\widthch<\widthchb\relax\setlength{\widthch}{\widthchb}\fi%

% These if statements were bypassing the \onech option.
% \ifdim\widthch<\fourthtabwidth
%     \fourch{#1}{#2}{#3}{#4}{#5}
% \else\ifdim\widthch<\halftabwidth
% \ifdim\widthch>\fourthtabwidth
%     \twoch{#1}{#2}{#3}{#4}{#5}
% \else
%      \onech{#1}{#2}{#3}{#4}{#5}
%  \fi\fi\fi}

% Allows for the \onech option.
\ifdim\widthch>\halftabwidth
    \onech{#1}{#2}{#3}{#4}{#5}
\else\ifdim\widthch<\halftabwidth
\ifdim\widthch>\fourthtabwidth
    \twoch{#1}{#2}{#3}{#4}{#5}
\else
    \fourch{#1}{#2}{#3}{#4}{#5}
\fi\fi\fi}


\begin{document}

\begin{table}[]
\begin{tabular}{lllllllllrlll}
\textit{Falcon} &  &  &  &  &  & \textbf{MYMENSINGH GIRLS' CADET COLLEGE} &  &                       & \multicolumn{1}{l}{}                        &                        &                         &                        \\
                &  &  &  &  &  & YEAR FINAL EXAMINATION - 2025            &  &                       & \multicolumn{1}{c}{}                        & \multicolumn{3}{l}{}                                                      \\
                &  &  &  &  &  & CLASS: XI                               &  &                       & \multicolumn{1}{c}{}                        & \multicolumn{3}{l}{}                                                      \\
                &  &  &  &  &  & MULTIPLE CHOICE QUESTIONS                &  &                       & \multicolumn{1}{c}{}                        & \multicolumn{3}{l}{}                                                      \\
                &  &  &  &  &  & STATISTICS                               &  &                       &                                             &                        &                         &                        \\ \cline{10-13} 
                &  &  &  &  &  & FIRST PAPER                                &  & \multicolumn{1}{l|}{} & \multicolumn{1}{r|}{\textbf{Subject Code:}} & \multicolumn{1}{l|}{1} & \multicolumn{1}{l|}{2}  & \multicolumn{1}{l|}{9} \\ \cline{10-13} 
                &  &  &  &  &  & [According to the Syllabus of 2026]      &  &                       &                                             &                        &                         &                        \\ \cline{12-12}
                &  &  &  &  &  & TIME – 25 minutes                        &  &                       & \textbf{Set:}                               & \multicolumn{1}{l|}{}  & \multicolumn{1}{l|}{Ka} &                        \\ \cline{12-12}
                &  &  &  &  &  & FULL MARKS – 25                          &  &                       & \multicolumn{1}{l}{}                        &                        &                         &                       
\end{tabular}
\end{table}



%  \normalfont\normalsize
 % 11.45a.m.~--~1.45p.m.
\hrule

\begin{center}
[N.B. – Answer all the questions. Each question carries ONE mark. Block fully, with a black ball- point pen, the circle of the letter that stands for the correct/best answer in the “Answer sheet” for the Multiple Choice Questions Examination.]\\

  
  \textbf{Candidates are asked not to leave any mark or spot on the question paper.}
\end{center}
\begin{questions}

\question \textbf{Which one represents an infinite population?}
\choice{Books in a library}{Fish in the Pacific Ocean}
{Members of a sports club}{Mobile phones in a city}{b}

\question \textbf{Which cannot be performed using Univariate data?}
\choice{Central tendency}{Dispersion}{Skewness}{Regression}{d}

\question \textbf{Given $\displaystyle \sum_{i=1}^{10} a_i^2=40$ and $\displaystyle \sum_{i=1}^{10} a_i=20$, find the value of $\displaystyle 2\sum_{i=1}^{10} a_i^2 - 3\sum_{i=1}^{10} a_i + 60$.}  
\choice{70}{100}{80}{50}{c}  

\question \textbf{What is the raw material of research?}
\choice{Data}{Theory}{Graph}{Mean}{a}

%--------Group Starts
\textbf{Answer the next THREE questions based on the following information}

Radius of 80 trees are recorded and this frequency distribution is constructed.

\begin{table}[H]
\centering
\begin{tabular}{c|c|c|c|c}
\begin{tabular}[c]{@{}c@{}}Radius (cm)\end{tabular} & 0-10 & 10-20 & 20-30 & 30-40 \\ \hline
No. of Trees & 20 & 15 & 21 & 24
\end{tabular}
\end{table}

\question \textbf{How many trees have radius between 10 and 30?}
\choice{30}{15}{36}{21}{c}

\question \textbf{How many trees have radius at least 20?}
\choice{44}{45}{24}{21}{b}

\question \textbf{What percent of trees have radius between 20 and 40?}
\choice{44\%}{56\%}{46\%}{53\%}{a}
%----Group Ends

\question \textbf{Which of the following is an example of secondary data?}

i. Data obtained from a published journal \\
ii. Data collected by a government agency and used by a researcher \\
iii. Data gathered directly through interviews

\textbf{Which one is correct?}

\choice{i and ii}{ii and iii}{i and iii}{i, ii and iii}{a}

\question \textbf{Which is not a measure of central tendency?}
\choice{Arithmetic mean}{Mode}{Range}{Quadratic mean}{c}

\question \textbf{A good measure of central tendency -}

i. is loosly defined \\
ii. takes into consideration all values \\
iii. easily understandable

\textbf{Which one is correct?}

\choice{i and ii}{i and iii}{ii and iii}{i, ii and iii}{c}

\newpage

\textbf{Answer the next three questions as per the following information.}

\begin{center}
42 44 59 64 70 72 74 91 94 are 9 values.
\end{center}

\question \textbf{What is the median?}
\choice {64}{70}{72}{71}{b}

\question \textbf{What is the first quartile?}
\choice {42.4}{44.7}{51.5}{64.2}{c}

\question \textbf{Above which value lie 60\% observations?}
\choice{70.4}{72.0}{74.6}{66.4}{c}


\question \textbf{What is the minimum possible value of standard deviation?}
\choice{$\infty$}{-1}{0}{1}{c}

\question \textbf{Which measure is unit-free?}
\choice{Range}{Mean deviation}{Standard deviation}{Coefficient of variation}{d}

\question \textbf{The moments around the origin are called --}
\choice{Central moments}{Raw moments}{First raw moment}{Measures of dispersion}{b}

% Multiple Completion Starts
\question \textbf{Standard deviation ---}

i. depends on all values \\
ii. is not affected by outliers \\
iii. can be analyzed algebraically 

\textbf{Which one is correct?}

\choice{i and ii}{i and iii}{ii and iii}{i, ii and iii}{b}
% Multiple Completion Ends

\question \textbf{First moment around zero is --}
\choice{0}{1}{-1}{Arithmetic Mean}{d}

\question \textbf{Which might have a negative value?}
\choice{$\mu_4$}{$\mu_3$}{$\mu_2'$}{$\mu_2$}{b}

\question \textbf{In a postively-skewed distribution--}

i. Frequencies of higher values are lower \\
ii. Frequencies of low values are higher \\
iii. Frequencies of higher values are higher

\textbf{Which one is correct?}

\question \textbf{What is the formula of IQR?}
\choice{$IQR = Q_3 + Q_1$}{$IQR = Q_3 - Q_1$}{$IQR = 2Q_3 - Q_1$}{$IQR = \frac{Q_3 - Q_1}{2}$}{b}

\question \textbf{The lowest possible value of the correlation coefficient ---}
\choice{1}{0}{$-\infty$}{-1}{d}

\question \textbf{Karl Pearson's method of determining the strength of correlation is not applicable for ---}
\choice{Qualitative variable}{Quantitative variable}{Discrete variable}{Continuous variable}{a}

\question \textbf{Two variables having changes in same direction at same rates display ---}
\choice{Perfect negative correlation}{Partial positive correlation}{Perfect positive correlation}{Partial negative correlation}{c}

\question \textbf{Which organization typically publishes non-official statistics in the field of health?}
\choice{UNICEF}{World Health Organization (WHO)}{World Bank}{United Nations (UN)}{b}

\end{questions}

 \vspace{.3cm}

\begin{center}

  \vfill
 --- Abdullah Al Mahmud ---
\end{center}

\pagebreak
%\newpage  %Uncomment to put on new age
\bigskip

%\begin{multicols}{3}
%[
%%Answer Key
%]
%\showallanswers % Phil Hirschorn
%\end{multicols}


\end{document}