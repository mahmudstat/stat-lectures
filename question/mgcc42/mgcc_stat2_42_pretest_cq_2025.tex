\documentclass[12pt]{article}
\usepackage{geometry}
\usepackage{amsfonts}
\usepackage{float}
\usepackage{amsmath}

\geometry{
legalpaper, total={177.8mm, 290mm},left=5mm, right=5mm,
top=7mm, bottom=12mm,
}

\usepackage{letltxmacro}

\LetLtxMacro\Oldfootnote\footnote

\newcommand{\EnableFootNotes}{%
  \LetLtxMacro\footnote\Oldfootnote%
}

\newcommand{\DisableFootNotes}{%
  \renewcommand{\footnote}[2][]{\relax}
}

\DisableFootNotes

\begin{document}

\iffalse
\begin{table}[]
\begin{tabular}{llcllcll}
 &  & \textbf{MYMENSINGH GIRLS’  CADET COLLEGE} &  &                                            & \multicolumn{1}{l}{}            &                        &                        \\
 &  & WHAT EXAMINATION - 2025                   &  &                                            &                                 &                        &                        \\
 &  & CLASS: WHAT                               &  &                                            &                                 &                        &                        \\
 &  & STATISTICS (CREATIVE)                     &  &                                            &                                 &                        &                        \\
 &  & WHAT PAPER                                &  &                                            & \multicolumn{1}{r}{}            &                        &                        \\ \cline{7-8} 
 &  & [According to the Syllabus of 2025]       &  & \multicolumn{1}{r}{\textbf{Subject Code:}} & \multicolumn{1}{l|}{\textbf{1}} & \multicolumn{1}{l|}{3} & \multicolumn{1}{l|}{0} \\ \cline{7-8} 
 &  & TIME – 2 hours \& 35 minutes              &  &                                            & \multicolumn{1}{r}{}            &                        &                        \\
 &  & FULL MARKS – 50                           &  &                                            & \multicolumn{1}{r}{\textbf{}}   &                        &                       
\end{tabular}
\end{table}
\fi

\begin{table}[]
\begin{tabular}{llllllllcllcl}
\textit{Canopus} &  &  &  &  &  &  &  & \textbf{MYMENSINGH GIRLS’ CADET COLLEGE} &                                             &                                 & \multicolumn{1}{l}{}            &                        \\
                 &  &  &  &  &  &  &  & PRETEST EXAMINATION - 2025          &                                             &                                 &                                 &                        \\
                 &  &  &  &  &  &  &  & CLASS: XII                               &                                             &                                 &                                 &                        \\
                 &  &  &  &  &  &  &  & STATISTICS (CREATIVE)                    &                                             &                                 &                                 &                        \\
                 &  &  &  &  &  &  &  & SECOND PAPER                             &                                             &                                 & \multicolumn{1}{r}{}            &                        \\ \cline{11-13} 
                 &  &  &  &  &  &  &  & [According to the Syllabus of 2026]      & \multicolumn{1}{r|}{\textbf{Subject Code:}} & \multicolumn{1}{l|}{\textbf{1}} & \multicolumn{1}{l|}{\textbf{3}} & \multicolumn{1}{l|}{0} \\ \cline{11-13} 
                 &  &  &  &  &  &  &  & TIME – 2 hours \& 35 minutes             &                                             &                                 & \multicolumn{1}{r}{}            &                        \\
                 &  &  &  &  &  &  &  & FULL MARKS – 50                          &                       
\end{tabular}
\end{table}

\vspace{-1cm}
\hrule

\begin{center}
[\textbf{N.B.} – The figures of the right margin indicate full marks. Read the stems carefully and answer the associated questions. Answer any \textbf{FIVE} questions taking at least two questions from each group]\\
\end{center}

\begin{center}
\textbf{Group  - A}
\end{center}
 \begin{enumerate}

  \item
	  \textbf{An unbiased coin is tossed 10 times.} 
  
  \begin{enumerate}
    \item
	If a coin is flung 3 times, how many outcomes are generated? \hfill 1
    \item
	If a coin is flung n times, show how many outcomes are generated. \hfill 2
    \item  
	What is the probability of getting a) at least 3 heads, b) at most 3 heads? \hfill 3
    \item
	Are these probabilities equal? a) Getting at least 2 heads \& b) Getting at least 2 tails. \\ Also justify logically. \hfill 4
  \end{enumerate}
  
     \item
	  \textbf{A continuos random variable X follows the following probability density function (pdf).} 
	  \begin{center}
	  $f(x) = 6x(1-x); 0\le x\le 1$
  \end{center}
  
  \begin{enumerate}
    \item
	Give an example of a continous random variable. \hfill 1
    \item
	Examine whether the given function is a pdf. \hfill 2
    \item  
	If $P(X>a) = P(X<a)$, find the value of a. \hfill 3
    \item
	Should $P(0.5 \le X \le 1)$  be equal to 0.5? \hfill 4
  \end{enumerate}

\item
\textbf{The probability distributions of monthly sales (in units) of two brands of laptops, Brand L (X) and Brand M (Y), are given below:}

\begin{table}[h]
\centering
\begin{tabular}{c|c|c|c|c|c}
Sales (units) & 50  & 100 & 150 & 200 & 250 \\ \hline
P(X)          & 0.08 & 0.22 & p   & 0.30 & 0.12 \\ \hline
P(Y)          & 0.10 & 0.25 & 0.35 & 0.20 & 0.10
\end{tabular}
\end{table}

\begin{enumerate}
    \item
    What is Expected Value in probability? \hfill 1
    \item
    Can Expected Value be negative? Justify with an example. \hfill 2
    \item
    Find p from the table. \hfill 3
    \item
    Which brand has higher expected sales? Compare variability using standard deviation. \hfill 4
  
\end{enumerate}  

     \item
  \textbf{Sampling is one of the essential steps in the analysis of observational studies. It eliminates the necessity of working with the entire population, thereby minimizing cost and time required to draw conclusions.}

  
  \begin{enumerate}
    \item
	What is an infinite population? \hfill 1
    \item
	Distinguish between statistic and parameter with examples. \hfill 2

    \item  
	Outline the steps in a sample survey. \hfill 3
    \item
	Illustrate the remainder approach with an example. \hfill 4
  \end{enumerate}
  \begin{center}
\textbf{Group  - B}
\end{center}

 \item
	  \textbf{The electric kettles produced by a certain manufacturer are 12\% 
	  defective on average. The company supplies 20 kettles in a packet. A retailer
	  bought 1000 packets.} 
  
  \begin{enumerate}
    \item What is denoted by $X$ in Binomial distribution? \hfill 1
    \item How does the probability mass function (PMF) of a binomial distribution change with increasing \( n \)? \hfill 2
    \item  
	What is the probability that no. of defective kettles is at most 2? \hfill 3
    \item
	In how many packtes, there are exactly 3 defective kettles? \hfill 4
  \end{enumerate}
  
 \item
	  \textbf{The standard deviation of a Poisson distribution is 2.} 
  
  \begin{enumerate}
  \item What is the mean of Poisson distribution? \hfill 1
  \item In a a Poisson distribution, P(2) = P(3). What is its mean?  \hfill 2
    \item  
	Find $P(X\ge 2)$ \hfill 3
    \item
	If P(a) = P(a+1), what is the value of a? \hfill 4
  \end{enumerate}
  
 \item
  \textbf{The mean and standard deviations of a normal variate are 25 and 5, respectively. }
  
  \begin{enumerate}
    \item
	What is the mean of  a normnal variate? \hfill 1
    \item
	What is the relationship between the Poission and Normal distribution? \hfill 2
    \item  
	Find the value of $P(20 \le X \le 40).$ \hfill 3
    \item
	Examine: $P(X>20) = P(X<20)$. Also, provide your intuitive reasoning. \hfill 4  
\end{enumerate}

  \item
\textbf{Price and Quantity Data for Three Essential Commodities in 2021 and 2022 are given below.}

\begin{table}[h]
\centering
\begin{tabular}{ccccccc}
\hline
Commodity & \multicolumn{2}{c}{(2021)} & \multicolumn{2}{c}{(2022)}  \\
& Price & Quantity & Price & Quantity\\
\hline
Rice & \$2.5 & 100 & \$3.0 & 90  \\
Wheat & \$2.0 & 120 & \$2.4 & 110  \\
Sugar & \$1.8 & 80 & \$2.2 & 85 \\
\hline
\end{tabular}
\end{table}

\begin{enumerate}
    \item
    What is a price index? \hfill 1
    \item
    How is Fisher’s Ideal Index constructed? \hfill 2
    \item
    From the given data, determine Laspeyres’ price index number. \hfill 3
    \item
    Compare inflation using Paasche and Laspeyres price indices. \hfill 4
\end{enumerate}
\end{enumerate}
 \vspace{2.5cm}

\begin{center}

  \vfill
  --AAM--
\end{center}


\newpage
\setcounter{page}{1}

\begin{table}[]
\begin{tabular}{llllllllcllcl}
\textit{Orion} &  &  &  &  &  &  &  & \textbf{MYMENSINGH GIRLS’ CADET COLLEGE} &                                             &                                 & \multicolumn{1}{l}{}            &                        \\
                 &  &  &  &  &  &  &  & PRETEST EXAMINATION - 2025          &                                             &                                 &                                 &                        \\
                 &  &  &  &  &  &  &  & CLASS: XII                               &                                             &                                 &                                 &                        \\
                 &  &  &  &  &  &  &  & STATISTICS (CREATIVE)                    &                                             &                                 &                                 &                        \\
                 &  &  &  &  &  &  &  & SECOND PAPER                             &                                             &                                 & \multicolumn{1}{r}{}            &                        \\ \cline{11-13} 
                 &  &  &  &  &  &  &  & [According to the Syllabus of 2026]      & \multicolumn{1}{r|}{\textbf{Subject Code:}} & \multicolumn{1}{l|}{\textbf{1}} & \multicolumn{1}{l|}{\textbf{3}} & \multicolumn{1}{l|}{0} \\ \cline{11-13} 
                 &  &  &  &  &  &  &  & TIME – 2 hours \& 35 minutes             &                                             &                                 & \multicolumn{1}{r}{}            &                        \\
                 &  &  &  &  &  &  &  & FULL MARKS – 50                          &                                             &                                 & \multicolumn{1}{r}{\textbf{}}   &                       
\end{tabular}
\end{table}


\hrule

\begin{center}
[\textbf{N.B.} – The figures of the right margin indicate full marks. Read the stems carefully and answer the associated questions. Answer any \textbf{FIVE} questions taking at least two questions from each group]\\
\end{center}

\begin{center}
\textbf{Group  - A}
\end{center}
 \begin{enumerate}
   \item
  \textbf{Hamida has recently graduated from the University of Dhaka. She applies to two firms - EduCube \& Digic- for a Data Analyst job. The probability of hiring by EduCube is 0.8 and by Digic is 0.4. The probability that none hires is 0.5.} 
  
  \begin{enumerate}
    \item
	What is a sample space? \hfill 1
    \item
	Explain how to find $P(\bar A \cap B)$ using Venn Diagram. \hfill 2
    \item  
	Find the probability of hiirng by Digic but not by EduCube. \hfill 3
    \item
	Find the probability that no firm will reject her. \hfill 4
  \end{enumerate}
  
    \item
  \textbf{The joint probability function of two random variables \( X \) and \( Y \) is given by:}
  
  \begin{center}
  \( \displaystyle P(X,Y) = \frac{x + y + 1}{42}; \quad x = 0, 1, 2; \quad y = 0, 1, 2, 3 \)
  \end{center}
 
  \begin{enumerate}
    \item How can calculate $P(X>3)$ using the concept of complementary probability?  \hfill 1
    \item What is the relationship between joint and marginal probability? Illustrate mathematically. \hfill 2
    \item
    	Calculate the marginal probability \( P(Y) \). \hfill 3
    \item
     	Determine \( P(Y \vert X = 1) \) and \( P(Y \vert X = 0) \). \hfill 4
  \end{enumerate}
  
           \item \textbf{The probability density function (pdf) of a continuous random variable is given below:}
	  
	  \begin{center}
	  $\displaystyle f(x) = \frac{1}{30} (k+2x); 2 < x < 5$
	  \end{center}
  
  \begin{enumerate}
    \item
    What is a random variable? \hfill 1
    \item
		Is probability a discrete variable? Explain in brief. \hfill 2
    \item  
	Find the value of k.  \hfill 3
    \item
	Determine the expectation and variance. \hfill 4
  \end{enumerate}
  
 \item
  \textbf{A sample survey is often preferred over a census because it significantly reduces the cost, time, and logistical complexity of data collection. By studying a representative subset of the population, researchers can obtain accurate and reliable estimates without needing to survey every individual.}

  
  \begin{enumerate}
    \item
	What is a finite population? \hfill 1
    \item
	Distinguish between sapling frame and sampling unit. \hfill 2

    \item  
	Outline the drawbacks of a census. \hfill 3
    \item
	Illustrate the quotient approach with an example. \hfill 4
  \end{enumerate}  
\begin{center}
\textbf{Group  - B}
\end{center}
  
  \item  
  \textbf{A smartphone company finds that 9\% of its phones have minor defects. Each shipment contains 35 phones. A retailer purchases 500 shipments.}  

  \begin{enumerate}  
   \item What is a Bernoulli trial?  \hfill 1
  \item What happens if $n=1$ in Binomial distribution? \hfill 2
    \item  
      What is the probability that a randomly selected shipment has at most 3 defective phones? \hfill 3  
    \item  
      In how many shipments can we expect to find between 4 and 7 defective phones (inclusive)? \hfill 5  
  \end{enumerate}  


   \item
	  \textbf{Between 1000 hrs and 1700 hrs, the average number of phonce calls per minute received by a power distribution company is 2.5. } 
  
  \begin{enumerate}
    \item
	Give an example where Poisson distribution is applicable. \hfill 1
    \item
	Find the relationship between expectation and standard deviation of Poisson distribution. \hfill 2
    \item  
	Find the probability that the number of calls is between 1 and 3 (inclusive). \hfill 3
    \item
	What is the probability that the number of calls received is greater than the average? \hfill 4
  \end{enumerate}
  
\item
\textbf{The IQ scores of 800 adults are normally distributed with a mean of 100 and a standard deviation of 15.}

\begin{enumerate}
    \item
    Describe the shape of the normal distribution. \hfill 1
    \item
    Which value(s) have the highest probability in a normal distribution? \hfill 2
    \item  
    Find the probability that a randomly chosen adult has an IQ score above 130. \hfill 3
    \item
    How many adults are expected to have IQ scores between 70 and 130? \hfill 4
\end{enumerate}  

 \item
  \textbf{Price and Quantity Data for Three Commodities in the year 2010 and 2011 are given below.}
  
  \begin{table}[h]
\centering
\begin{tabular}{ccccccc}
\hline
Commodity & \multicolumn{2}{c}{(2010)} & \multicolumn{2}{c}{(2011)}  \\
& Price & Quantity & Price & Quantity\\
\hline
A & \$10 & 10 & \$25 & 15  \\
B & \$15 & 20 & \$30 & 25  \\
C & \$20 & 30 & \$35 & 35 \\
\hline
\end{tabular}
\caption{Price and Quantity Data for Three Commodities}
\label{tab:price_quantity}
\end{table}

  
  \begin{enumerate}
    \item
	What is an index number? \hfill 1
    \item
	How is the Marshall-Edgeworth Index formed? \hfill 2
    \item  
	From the given data, determine Paasche's quantity index number \hfill 3
    \item
	Analyze, using the cost of living index, whether price of daily necessities has increased.  \hfill 4
  \end{enumerate}
\end{enumerate}

 \vspace{2.5cm}

\begin{center}
  \vfill
  --AAM--
\end{center}

\end{document}