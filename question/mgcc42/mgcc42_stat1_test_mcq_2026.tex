\documentclass[12pt]{exam}
%\documentclass[11pt,a4paper]{exam}
\usepackage{amsmath,amsthm,amsfonts,amssymb,dsfont}
\usepackage{float}
\usepackage{ifthen}
\usepackage{array}
\usepackage{geometry}
\geometry{
legalpaper, total={177.8mm, 290mm},left=10mm, right=10mm,
top=7mm, bottom=10mm,
}
\usepackage{enumerate}% http://ctan.org/pkg/enumerate
\usepackage{multicol}
\usepackage{hhline}
\usepackage[table]{xcolor}


% Accumulate the answers. Unmodified from Phil Hirschorn's answer
% https://tex.stackexchange.com/questions/15350/showing-solutions-of-the-questions-separately/15353
\newbox\allanswers
\setbox\allanswers=\vbox{}

\newenvironment{answer}
{%
    \global\setbox\allanswers=\vbox\bgroup
    \unvbox\allanswers
}%
{%
    \bigbreak
    \egroup
}

\newcommand{\showallanswers}{\par\unvbox\allanswers}
% End Phil's answer


% Is there a better way?
\newcommand*{\getanswer}[5]{%
    \ifthenelse{\equal{#5}{a}}
    {\begin{answer}\thequestion. (a)~#1\end{answer}}
    {\ifthenelse{\equal{#5}{b}}
        {\begin{answer}\thequestion. (b)~#2\end{answer}}
        {\ifthenelse{\equal{#5}{c}}
            {\begin{answer}\thequestion. (c)~#3\end{answer}}
            {\ifthenelse{\equal{#5}{d}}
                {\begin{answer}\thequestion. (d)~#4\end{answer}}
                {\begin{answer}\textbf{\thequestion. (#5)~Invalid answer choice.}\end{answer}}}}}
}

\setlength\parindent{0pt}
%usage \choice{ }{ }{ }{ }
%(A)(B)(C)(D)
\newcommand{\fourch}[5]{
    \par
    \begin{tabular}{*{4}{@{}p{0.23\textwidth}}}
        (a)~#1 & (b)~#2 & (c)~#3 & (d)~#4
    \end{tabular}
    \getanswer{#1}{#2}{#3}{#4}{#5}
}

%(A)(B)
%(C)(D)
\newcommand{\twoch}[5]{
    \par
    \begin{tabular}{*{2}{@{}p{0.46\textwidth}}}
        (a)~#1 & (b)~#2
    \end{tabular}
    \par
    \begin{tabular}{*{2}{@{}p{0.46\textwidth}}}
        (c)~#3 & (d)~#4
    \end{tabular}
    \getanswer{#1}{#2}{#3}{#4}{#5}
}

%(A)
%(B)
%(C)
%(D)
\newcommand{\onech}[5]{
    \par
    (a)~#1 \par (b)~#2 \par (c)~#3 \par (d)~#4
    \getanswer{#1}{#2}{#3}{#4}{#5}
}

\newlength\widthcha
\newlength\widthchb
\newlength\widthchc
\newlength\widthchd
\newlength\widthch
\newlength\tabmaxwidth

\setlength\tabmaxwidth{0.96\textwidth}
\newlength\fourthtabwidth
\setlength\fourthtabwidth{0.25\textwidth}
\newlength\halftabwidth
\setlength\halftabwidth{0.5\textwidth}

\newcommand{\choice}[5]{%
\settowidth\widthcha{AM.#1}\setlength{\widthch}{\widthcha}%
\settowidth\widthchb{BM.#2}%
\ifdim\widthch<\widthchb\relax\setlength{\widthch}{\widthchb}\fi%
    \settowidth\widthchb{CM.#3}%
\ifdim\widthch<\widthchb\relax\setlength{\widthch}{\widthchb}\fi%
    \settowidth\widthchb{DM.#4}%
\ifdim\widthch<\widthchb\relax\setlength{\widthch}{\widthchb}\fi%

% These if statements were bypassing the \onech option.
% \ifdim\widthch<\fourthtabwidth
%     \fourch{#1}{#2}{#3}{#4}{#5}
% \else\ifdim\widthch<\halftabwidth
% \ifdim\widthch>\fourthtabwidth
%     \twoch{#1}{#2}{#3}{#4}{#5}
% \else
%      \onech{#1}{#2}{#3}{#4}{#5}
%  \fi\fi\fi}

% Allows for the \onech option.
\ifdim\widthch>\halftabwidth
    \onech{#1}{#2}{#3}{#4}{#5}
\else\ifdim\widthch<\halftabwidth
\ifdim\widthch>\fourthtabwidth
    \twoch{#1}{#2}{#3}{#4}{#5}
\else
    \fourch{#1}{#2}{#3}{#4}{#5}
\fi\fi\fi}


\begin{document}


\iffalse

\begin{table}[]
\begin{tabular}{lllllllllrlll}
\textit{Falcon} &  &  &  &  &  & \textbf{SYLHET CADET COLLEGE}       &  &                       & \multicolumn{1}{l}{}                        &                        &                        &                        \\ \cline{10-13} 
       &  &  &  &  &  & 2ND TERM END EXAMINATION - 2025             &  & \multicolumn{1}{l|}{} & \multicolumn{1}{c|}{Ques Setter}            & \multicolumn{3}{l|}{}                                                    \\ \cline{10-13} 
       &  &  &  &  &  & CLASS: XII                          &  & \multicolumn{1}{l|}{} & \multicolumn{1}{c|}{Moderator}              & \multicolumn{3}{l|}{}                                                    \\ \cline{10-13} 
       &  &  &  &  &  & MULTIPLE CHOICE QUESTIONS           &  & \multicolumn{1}{l|}{} & \multicolumn{1}{c|}{VP}                     & \multicolumn{3}{l|}{}                                                    \\ \cline{10-13} 
       &  &  &  &  &  & STATISTICS                          &  &                       &                                             &                        &                        &                        \\ \cline{11-13} 
       &  &  &  &  &  & FIRST PAPER                        &  &                       & \multicolumn{1}{r|}{\textbf{Subject Code:}} & \multicolumn{1}{l|}{1} & \multicolumn{1}{l|}{3} & \multicolumn{1}{l|}{0} \\ \cline{11-13} 
       &  &  &  &  &  & [According to the Syllabus of 20--] &  &                       &                                             &                        &                        &                        \\ \cline{12-12}
       &  &  &  &  &  & TIME – 25 minutes                   &  &                       & \textbf{Set:}                               & \multicolumn{1}{l|}{}  & \multicolumn{1}{l|}{Ka} &                        \\ \cline{12-12}
       &  &  &  &  &  & FULL MARKS – 25                     &  &                       & \multicolumn{1}{l}{}                        &                        &                        &                       
\end{tabular}
\end{table}
\fi


\begin{table}[]
\begin{tabular}{llllllllcllll}
\textit{Rigel} &  &  &  &  &  &  &  & \multicolumn{1}{l}{\textbf{MYMENSINGH GIRLS' CADET COLLEGE}} &                                             &                        &                                 &                        \\
                &  &  &  &  &  &  &  & \multicolumn{1}{c}{TEST EXAMINATION - 2025}       &                                             &                        & \multicolumn{1}{c}{}            &                        \\
                &  &  &  &  &  &  &  & CLASS: XII                                                   &                                             &                        & \multicolumn{1}{c}{}            &                        \\
                &  &  &  &  &  &  &  & MULTIPLE CHOICE QUESTIONS                                    &                                             &                        & \multicolumn{1}{c}{}            &                        \\
                &  &  &  &  &  &  &  & STATISTICS                                                   &                                             &                        & \multicolumn{1}{r}{}            &                        \\ \cline{11-13} 
                &  &  &  &  &  &  &  & FIRST PAPER                                                 & \multicolumn{1}{r|}{\textbf{Subject Code:}} & \multicolumn{1}{l|}{1} & \multicolumn{1}{l|}{\textbf{2}} & \multicolumn{1}{l|}{9} \\ \cline{11-13} 
                &  &  &  &  &  &  &  & [According to the Syllabus of 2026]                          & \multicolumn{1}{r}{}                        &                        &                                 &                        \\ \cline{12-12}
                &  &  &  &  &  &  &  & TIME – 25 minutes                                            & \multicolumn{1}{r}{\textbf{Set:}}           & \multicolumn{1}{l|}{}  & \multicolumn{1}{l|}{\textbf{Ka}} &                        \\ \cline{12-12}
                &  &  &  &  &  &  &  & FULL MARKS – 25                                              &                                             &                        &                                 &                       
\end{tabular}
\end{table}

%  \normalfont\normalsize
 % 11.45a.m.~--~1.45p.m.
\hrule

\begin{center}
[N.B. – Answer all the questions. Each question carries ONE mark. Block fully, with a black ball- point pen, the circle of the letter that stands for the correct/best answer in the “Answer sheet” for the Multiple Choice Questions Examination.]\\

  
  \textbf{Candidates are asked not to leave any mark or spot on the question paper.}
\end{center}
\begin{questions}

\question \textbf{Classifying students based on their grades (A, B, C, etc.) represents which measurement scale?}  
\choice{Nominal}{Ordinal}{Interval}{Ratio}{b} 

\question \textbf{If $y_1=5$, $y_2=2$, $y_3=-1$, and $y_4=4$, $\displaystyle \sum_{i=1}^4 (y_i^2 + 2)=?$}  
\choice{50}{40}{54}{60}{c}  

\question \textbf{Which one falls in the category of nominal scale?}  
\choice{Height}{Temperature}{Gender}{Age}{c}  

\question \textbf{If all the rats in Sherpur is a population, all the rats in Sherpur city is --}
\choice{Data}{Sample}{Statistics}{Frequency}{b}

\question \textbf{To show runs per over in a cricket match, which diagram can be used?}
\choice{Histogram}{Bar Diagram}{Ogive}{Frequency polygon}{b}

\question \textbf{Which statement is correct}
\choice{Quartiles are well defined}{Outliers affect Median}
{Median is always present in data}{Quadratic mean is widely used}{a}

\question \textbf{An equation is: y = 5x + 9. If $\bar x = 20, \bar y = ?$}
\choice{100}{209}{109}{29}{c}

\question \textbf{Median can be determined from the--}
\choice{Histogram}{Frequency curve}{Ogive}{Pie Chart}{c}

% Situation Set Starts
\textbf{Answer the next two (2) questions based on the following information}

\begin{table}[h]
\centering
\begin{tabular}{c|c|c|c|c|c|c}
Class                                                           & $\le 20$ & 20-25 & 25-50 & 50-60 & 69-70 & $\ge 70$ \\ \hline
Frequency                                                       & 5        & 10    & 10    & 7     & 5     & 3        \\ \hline
\begin{tabular}[c]{@{}c@{}}Cumulative \\ Frequency\end{tabular} & 5        & 15    & 25    & 32    & 37    & 40      
\end{tabular}
\end{table}

\question \textbf{How many values are between 20 and 70?}
\choice{20}{32}{35}{37}{b}

\question \textbf{Which one is the median class?}
\choice{20-25}{25-50}{50-60}{60-70}{b}

\question \textbf{Two sets of data are X = 2, 4, 6 and Y = 4, 6, 8; which has higher variance?}
\choice{X}{Y}{V(X) = V(Y)}{Cannot be determined from the given information}{c}

\question \textbf{For two values, the range is found to be 12. What are the values of mean deviation and standard deviation}
\choice{(2,4)}{(4,4)}{(6, 6)}{(8,8)}{c}

\question \textbf{Which measure is unit-free?}
\choice{Range}{Mean deviation}{Standard deviation}{Coefficient of variation}{d}

\question \textbf{First moment around zero is --}
\choice{0}{1}{-1}{Arithmetic Mean}{d}

\question \textbf{The arithmetic mean of a variable is 4. What is the 
first raw moment around 2?}
\choice{2}{-2}{0}{8}{a}

\question \textbf{Moments can be--}

i. positive \\
ii. not negative \\
iii. positive or negative

\textbf{Which one is correct?}

\choice{i and ii}{i and iii}{ii and iii}{i, ii and iii}{b}

\question \textbf{The second central moment of the natural numbers 1 through 12 is --}
\choice{12.93}{11.92}{10.94}{12.60}{b}

\question \textbf{If $\gamma_1 > 0$, the data is -}
\choice{Negatively skewed}{Positively skewed}{Symmetric}{Uncertain}{b}

\question \textbf{Which of the following has the strongest correlation?}
\choice{-0.92}{0.67}{0.50}{0.91}{a}

\question \textbf{If $b_1 = 0.25$ and $b_2 = -0.43$, which implies higher impact of the independent variable?}
\choice{$b_1$}{$b_2$}{Cannot be determined}{Equal impact}{c}

\question \textbf{In additive model, in the long run, $\sum R_t = --$}
\choice{0}{1}{Undefine}{Infinity}{a}


\textbf{Answer the next THREE questions based on the following information}

\begin{table}[h]
\begin{tabular}{c|cccccccc}
Year & 2015 & 2016 & 2017 & 2018 & 2019 & 2020 & 2021 & 2022 \\ \hline
Average Temperature (°C) & 22.5 & 23.0 & 24.2 & 24.5 & 25.0 & 25.5 & 26.0 & 27.0
\end{tabular}
%\caption{\label{tempdata}Source--National Weather Service}
\end{table}

\question \textbf{What is the second value of the semi-average method?}
\choice{25.75}{26.00}{25.88}{24.29}{c}

\question \textbf{What kind of trend do the data show?}
\choice{Upward}{Downward}{Both upward \& downward}{No trend}{a}

\question \textbf{Which component of the time series is most prominent in the data?}
\choice{Seasonal Variation}{General Trend}{Irregular Variation}{Cyclic Variation}{b}

\end{questions}
\vspace{1cm}
\vfill
\begin{center}
%“You can have data without information, but you cannot have information without data.” \\ %- Daniel Keys Moran
--AAM-- 
\end{center}

\pagebreak
%\newpage  %Uncomment to put on new age
\bigskip

%\begin{multicols}{3}
%[
%Answer Key
%]
%\showallanswers % Phil Hirschorn
%\end{multicols}


\end{document}