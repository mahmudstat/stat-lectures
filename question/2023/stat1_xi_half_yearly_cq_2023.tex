\documentclass{article}
\usepackage{geometry}
\usepackage{amsfonts}



\geometry{
legalpaper, total={177.8mm, 290mm},left=20mm,
top=27mm, bottom=27mm,
}

\begin{document}

\begin{center}
  \bfseries\large
  Sylhet Cadet College

\normalsize
  Half-Yearly Examination - 2023

  Class: XI

  Subject: Statistics First Paper (Creative) 
  \fbox{Set: A} \\
  Time: 2 hour \& 10 minutes \qquad \qquad \qquad Subject Code: 129  \qquad  \qquad \qquad Full Marks: 50

%  \normalfont\normalsize
 % 11.45a.m.~--~1.45p.m.
\end{center}

\noindent
\begin{tabular}{p{\dimexpr\linewidth-2\tabcolsep}}
  N:B: Read the questions carefully and answer the associated questions correctly. Answer any FIVE questions. Figures in the right indicate full marks.\\
  \hline
\end{tabular}


  \begin{enumerate}

   \item
	  \textbf{An analyst obtains some data:}
	  \begin{center}
	  $x_1=15, x_2=-12, x_3=17, x_4=11, x_5=23$
  \end{center}
  \begin{enumerate}
    \item
	What is sample? \hfill 1
    \item
	Briefly explain shift or origin and scale. \hfill 2
    \item  
	Compute the value of $\displaystyle \sum_{i=1}^5 (x_i-10)^2$ \hfill 3
    \item
	Find the value of $\displaystyle \sum_{i=1}^5 (5x_i^2-4x_i-3)$ and examine its dependency on origin and scale. \hfill 4
  \end{enumerate}
  

    \item
  \textbf{In the test examination, marks of 11 students in statistics are: 90, 92, 93, 49, 44, 88, 80, 58, 83, 71, 76.}
  \begin{enumerate}
    \item
	What is central tendency? \hfill 1
    \item
	When is median better than arithmetic mean? Explain with an example. \hfill 2
    \item  
	Find the 3rd the quartile and 61st percentile from the data and explain.  \hfill 3
    \item
	Do quantiles depend on change of origin and scale. Prove using two examples.\hfill 4
\end{enumerate}
  
   \item
	  \textbf{A cyclist moves around a square-shaped lake with the speeds 20, 25, 30, and 16 km per hour.} 
  
  \begin{enumerate}
    \item
	What is grouped data? \hfill 1
    \item
	Is arithmetic mean suitable for this data? \hfill 2
    \item  
	Find the average speed of the cyclist. \hfill 3
    \item
	Can we use some other formula for finding the average? Demonstrate. \hfill 4
  \end{enumerate}

   \item
	  \textbf{There has been an increase in average lifetime of people of Bangladesh. To get more insight on this, a research was conducted, in which ages of retired government employees were recorded. A sample of 10 people is given below:}
	  
	  \begin{center}
	  75, 62, 63, 72, 66, 76, 59, 77, 70, 79
	  \end{center}
    \begin{enumerate}
    \item
	What is the 2nd central moment? \hfill 1
    \item
	Show that the first central moment is zero. \hfill 2
    \item  
	Find the variance of the data. \hfill 3
    \item
	Are the data symmetric? Justify. \hfill 4
  \end{enumerate}
  
   \item
	  \textbf{The first four moments about 3 of a distribution are -1, 5, -10, and 120.} 
  
  \begin{enumerate}
    \item
	What are moments used for? \hfill 1
    \item
	Can the second central moment be greater than the third central moment? \hfill 2
    \item  
	Find the second and third moments about arithmetic mean of the distribution. \hfill 3
    \item
	Find skewness and kurtosis and comment on the values.  \hfill 4
\end{enumerate}

 \item
	  \textbf{Annual sales of company are as given in the following}\
	  
	  \begin{table}[h]
	  \centering
\begin{tabular}{l|l|l|l|l|l|l|l}
Year & 2010 & 2011 & 2012 & 2013 & 2014 & 2015 & 2016 \\ \hline
Profit (million) & 40 & 45 & 46 & 53 & 65 & 70 & 73
\end{tabular}
\end{table}

  \begin{enumerate}
    \item
	What is a trend? \hfill 1
    \item
	Do the data in the stem seem to have a trend? \hfill 2
    \item  
	Find the trend using semi-average method. \hfill 3
    \item
	Find the trend using 2-yearly moving average method. Would it better if we used 3-yearly  \hfill 4 \\  method?
\end{enumerate}


   \item
	  \textbf{Every country has one or more agencies to deal with statistics of the country for proper management of its assets and population. Bangladesh Bureau of Statistics (BBS) serves as the centralized official bureau in Bangladesh for collecting and disseminating statistics in Bangladesh. USA has several such agencies, like Census Bureau or Bureau of Labor Statistics.} 
  
  \begin{enumerate}
    \item
	What is data? \hfill 1
    \item
	How is statistics important in planning?\hfill 2
    \item  
	Differentiate between official and non-official statistics. \hfill 3
    \item
	Elucidate the classification of published statistics in Bangladesh.  \hfill 4
  \end{enumerate}
  
\end{enumerate}
\end{document}