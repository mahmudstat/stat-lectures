\documentclass{exam}
%\documentclass[11pt,a4paper]{exam}
\usepackage{amsmath,amsthm,amsfonts,amssymb,dsfont}
\usepackage{ifthen}
\usepackage[legalpaper, total={177.8mm, 290mm}]{geometry}
\usepackage{enumerate}% http://ctan.org/pkg/enumerate
\usepackage{multicol}



% Accumulate the answers. Unmodified from Phil Hirschorn's answer
% https://tex.stackexchange.com/questions/15350/showing-solutions-of-the-questions-separately/15353
\newbox\allanswers
\setbox\allanswers=\vbox{}

\newenvironment{answer}
{%
    \global\setbox\allanswers=\vbox\bgroup
    \unvbox\allanswers
}%
{%
    \bigbreak
    \egroup
}

\newcommand{\showallanswers}{\par\unvbox\allanswers}
% End Phil's answer


% Is there a better way?
\newcommand*{\getanswer}[5]{%
    \ifthenelse{\equal{#5}{a}}
    {\begin{answer}\thequestion. (a)~#1\end{answer}}
    {\ifthenelse{\equal{#5}{b}}
        {\begin{answer}\thequestion. (b)~#2\end{answer}}
        {\ifthenelse{\equal{#5}{c}}
            {\begin{answer}\thequestion. (c)~#3\end{answer}}
            {\ifthenelse{\equal{#5}{d}}
                {\begin{answer}\thequestion. (d)~#4\end{answer}}
                {\begin{answer}\textbf{\thequestion. (#5)~Invalid answer choice.}\end{answer}}}}}
}

\setlength\parindent{0pt}
%usage \choice{ }{ }{ }{ }
%(A)(B)(C)(D)
\newcommand{\fourch}[5]{
    \par
    \begin{tabular}{*{4}{@{}p{0.23\textwidth}}}
        (a)~#1 & (b)~#2 & (c)~#3 & (d)~#4
    \end{tabular}
    \getanswer{#1}{#2}{#3}{#4}{#5}
}

%(A)(B)
%(C)(D)
\newcommand{\twoch}[5]{
    \par
    \begin{tabular}{*{2}{@{}p{0.46\textwidth}}}
        (a)~#1 & (b)~#2
    \end{tabular}
    \par
    \begin{tabular}{*{2}{@{}p{0.46\textwidth}}}
        (c)~#3 & (d)~#4
    \end{tabular}
    \getanswer{#1}{#2}{#3}{#4}{#5}
}

%(A)
%(B)
%(C)
%(D)
\newcommand{\onech}[5]{
    \par
    (a)~#1 \par (b)~#2 \par (c)~#3 \par (d)~#4
    \getanswer{#1}{#2}{#3}{#4}{#5}
}

\newlength\widthcha
\newlength\widthchb
\newlength\widthchc
\newlength\widthchd
\newlength\widthch
\newlength\tabmaxwidth

\setlength\tabmaxwidth{0.96\textwidth}
\newlength\fourthtabwidth
\setlength\fourthtabwidth{0.25\textwidth}
\newlength\halftabwidth
\setlength\halftabwidth{0.5\textwidth}

\newcommand{\choice}[5]{%
\settowidth\widthcha{AM.#1}\setlength{\widthch}{\widthcha}%
\settowidth\widthchb{BM.#2}%
\ifdim\widthch<\widthchb\relax\setlength{\widthch}{\widthchb}\fi%
    \settowidth\widthchb{CM.#3}%
\ifdim\widthch<\widthchb\relax\setlength{\widthch}{\widthchb}\fi%
    \settowidth\widthchb{DM.#4}%
\ifdim\widthch<\widthchb\relax\setlength{\widthch}{\widthchb}\fi%

% These if statements were bypassing the \onech option.
% \ifdim\widthch<\fourthtabwidth
%     \fourch{#1}{#2}{#3}{#4}{#5}
% \else\ifdim\widthch<\halftabwidth
% \ifdim\widthch>\fourthtabwidth
%     \twoch{#1}{#2}{#3}{#4}{#5}
% \else
%      \onech{#1}{#2}{#3}{#4}{#5}
%  \fi\fi\fi}

% Allows for the \onech option.
\ifdim\widthch>\halftabwidth
    \onech{#1}{#2}{#3}{#4}{#5}
\else\ifdim\widthch<\halftabwidth
\ifdim\widthch>\fourthtabwidth
    \twoch{#1}{#2}{#3}{#4}{#5}
\else
    \fourch{#1}{#2}{#3}{#4}{#5}
\fi\fi\fi}


\begin{document}

\begin{center}
  \bfseries\large
  Sylhet Cadet College

\normalsize
  First Term-End Examination - 2023

  Class: XI

  Subject: Statistics First Paper (MCQ)  \fbox{Set: A}

  Time: 20 minutes \qquad \qquad \qquad \qquad Subject Code: 129   \qquad \qquad \qquad  \qquad Full Marks: 25

%  \normalfont\normalsize
 % 11.45a.m.~--~1.45p.m.
\end{center}

\noindent
\begin{tabular}{p{\dimexpr\linewidth-2\tabcolsep}}
 Answer all the questions. Each question is worth one (1) mark.\\
  \hline
\end{tabular}

\begin{questions}

\question \textbf{Who is known as the Father of modern statistics?}
\choice{P.C. Mahalanobis}{Kazi Motaher Hossain}{Karl Pearson}{R.A. Fisher}{d}

\question \textbf{If $\displaystyle \sum_{i=1}^{20} x_i^2=20$ and $\displaystyle \sum_{i=1}^{20} x_i=30$, what is the value of $\displaystyle \sum_{i=1}^{20} x_i^2 + \sum_{i=1}^{20} x_i + 100$?}
\choice{130}{200}{230}{2130}{c}

\question \textbf{A subset of a population is called--}
\choice{Constant}{Variable}{Sample}{Scale}{c}

\textbf{Answer the next 2 question based on the following information.}

\textbf{A farmer collects growth (in cm) of 10 plants in a month and finds that \\ $\sum x_i = 7$ and $\sum x_i^2=15$}

\question \textbf{What is the value of $\sum (x_i+4)$?}
\choice{23}{$\sum x_i +4n$}{22}{11}{a}

\question \textbf{What is the value of $\sum (x_i-4)^2$?}
\choice{23}{135}{484}{121}{a}

\question \textbf{How many measurement scales are there?}
\choice{2}{3}{4}{5}{c}

\question \textbf{Which of the following is a continuous variable?}
\choice{Number of goals}{Natural number}{Summation of Fibonacci series}{Success rate}{d}

\question \textbf{In which scale of measurement, zero is regarded as true zero?}
\choice{Nominal scale}{Interval scale}{Ratio scale}{Ordinal scale}{c}

\question \textbf{How many sources of data are there?}
\choice{5}{4}{3}{2}{d}

\question \textbf{Data obtained through direct observation is called--}
\choice{Primary data}{Secondary data}{Original Data}{Informal data}{a}

\question \textbf{Who invented Stem and Leaf plot?}
\choice{Karl Pearson}{R.A. Fisher}{David Cox}{John Tukey}{d}

\question \textbf{How many measure of central tendency are there?}
\choice{2}{3}{4}{5}{d}

\question \textbf{Which measure of central tendencyis suitable for qualitative variable?}
\choice{Arithmetic Mean}{Harmonic Mean}{Quadratic Mean}{Mode}{d}

\textbf{Answer the next two (2) questions based on the following information}

\begin{table}[h]
\centering
\begin{tabular}{c|c|c|c|c|c|c}
Class                                                           & $\le 20$ & 20-25 & 25-50 & 50-60 & 69-70 & $\ge 70$ \\ \hline
Frequency                                                       & 5        & 10    & 10    & 7     & 5     & 3        \\ \hline
\begin{tabular}[c]{@{}c@{}}Cumulative \\ Frequency\end{tabular} & 5        & 15    & 25    & 32    & 37    & 40      
\end{tabular}
\end{table}

\question \textbf{How many values are between 20 and 70?}
\choice{20}{32}{35}{37}{b}

\question \textbf{Which one is the median class?}
\choice{20-25}{25-50}{50-60}{60-70}{b}

\question \textbf{In presence of negative values, which measure is not usable?}
\choice{Arithmetic Mean}{Geometric Mean}{Quadratic Mean}{Harmonic Mean}{b}

\question \textbf{For grouped data, which formula is correct for Arithmetic Mean?}
\choice{$\displaystyle \bar x = \frac{\sum f_ix_i}{\sum f_i}$}{$\displaystyle \bar x = \frac{\sum x_i}{N}$}{$\displaystyle \bar x = \frac{\sum f_ix_i}{n}$}{$ \displaystyle\bar x = \frac{\sum f_i}{N}$}{a}

\question \textbf{Arithmetic mean of the series 2, 12, 22, $\cdots$, 92 is--}
\choice{45}{46}{47}{55}{c}

\question \textbf{Median can be determined from the--}
\choice{Histogram}{Frequency curve}{Ogive}{Pie Chart}{c}

\question \textbf{The formula of coefficient of variance (CV) is --}
\choice{$\frac{\mu_2}{n}\times 100$}{$\frac{\mu_2}{\mu_1}\times 100$}{$\frac{\mu_2}{\bar x}\times 100$}{$\frac{\mu_3}{\sigma}\times 100$}{c}

\question \textbf{Which of the following is the best measure of dispersion?}
\choice{Range}{Mean deviation}{Standard deviation}{Coefficient of variation}{c}

\question \textbf{What is the minimum possible value of standard deviation?}
\choice{$\infty$}{-1}{0}{1}{c}

\question \textbf{For two values, range is found to be 8. What are the values of mean deviation and standard deviation}
\choice{(2,4)}{(4,4)}{(4,8)}{(8,8)}{a}

\question \textbf{What is the standard deviation of first 10 natural numbers?}
\choice{2.87}{3.02}{0}{2.78}{a}

\question \textbf{Which measure is unit-free?}
\choice{Range}{Mean deviation}{Standard deviation}{Coefficient of variation}{d}

%\question \textbf{To complete the song, the last answer should be
%\choice{a}{b}{c}{d}{e} % Invalid answer choice

\end{questions}

\pagebreak
%\newpage  %Uncomment to put on new age
\bigskip

\begin{multicols}{3}
[
Answer Key
]
\showallanswers % Phil Hirschorn
\end{multicols}


\end{document}