\documentclass{article}
\usepackage{geometry}
\usepackage{amsfonts}



\geometry{
legalpaper, total={177.8mm, 290mm},left=20mm,
top=10mm, bottom=10mm,
}

\begin{document}

\begin{center}
  \bfseries\large
  Sylhet Cadet College

\normalsize
  First Term-End Examination - 2023

  Class: XI

  Subject: Statistics First Paper (Creative)

  Time: 2 hour \& 35 minutes \qquad \qquad \qquad Subject Code: 129   \qquad \qquad Full Marks: 50

%  \normalfont\normalsize
 % 11.45a.m.~--~1.45p.m.
\end{center}

\noindent
\begin{tabular}{p{\dimexpr\linewidth-2\tabcolsep}}
  Answer FIVE questions taking at least two (2) from each group. Figures in the right indicate full marks.\\
  \hline
\end{tabular}

\begin{center}
\textbf{Group A}
\end{center}
  \begin{enumerate}

 \item
	  \textbf{Height (in inches) of 10 cadets in a class are: 50, 60, 55, 65, 66, 70, 54, 64, 62, 72} 
	 
  \begin{enumerate}
    \item
	What is population in statistics? \hfill 1
    \item
	Is height discrete or continuous? \hfill 2
    \item  
	Find $\displaystyle \sum_{i=1}^{10} x_i^2$ \hfill 3
    \item
	Find the square of mean and mean of square. Are they equal? \hfill 4
  \end{enumerate}
  
     \item
	  \textbf{An analyst obtains some data:}
	  \begin{center}
	  $x_1=15, x_2=-12, x_3=17, x_4=11, x_5=23$
  \end{center}
  \begin{enumerate}
    \item
	What is sample? \hfill 1
    \item
	Briefly explain shift or origin and scale. \hfill 2
    \item  
	Compute the value of $\displaystyle \sum_{i=1}^5 (x_i-10)^2$ \hfill 3
    \item
	Find the value of $\displaystyle \sum_{i=1}^5 (5x_i^2-4x_i-3)$ and examine its dependency on origin and scale. \hfill 4
  \end{enumerate}

  
         \item
	  \textbf{Hourly wages of 100 workers in an idustry were collected by a market analyst. The analyst desires to mine a patter and useful insights from the collected data about the industry. The obtained data are demonstrated below:}
	  
	  \begin{table}[h]
	  \centering
\begin{tabular}{c|c|c|c|c|c|c|c}
Wage              & 51-55 & 56-60 & 61-65 & 66-70 & 71-75 & 76-80 & 81-85 \\ \hline
Number of workers & 7     & 11    & 18    & 36    & 15    & 8     & 5    
\end{tabular}
\end{table}

  \begin{enumerate}
    \item
	What is class interval? \hfill 1
    \item
	How does a frequency distribution help us to find patter in data? \hfill 2
    \item  
	Draw an Ogive from the data provided and explain. \hfill 3
    \item
	Write five useful insights about the data combining information from Ogive and the table. \hfill 4
  \end{enumerate}
  
 \item
	  \textbf{For two non-zero positive numbers, $GM=4\sqrt3$ and $HM=6$, where the quantities bear usual notations}  
  \begin{enumerate}
    \item
	When is Harmonic mean suitable? \hfill 1
    \item
	For two numbers, what is the relationship between AM, GM, and HM? \hfill 2
    \item  
	What is the Arithmetic mean? \hfill 3
    \item
	Determine the numbers. \hfill 4
  \end{enumerate}

\begin{center}
\textbf{Group B}
\end{center}

     \item
	  \textbf{12 is deducted from each value of a variable and then divided by 3. The new arithmetic mean (AM) is found to be 4.} 
  
  \begin{enumerate}
    \item
	What is change of origin? \hfill 1
    \item
	Does AM depend on origin? Prove with an example. \hfill 2
    \item  
	From the stem, find the original AM. \hfill 3
    \item
	Does the origin or the scale have greater impact on AM in this example? \hfill 4
  \end{enumerate}
  
    \item
  \textbf{In the test examination, marks of 11 students in statistics are: 90, 92, 93, 49, 44, 88, 80, 58, 83, 71, 76.}
  \begin{enumerate}
    \item
	What is central tendency? \hfill 1
    \item
	When is median better than arithmetic mean? Explain with an example. \hfill 2
    \item  
	Find the 3rd the quartile and 61st percentile from the data and explain.  \hfill 3
    \item
	Do quantiles depend on change of origin and scale. Prove using two examples.\hfill 4
\end{enumerate}
  
\end{enumerate}
\end{document}