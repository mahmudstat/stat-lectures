\documentclass{article}
\usepackage{geometry}
\usepackage{amsfonts}

\geometry{
legalpaper, total={177.8mm, 290mm},left=20mm,
top=27mm, bottom=27mm,
}

\begin{document}

\begin{table}[h]
\centering
\begin{tabular}{lllll}
\textbf{\large SYLHET CADET COLLEGE} &  &  &  &  \\ \cline{4-5} 
FIRST TERM-END EXAMINATION - 2023 &  & \multicolumn{1}{l|}{} & \multicolumn{1}{l|}{Set} & \multicolumn{1}{l|}{A} \\ \cline{4-5} 
CLASS: XI &  &  &  &  \\ \cline{3-5} 
STATISTICS (CREATIVE)& \multicolumn{1}{l|}{\textbf{Subject Code:}} & \multicolumn{1}{l|}{1} & \multicolumn{1}{l|}{2} & \multicolumn{1}{l|}{9} \\ \cline{3-5} 
 FIRST PAPER &  &  &  &  \\
TIME – 2 hours \& 35 minutes &  &  &  &  \\
FULL MARKS – 50 &  &  &  & 
\end{tabular}
\end{table}
%  \normalfont\normalsize
 % 11.45a.m.~--~1.45p.m.

\hrule

\begin{center}
[\textbf{N.B.} – The figures of the right margin indicate full marks. Read the stems carefully and answer the associated questions. Answer any \textbf{FIVE} questions taking at least two from each group.]\\

\end{center}

  \begin{center}
  \textbf{Group--A}
  \end{center}
  
  \begin{enumerate}
  
 \item
	  \textbf{Marks of 10 students in Statistics in a class were found to be the following:} 
	 
	 \begin{center} 
	  99, 88, 98, 85, 97, 71, 87, 79, 70, 84
	  
	  \end{center}
	  
	  Later it was discovered that all marks should be 5 less than the recorded marks. 
  
  \begin{enumerate}
    \item
	What is change of origin? \hfill 1
    \item
	Does summation of a variable depend an change of origin?  \hfill 2
    \item  
	Considering the data in stem as X, find $\displaystyle \sum_{i=1}^{10} X_i$ and $\displaystyle \sum_{i=1}^{10} (X_i+3)$ \hfill 3
    \item
	Find the arithmetic mean of the corrected values, employing the concept of shift of origin. \hfill 4
  \end{enumerate}
  
  \item
	  \textbf{Goals scored by a footballer in 25 matches are summarized as shown below.} 
	  
	  \begin{table}[h]
	  \centering
\begin{tabular}{|c|ccccc|}
Goals & 0 & 1 & 2 & 3 & 4 \\ \hline
Times & 8 & 9 & 5 & 2 & 1
\end{tabular}
\end{table}
  
  \begin{enumerate}
    \item
	Is no. goals a discrete or continuous variable? \hfill 1
    \item
	Verify theoretically: $\displaystyle \sum_{i=2}^{2} X_iY_i = \sum_{i=1}^{2} X_i \times \sum_{i=1}^{2} Y_i$ \hfill 2
    \item  
	Find the total number of goals using a suitable notation. \hfill 3
    \item
	If he scores two (2) goals in the next match, will the scoring rate increase? \hfill 4
  \end{enumerate}
  
   \item
	  \textbf{Scores by Travis Head in the last two matches of ICC Men's Cricekt World - 2023 are given. In Cricket, Strike Rate (SR) is computed by dividing Balls by Runs and then multiplying the quotient by 100.} 
	  
	  \begin{table}[h]
	  \centering
\begin{tabular}{c|c|c}
Match & Runs & Balls \\ \hline 
1 & 62 & 48 \\
2 & 137 & 120 \\ \hline 
\end{tabular}
\end{table}
  
  \begin{enumerate}
    \item
	How many averages do you know of? \hfill 1
    \item
	Give an example when arithmetic mean is appropriate instead of harmonic mean. \hfill 2
    \item  
	When is Weighted Harmonic mean is used. Show a numerical example. \hfill 3
    \item
	Determine the average Strike Rate of the batter \hfill 4
  \end{enumerate}
  
     \item
	  \textbf{Average height of the four tallest towers in Dhaka is 153.25 meters. The heights of first three towers is 171, 153 and 152 meters, respectively. A new tower has been built with height 150 meters.} 
  
  \begin{enumerate}
    \item
	Write two primary uses of central tendency. \hfill 1
    \item
	Prove mathematically: $\displaystyle \sum_{i=1}^n (x_i-\bar x) = 0$ \hfill 2
    \item  
	Compute the height of the forth tower. \hfill 3
    \item
	After the addition of the new tower, will the average increase or decrease? Explain logically and empirically (using data). \hfill 4
  \end{enumerate}
  
    \begin{center}
  \textbf{Group--B}
  \end{center}
  
     \item
	  \textbf{Duration of stays of a spy in foreign countries are obtained by a researcher. As part of an analysis, s/he starts with the following summary.} 
	  
	  \begin{table}[h]
	  \centering
\begin{tabular}{c|cccccc}
Duration & 1-10 & 11-20 & 21-30 & 31-40 & 41-50 & 51-60 \\ \hline
Frequency & 4 & 3 & 3 & 2 & 5 & 2
\end{tabular}
\end{table}
  
  \begin{enumerate}
    \item
	What is symmetry? \hfill 1
    \item
	What is implied by the value of coefficient of skewness 0.8 \hfill 2
    \item  
	Estimate the median of the data and interpret. \hfill 3
    \item
	Obtain coefficient of skewness from data and comment on the life of the spy based on it. \hfill 4
  \end{enumerate}
  
 \item
	  \textbf{Marks obtained by a student in 7 subjects are} 
	  \begin{center}
	  70, 66, 55, 45, 80, 30, 82
	\end{center}
  
  \begin{enumerate}
    \item
	What is negative skewness? \hfill 1
    \item
	Draw graphs of positive and negative skewness showing the locations of mean and median. \hfill 2
    \item  
	Determine the five number summary from the stem and explain. \hfill 3
    \item
	Are the data symmetric? If not, comment on the pattern of data. \hfill 4
\end{enumerate}

 \item
	  \textbf{Average monthly temperatures (in $^o C$) in the city of Sylhet are collected by an analyst. The analyst assumes the next month will not follow the current trend.} 
	  
	  \begin{table}[h]
	  \centering
\begin{tabular}{c|c|c|c|c|c|c|c}
Month & Jan & Feb & Mar & Apr & May & Jun & Jul \\ \hline
Temperature & 25.2 & 27.1 & 30.4 & 30.8 & 30.9 & 30.9 & 31.6
\end{tabular}
\end{table}
  
  \begin{enumerate}
    \item
	What is seasonal variation? \hfill 1
    \item
	Differentiate between seasonal variation and cyclic variation. \hfill 2
    \item  
	Find the general trend using semi-average method. \hfill 3
    \item
	Find the trend using moving average method and examine the assumption of the analyst. [the genuine next value is 31.2] \hfill 4
  \end{enumerate}
  
   \item
	  \textbf{In 2015, tens of thousands of Rohingya people were forcibly displaced from their villages and IDP camps in Rakhine state, Mynmar. Many of them fled to neighboring countries, including Bangladesh, Malaysia, Indonesia. Many national and internations agencies collect data on the issue.} 
  
  \begin{enumerate}
    \item
	What is non-official statistics? \hfill 1
    \item
	Name five sources of official statistics. \hfill 2
    \item  
	Shed some light on the limitations of official statistics. \hfill 3
    \item
	How can the quality of published statistics in Bangladesh be improved? \hfill 4
  \end{enumerate}


\begin{center}
\textbf{\textit{Absence of evidence is not evidence of absence.} – Carl Sagan}
\end{center}
  
\end{enumerate}
\end{document}