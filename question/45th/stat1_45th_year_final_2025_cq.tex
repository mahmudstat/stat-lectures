\documentclass[12pt]{article}
\usepackage{geometry}
\usepackage{amsfonts}
\usepackage{float}
\usepackage{amsmath}

\geometry{
legalpaper, total={177.8mm, 290mm},left=10mm, right=10mm,
top=7mm, bottom=12mm,
}


\begin{document}

\begin{table}[]
\begin{tabular}{llllllcllrlll}
\textit{Polaris} &  &  &  &  &  & \textbf{SYLHET CADET COLLEGE}       &  &                       & \multicolumn{1}{l}{}                        &                        &                        &                        \\ \cline{10-13} 
                 &  &  &  &  &  & YEAR-FINAL EXAMINATION - 2025             &  & \multicolumn{1}{l|}{} & \multicolumn{1}{c|}{Ques Setter}            & \multicolumn{3}{l|}{}                                                    \\ \cline{10-13} 
                 &  &  &  &  &  & CLASS: XI                         &  & \multicolumn{1}{l|}{} & \multicolumn{1}{c|}{Moderator}              & \multicolumn{3}{l|}{}                                                    \\ \cline{10-13} 
                 &  &  &  &  &  & STATISTICS (CREATIVE)               &  & \multicolumn{1}{l|}{} & \multicolumn{1}{c|}{VP}                     & \multicolumn{3}{l|}{}                                                    \\ \cline{10-13} 
                 &  &  &  &  &  & FIRST PAPER                        &  &                       &                                             &                        &                        &                        \\ \cline{11-13} 
                 &  &  &  &  &  & [According to the Syllabus of 2026] &  &                       & \multicolumn{1}{r|}{\textbf{Subject Code:}} & \multicolumn{1}{l|}{1} & \multicolumn{1}{l|}{3} & \multicolumn{1}{l|}{0} \\ \cline{11-13} 
                 &  &  &  &  &  & TIME – 2 hours \& 25 minutes        &  &                       &                                             &                        &                        &                        \\
                 &  &  &  &  &  & FULL MARKS – 50                     &  &                       & \textbf{}                                   &                        &                        &                       
\end{tabular}
\end{table}

\hrule

\begin{center}
[\textbf{N.B.} – The figures of the right margin indicate full marks. Read the stems carefully and answer the associated questions. Answer any \textbf{FIVE} questions taking at least two questions from each group]\\
\end{center}

\begin{center}
\textbf{Group  - A}
\end{center}

  \begin{enumerate}
  
\item
\textbf{A botanist is measuring the heights (in centimeters) of four seedlings after one week. The observed heights are:}
\begin{center}
$h_1 = 15, h_2 = 12, h_3 = 18, h_4 = 10$
\end{center}
\begin{enumerate}
    \item What is a qualitative variable? \hfill 1
    \item Differentiate between ratio and interval scale. \hfill 2
    \item Compute the value of $\displaystyle \sum_{i=1}^4 (h_i - 14)^2$. \hfill 3
    \item Calculate $\displaystyle \sum_{i=1}^4 (3h_i^2 - 2h_i + 1)$ using both a direct approach and by splitting the summation terms. Also demonstrate that they are mathematically equivalent. \hfill 4
\end{enumerate}

  \item
  \textbf{A school health survey was conducted to understand the weight distribution among students. As part of this initiative, the weights (in kilograms) of 20 randomly selected students from different grades were measured and recorded to assess the general health status and identify patterns or anomalies in weight across the group. The collected data are as follows:}

  \begin{center}
  48, 52, 55, 50, 60, 62, 53, 58, 51, 54 \\
  56, 59, 49, 47, 61, 63, 57, 46, 45, 50 \\
  \end{center}

\begin{enumerate}
  \item What is the purpose of a frequency distribution? \hfill 1
  \item Differentiate between primary and secondary data. \hfill 2
  \item  
  Construct a frequency distribution table using a suitable class interval. \hfill 3

  \item
  Create an Ogive and hence estimate the quartiles and interpret. \hfill 4
\end{enumerate}


  \item
    \textbf{Scores of four athletes in different events at a track meet are recorded below:}

\begin{table}[h]
\centering
\begin{tabular}{c|c|c|c|c}
\hline
Event & High Jump & Long Jump & Shot Put & Javelin Throw \\ \hline
Score & 8.5 & 7.2 & 12.8 & 55.5 \\ \hline
Difficulty Factor & 2 & 3 & 2.5 & 1.5 \\ \hline
\end{tabular}
\end{table}

\textbf{A coach believes that events with higher difficulty factors should contribute more to the overall ranking and suggested a new weighting where the weight for each event is the square of its difficulty factor.}

  \begin{enumerate}
    \item
	Write down the formula of weighted mean. \hfill 1
    \item
	What is difference between weight and frequency? \hfill 2
	 \item Calculate the weighted average score of the athletes across all four events. \hfill 3
    \item If the coach's suggestion is implemented, would the mean be shifted upward or downward? Show mathematically and empricially. \hfill 4
  \end{enumerate}

\item  
  \textbf{Monthly sales (in thousand dollars) of two stores over five months are given below:}

\begin{table}[h]
\centering
\begin{tabular}{c|ccccc}
Month     & 1  & 2  & 3  & 4  & 5  \\ \hline
Store P  & 50 & 55 & 48 & 52 & 49 \\
Store Q  & 60 & 58 & 65 & 63 & 61
\end{tabular}
\end{table}

  \begin{enumerate}
   \item Is Range influenced by extreme values or outliers? \hfill 1
   \item Does Mean Deviation depend on change of origin and scale? Verify. \hfill 2
    \item  
	Calculate the variance of Store P’s sales data.  \hfill 3  
    \item  
	Compare the sales stability of both stores using an appropriate statistical measure.  \hfill 4  
\end{enumerate}  


% Questions here

\begin{center}
\textbf{Group  - B}
\end{center}
  
     \item
	  \textbf{The first four moments around 4 of productions of a company over four years are -1.5, 17, -30, and 108.} 
  
  \begin{enumerate}
  \item Draw a symmetrical distribution.   \hfill 1
    \item
	Can the second central moment be negative? \hfill 2
    \item  
	Determine the second and third central moments. \hfill 3
    \item
	What kind of kurtosis do the data have? \hfill 4
  \end{enumerate}


  \item
	  \textbf{The following table shows the exam scores of 10 students and the number of hours they studied for the exam. It is hypothesized that the exam score depends on the number of hours studied.} 
	  
\begin{table}[h!]
\centering
\begin{tabular}{c|c|c|c|c|c|c|c|c|c|c}
Hours Studied (X) & 2 & 3 & 4 & 5 & 6 & 7 & 8 & 9 & 10 & 11 \\ \hline
Exam Score (Y) & 65 & 70 & 72 & 78 & 80 & 85 & 88 & 92 & 95 & 98 \\
\end{tabular}
\end{table}
  
  \begin{enumerate}
    \item What does $r=-1$ imply? \hfill 1
  \item Does the regression coefficient depend on scale? Verify mathematically. \hfill 2
    \item  
	Estimate and interpret the regression coefficient of y on x. \hfill 3
    \item
	Using data, show $r = \sqrt{b_{yx} \times b_{xy}}$; How much variation is explained by the obtained model? \hfill 4
  \end{enumerate}
  
\item
\textbf{The average monthly temperature (in degrees Celsius) recorded at a weather station over seven months is given below:}

\begin{table}[H]
\centering
\begin{tabular}{c|c|c|c|c|c|c|c}
Month & September & October & November & December & January & February & March \\ \hline
Temperature & 22 & 20 & 18 & 15 & 12 & 16 & 20
\end{tabular}
\end{table}

\begin{enumerate}
  \item What is trend?  \hfill 1
  \item What are the components of Time Series?  \hfill 2
  \item
Compute the trend using the three-monthly moving average method. \hfill 3
  \item
Estimate the approximate average temperature for the month of April using both graphical and moving average methods, and compare the projections. \hfill 4
\end{enumerate}  
  
  
\end{enumerate}

\end{document}