\documentclass[12pt]{article}
\usepackage{geometry}
\usepackage{amsfonts}
\usepackage{float}
\usepackage{amsmath}

\geometry{
legalpaper, total={177.8mm, 290mm},left=10mm, right=10mm,
top=7mm, bottom=10mm,
}


\begin{document}

\begin{table}[]
\begin{tabular}{llllllcllrlll}
\textit{Polaris} &  &  &  &  &  & \textbf{SYLHET CADET COLLEGE}       &  &                       & \multicolumn{1}{l}{}                        &                        &                        &                        \\ \cline{10-13} 
                 &  &  &  &  &  & YEAR-FINAL EXAMINATION - 2025             &  & \multicolumn{1}{l|}{} & \multicolumn{1}{c|}{Ques Setter}            & \multicolumn{3}{l|}{}                                                    \\ \cline{10-13} 
                 &  &  &  &  &  & CLASS: XI                         &  & \multicolumn{1}{l|}{} & \multicolumn{1}{c|}{Moderator}              & \multicolumn{3}{l|}{}                                                    \\ \cline{10-13} 
                 &  &  &  &  &  & STATISTICS (CREATIVE)               &  & \multicolumn{1}{l|}{} & \multicolumn{1}{c|}{VP}                     & \multicolumn{3}{l|}{}                                                    \\ \cline{10-13} 
                 &  &  &  &  &  & FIRST PAPER                        &  &                       &                                             &                        &                        &                        \\ \cline{11-13} 
                 &  &  &  &  &  & [According to the Syllabus of 2025] &  &                       & \multicolumn{1}{r|}{\textbf{Subject Code:}} & \multicolumn{1}{l|}{1} & \multicolumn{1}{l|}{3} & \multicolumn{1}{l|}{0} \\ \cline{11-13} 
                 &  &  &  &  &  & TIME – 2 hours \& 25 minutes        &  &                       &                                             &                        &                        &                        \\
                 &  &  &  &  &  & FULL MARKS – 50                     &  &                       & \textbf{}                                   &                        &                        &                       
\end{tabular}
\end{table}

\hrule

\begin{center}
[\textbf{N.B.} – The figures of the right margin indicate full marks. Read the stems carefully and answer the associated questions. Answer any \textbf{FIVE} questions taking at least two questions from each group]\\
\end{center}

\begin{center}
\textbf{Group  - A}
\end{center}

  \begin{enumerate}
  
\item
\textbf{A botanist is measuring the heights (in centimeters) of four seedlings after one week. The observed heights are:}
\begin{center}
$h_1 = 15, h_2 = 12, h_3 = 18, h_4 = 10$
\end{center}
\begin{enumerate}
    \item Compute the value of $\displaystyle \sum_{i=1}^4 (h_i - 14)^2$. \hfill 3
    \item Calculate $\displaystyle \sum_{i=1}^4 (3h_i^2 - 2h_i + 1)$ using both a direct approach and by splitting the summation terms. Also demonstrate that they are mathematically equivalent. \hfill 4
\end{enumerate}

  \item
    \textbf{Scores of four athletes in different events at a track meet are recorded below:}

\begin{table}[h]
\centering
\begin{tabular}{c|c|c|c|c}
\hline
Event & High Jump & Long Jump & Shot Put & Javelin Throw \\ \hline
Score & 8.5 & 7.2 & 12.8 & 55.5 \\ \hline
Difficulty Factor & 2 & 3 & 2.5 & 1.5 \\ \hline
\end{tabular}
\end{table}

\textbf{A coach believes that events with higher difficulty factors should contribute more to the overall ranking and suggested a new weighting where the weight for each event is the square of its difficulty factor.}

  \begin{enumerate}
    \item
	Write down the formula of weighted mean. \hfill 1
    \item
	What is difference between weight and frequency? \hfill 2
	 \item Calculate the weighted average score of the athletes across all four events. \hfill 3
    \item If the coach's suggestion is implemented, would the mean be shifted upward or downward? Show mathematically and empricially. \hfill 4
  \end{enumerate}


% Questions here

\begin{center}
\textbf{Group  - B}
\end{center}
  
     \item
	  \textbf{The first four moments around 4 of productions of a company over four years are -1.5, 17, -30, and 108.} 
  
  \begin{enumerate}
    \item
	Can the second central moment be negative? \hfill 2
    \item  
	Determine the second and third central moments. \hfill 3
    \item
	What kind of kurtosis do the data have? \hfill 4
  \end{enumerate}
\end{enumerate}

 \vspace{2.5cm}

\begin{center}
 “Quote” \\ -- Author
\end{center}

\end{document}