\documentclass[12pt]{exam}
%\documentclass[11pt,a4paper]{exam}
\usepackage{amsmath,amsthm,amsfonts,amssymb,dsfont}
\usepackage{float}
\usepackage{ifthen}
\usepackage{array}
\usepackage{geometry}
\geometry{
legalpaper, total={177.8mm, 290mm},left=10mm, right=10mm,
top=7mm, bottom=27mm,
}
\usepackage{enumerate}% http://ctan.org/pkg/enumerate
\usepackage{multicol}
\usepackage{hhline}
\usepackage[table]{xcolor}


% Accumulate the answers. Unmodified from Phil Hirschorn's answer
% https://tex.stackexchange.com/questions/15350/showing-solutions-of-the-questions-separately/15353
\newbox\allanswers
\setbox\allanswers=\vbox{}

\newenvironment{answer}
{%
    \global\setbox\allanswers=\vbox\bgroup
    \unvbox\allanswers
}%
{%
    \bigbreak
    \egroup
}

\newcommand{\showallanswers}{\par\unvbox\allanswers}
% End Phil's answer


% Is there a better way?
\newcommand*{\getanswer}[5]{%
    \ifthenelse{\equal{#5}{a}}
    {\begin{answer}\thequestion. (a)~#1\end{answer}}
    {\ifthenelse{\equal{#5}{b}}
        {\begin{answer}\thequestion. (b)~#2\end{answer}}
        {\ifthenelse{\equal{#5}{c}}
            {\begin{answer}\thequestion. (c)~#3\end{answer}}
            {\ifthenelse{\equal{#5}{d}}
                {\begin{answer}\thequestion. (d)~#4\end{answer}}
                {\begin{answer}\textbf{\thequestion. (#5)~Invalid answer choice.}\end{answer}}}}}
}

\setlength\parindent{0pt}
%usage \choice{ }{ }{ }{ }
%(A)(B)(C)(D)
\newcommand{\fourch}[5]{
    \par
    \begin{tabular}{*{4}{@{}p{0.23\textwidth}}}
        (a)~#1 & (b)~#2 & (c)~#3 & (d)~#4
    \end{tabular}
    \getanswer{#1}{#2}{#3}{#4}{#5}
}

%(A)(B)
%(C)(D)
\newcommand{\twoch}[5]{
    \par
    \begin{tabular}{*{2}{@{}p{0.46\textwidth}}}
        (a)~#1 & (b)~#2
    \end{tabular}
    \par
    \begin{tabular}{*{2}{@{}p{0.46\textwidth}}}
        (c)~#3 & (d)~#4
    \end{tabular}
    \getanswer{#1}{#2}{#3}{#4}{#5}
}

%(A)
%(B)
%(C)
%(D)
\newcommand{\onech}[5]{
    \par
    (a)~#1 \par (b)~#2 \par (c)~#3 \par (d)~#4
    \getanswer{#1}{#2}{#3}{#4}{#5}
}

\newlength\widthcha
\newlength\widthchb
\newlength\widthchc
\newlength\widthchd
\newlength\widthch
\newlength\tabmaxwidth

\setlength\tabmaxwidth{0.96\textwidth}
\newlength\fourthtabwidth
\setlength\fourthtabwidth{0.25\textwidth}
\newlength\halftabwidth
\setlength\halftabwidth{0.5\textwidth}

\newcommand{\choice}[5]{%
\settowidth\widthcha{AM.#1}\setlength{\widthch}{\widthcha}%
\settowidth\widthchb{BM.#2}%
\ifdim\widthch<\widthchb\relax\setlength{\widthch}{\widthchb}\fi%
    \settowidth\widthchb{CM.#3}%
\ifdim\widthch<\widthchb\relax\setlength{\widthch}{\widthchb}\fi%
    \settowidth\widthchb{DM.#4}%
\ifdim\widthch<\widthchb\relax\setlength{\widthch}{\widthchb}\fi%

% These if statements were bypassing the \onech option.
% \ifdim\widthch<\fourthtabwidth
%     \fourch{#1}{#2}{#3}{#4}{#5}
% \else\ifdim\widthch<\halftabwidth
% \ifdim\widthch>\fourthtabwidth
%     \twoch{#1}{#2}{#3}{#4}{#5}
% \else
%      \onech{#1}{#2}{#3}{#4}{#5}
%  \fi\fi\fi}

% Allows for the \onech option.
\ifdim\widthch>\halftabwidth
    \onech{#1}{#2}{#3}{#4}{#5}
\else\ifdim\widthch<\halftabwidth
\ifdim\widthch>\fourthtabwidth
    \twoch{#1}{#2}{#3}{#4}{#5}
\else
    \fourch{#1}{#2}{#3}{#4}{#5}
\fi\fi\fi}


\begin{document}

\begin{table}[]
\begin{tabular}{lllllllllrlll}
\textit{Corr} &  &  &  &  &  & \textbf{SYLHET CADET COLLEGE}       &  &                       & \multicolumn{1}{l}{}                        &                        &                        &                        \\ \cline{10-13} 
       &  &  &  &  &  & Year Final EXAMINATION - 2025             &  & \multicolumn{1}{l|}{} & \multicolumn{1}{c|}{Ques Setter}            & \multicolumn{3}{l|}{}                                                    \\ \cline{10-13} 
       &  &  &  &  &  & CLASS: XI                          &  & \multicolumn{1}{l|}{} & \multicolumn{1}{c|}{Moderator}              & \multicolumn{3}{l|}{}                                                    \\ \cline{10-13} 
       &  &  &  &  &  & SAQ and MCQ           &  & \multicolumn{1}{l|}{} & \multicolumn{1}{c|}{VP}                     & \multicolumn{3}{l|}{}                                                    \\ \cline{10-13} 
       &  &  &  &  &  & STATISTICS                          &  &                       &                                             &                        &                        &                        \\ \cline{11-13} 
       &  &  &  &  &  & FIRST PAPER                        &  &                       & \multicolumn{1}{r|}{\textbf{Subject Code:}} & \multicolumn{1}{l|}{1} & \multicolumn{1}{l|}{2} & \multicolumn{1}{l|}{9} \\ \cline{11-13} 
       &  &  &  &  &  & [According to the Syllabus of 2025] &  &                       &                                             &                        &                        &                        \\ \cline{12-12}
       &  &  &  &  &  & TIME – 25 minutes                   &  &                       & \textbf{Set:}                               & \multicolumn{1}{l|}{}  & \multicolumn{1}{l|}{C} &                        \\ \cline{12-12}
       &  &  &  &  &  & FULL MARKS – 25                     &  &                       & \multicolumn{1}{l}{}                        &                        &                        &                       
\end{tabular}
\end{table}



%  \normalfont\normalsize
 % 11.45a.m.~--~1.45p.m.
\hrule

\begin{center}
[N.B. – Answer all the questions. Each question carries ONE mark. Block fully, with a black ball- point pen, the circle of the letter that stands for the correct/best answer in the “Answer sheet” for the Multiple Choice Questions Examination.]\\

  
  \textbf{Candidates are asked not to leave any mark or spot on the question paper.}
\end{center}
\begin{questions}



\question \textbf{Which is not an example of shift of scale?}
\choice{$y_i = \frac{x_i}{a}$}{$y_i = cx_i$}{$y_i = x_i-2$}
{$y_i = \frac{cx_i}{d}$}{a}

\question \textbf{Given $\displaystyle \sum_{i=1}^{10} a_i^2=40$ and $\displaystyle \sum_{i=1}^{10} a_i=20$, find the value of $\displaystyle 2\sum_{i=1}^{10} a_i^2 - 3\sum_{i=1}^{10} a_i + 60$.}  
\choice{70}{100}{80}{50}{c} 


\question \textbf{A researcher collected data on age and income of the 
people in a city. The variables are --}

i. bi-variate \\
ii. quantitative \\
iii. qualitative

\textbf{Which one is correct?}

\choice{i and ii}{i and iii}{ii and iii}{i, ii and iii}{a}

% Situation Set Starts
\textbf{Answer the next two questions based on the following plot}

\begin{center}
\noindent\textbf{Data:} 18, 21, 22, 23, 24, 26, 31, 33, 33, 35, 37, 42

\bigskip

\begin{tabular}{r|l}
\textbf{Stem} & \textbf{Leaf} \\
\hline
1 & 8 \\
2 & 1 2 3 4 6 \\
3 & 1 3 3 5 7 \\
4 & 2 \\
\end{tabular}

\bigskip

\noindent\textbf{Key:} 2~|~1 means \textbf{21}
\end{center}

\question \textbf{How many data values are greater than 30 in the stem-and-leaf plot?}  
\choice{3}{4}{5}{6}{d}

\question \textbf{What is the median of the data shown in the stem-and-leaf plot?}  
\choice{26}{31}{30}{29}{b}
% Situation Set Ends


\question \textbf{If $\sum (x_i-k)=0$, what is the value of k?}
\choice{$n$}{$\bar x$}{$x$}{$n \bar x$}{b}

\question \textbf{Median is --}  

i. Affected by extreme values \\  
ii. Rigidly defined \\  
iii. Suitable for open-ended distributions  

\textbf{Which one is correct?}  

\choice{i and ii}{i and iii}{ii and iii}{i, ii and iii}{b}  

\question \textbf{Which of the following may be used to determine mode?}
\choice{Histogram}{Frequency Curve}{Ogive}{Frequency Polygon}{a}

\question \textbf{What is the minimum possible value of standard deviation?}
\choice{$\infty$}{-1}{0}{1}{c}


\question \textbf{The mean and coefficient of variation of a distribution are 5 and 30\%, respectively. What is the value of standard deviation?}
\choice{1.5}{6.5}{7.6}{10.2}{a}

% Situation Set Starts
\textbf{Answer the next two questions based on the following information}

\begin{center}
\textbf{The temperatures (in $^oC$ of two cities in a country are 30 and 35.} 
\end{center}

\question \textbf{What is their Mean deviation?}
\choice{1.2}{2.5}{3.0}{5.5}{b}

\question \textbf{What is the coefficient of variation?}
\choice{2.7\%}{8.3\%}{5.8\%}{7.7\%}{d}
% Situation Set Ends

% Situation Set Starts
\textbf{Answer the next two questions based on the following information}

\begin{center}
A study was conducted to find the impact of study hour on students' GPA and the following was found:

$\displaystyle \sum (x_i- \bar x)(y_i - \bar) = 30, \sum(x_i-\bar x)^2 = 45, \text{ and } \sum(y_i-\bar y)^2 = 55$
\end{center}

\question \textbf{What is the value of correlation coefficient?}
\choice{0.50}{0.60}{-0.60}{-0.50}{b}

\question \textbf{What is the value of $b_{yx}?$}
\choice{0.58}{-0.67}{0.67}{-1.75}{c}
% Situation Set Ends
\end{questions}

\textbf{SAQ} \hfill $10 \times 1 =10$
\begin{enumerate}
  \item If the scores of five students in a test are 78, 85, 92, 88, 95, 
find $\displaystyle \sum_{i=1}^5 (x_i^2 - 2x_i + 3)$ \hfill 1
  \item What is an open-ended distribution? \hfill 1
  \item Does Median lie in the data set from which it is calculated? \hfill 1
  \item If $\bar X = 25, CV = 50\%, \sigma^2=?$ \hfill 1
  \item Which Percentile is equal to 3rd Quartile?
  \item  What is the variance of first 5 natural numbers?
  \item What does $\gamma_2>0$ imply? \hfill 1
  \item Which measure of dispersion is suitable for an open-ended distribution?  \underline{\hspace{3cm}}
  \item What is the range of the regression coefficient? --- \hfill 1
\end{enumerate}






 \vspace{2.5cm}

\begin{center}
 “Quote” \\ -- Author
\end{center}

\pagebreak
%\newpage  %Uncomment to put on new age
\bigskip

\iffalse
\begin{multicols}{3}
[
Answer Key
]
\showallanswers % Phil Hirschorn
\end{multicols}

\fi



\end{document}