\documentclass[10pt]{article}
\usepackage{geometry}
\usepackage{amsfonts}

\geometry{
legalpaper, total={7.00in, 11.31in},left=20mm, right=20mm, top=10mm,
}

\begin{document}

\begin{center}
  \bfseries\large
  Sylhet Cadet College

\normalsize
Fortnightly Examination - 2024

  Class: XI

  Subject: Statistics First Paper

  Time: 40 minutes \qquad \qquad  \qquad \qquad Subject Code: 129  \qquad  \qquad \qquad  \qquad Full Marks: 20

%  \normalfont\normalsize
 % 11.45a.m.~--~1.45p.m.
\end{center}

\noindent
\begin{tabular}{p{\dimexpr\linewidth-2\tabcolsep}}
  \textbf{Answer all the questions. Figures in the right indicate full marks.}\\
  \hline
\end{tabular}

\begin{enumerate}

  \item Does Arithmetic Mean depend on change of origin and scale? Prove mathematically and with an example. \hfill 3

 \item $Y_i = 3X_i$. If $G_y = 9, G_x=?$ [G stands for Geometric Mean] \hfill 3
 
 \item Find 1st and 3rd Quartiles and interpret: $2,1,0,5,-6,7,-4$ \hfill 3
 
 \item For two distinct non-zero values, what is the relationship among AM, GM, and HM? \hfill 1
 
\item
	  \textbf{Arithmetic (AM) and Harmonic Mean (HM) of two numbers are 25 and 9, respectively.} 
    \begin{enumerate}
    \item
	When is HM useful? \hfill 1
    \item
	Derive HM formula using the concept of average velocity. \hfill 2
    \item  
	Find the two values from the stem. \hfill 3
    \item
	Show mathematically that $HM \le AM$ (for n=2) \hfill 4
  \end{enumerate}

\end{enumerate}

\vspace{0.5in}


\begin{center}
  \bfseries\large
  Sylhet Cadet College

\normalsize
Fortnightly Examination - 2024

  Class: XI

  Subject: Statistics First Paper

  Time: 40 minutes \qquad \qquad  \qquad \qquad Subject Code: 129  \qquad  \qquad \qquad  \qquad Full Marks: 20

%  \normalfont\normalsize
 % 11.45a.m.~--~1.45p.m.
\end{center}

\noindent
\begin{tabular}{p{\dimexpr\linewidth-2\tabcolsep}}
  \textbf{Answer all the questions. Figures in the right indicate full marks.}\\
  \hline
\end{tabular}

\begin{enumerate}

  \item Does Arithmetic Mean depend on change of origin and scale? Prove mathematically and with an example. \hfill 3

 \item $Y_i = 3X_i$. If $G_y = 9, G_x=?$ [G stands for Geometric Mean] \hfill 3
 
 \item Find 1st and 3rd Quartiles and interpret: $2,1,0,5,-6,7,-4$ \hfill 3
 
 \item For two distinct non-zero values, what is the relationship among AM, GM, and HM? \hfill 1
 
\item
	  \textbf{Arithmetic (AM) and Harmonic Mean (HM) of two numbers are 25 and 9, respectively.} 
    \begin{enumerate}
    \item
	When is HM useful? \hfill 1
    \item
	Derive HM formula using the concept of average velocity. \hfill 2
    \item  
	Find the two values from the stem. \hfill 3
    \item
	Show mathematically that $HM \le AM$ (for n=2) \hfill 4
  \end{enumerate}

\end{enumerate}

\vspace{0.5in}


\begin{center}
  \bfseries\large
  Sylhet Cadet College

\normalsize
Fortnightly Examination - 2024

  Class: XI

  Subject: Statistics First Paper

  Time: 40 minutes \qquad \qquad  \qquad \qquad Subject Code: 129  \qquad  \qquad \qquad  \qquad Full Marks: 20

%  \normalfont\normalsize
 % 11.45a.m.~--~1.45p.m.
\end{center}

\noindent
\begin{tabular}{p{\dimexpr\linewidth-2\tabcolsep}}
  \textbf{Answer all the questions. Figures in the right indicate full marks.}\\
  \hline
\end{tabular}

\begin{enumerate}

  \item Does Arithmetic Mean depend on change of origin and scale? Prove mathematically and with an example. \hfill 3

 \item $Y_i = 3X_i$. If $G_y = 9, G_x=?$ [G stands for Geometric Mean] \hfill 3
 
 \item Find 1st and 3rd Quartiles and interpret: $2,1,0,5,-6,7,-4$ \hfill 3
 
 \item For two distinct non-zero values, what is the relationship among AM, GM, and HM? \hfill 1
 
\item
	  \textbf{Arithmetic (AM) and Harmonic Mean (HM) of two numbers are 25 and 9, respectively.} 
    \begin{enumerate}
    \item
	When is HM useful? \hfill 1
    \item
	Derive HM formula using the concept of average velocity. \hfill 2
    \item  
	Find the two values from the stem. \hfill 3
    \item
	Show mathematically that $HM \le AM$ (for n=2) \hfill 4
  \end{enumerate}

\end{enumerate}

\end{document}