\documentclass{exam}
%\documentclass[11pt,a4paper]{exam}
\usepackage{amsmath,amsthm,amsfonts,amssymb,dsfont}
\usepackage{ifthen}
\usepackage[legalpaper, total={177.8mm, 290mm}]{geometry}
\usepackage{enumerate}% http://ctan.org/pkg/enumerate
\usepackage{multicol}
\usepackage{hhline}
\usepackage[table]{xcolor}


% Accumulate the answers. Unmodified from Phil Hirschorn's answer
% https://tex.stackexchange.com/questions/15350/showing-solutions-of-the-questions-separately/15353
\newbox\allanswers
\setbox\allanswers=\vbox{}

\newenvironment{answer}
{%
    \global\setbox\allanswers=\vbox\bgroup
    \unvbox\allanswers
}%
{%
    \bigbreak
    \egroup
}

\newcommand{\showallanswers}{\par\unvbox\allanswers}
% End Phil's answer


% Is there a better way?
\newcommand*{\getanswer}[5]{%
    \ifthenelse{\equal{#5}{a}}
    {\begin{answer}\thequestion. (a)~#1\end{answer}}
    {\ifthenelse{\equal{#5}{b}}
        {\begin{answer}\thequestion. (b)~#2\end{answer}}
        {\ifthenelse{\equal{#5}{c}}
            {\begin{answer}\thequestion. (c)~#3\end{answer}}
            {\ifthenelse{\equal{#5}{d}}
                {\begin{answer}\thequestion. (d)~#4\end{answer}}
                {\begin{answer}\textbf{\thequestion. (#5)~Invalid answer choice.}\end{answer}}}}}
}

\setlength\parindent{0pt}
%usage \choice{ }{ }{ }{ }
%(A)(B)(C)(D)
\newcommand{\fourch}[5]{
    \par
    \begin{tabular}{*{4}{@{}p{0.23\textwidth}}}
        (a)~#1 & (b)~#2 & (c)~#3 & (d)~#4
    \end{tabular}
    \getanswer{#1}{#2}{#3}{#4}{#5}
}

%(A)(B)
%(C)(D)
\newcommand{\twoch}[5]{
    \par
    \begin{tabular}{*{2}{@{}p{0.46\textwidth}}}
        (a)~#1 & (b)~#2
    \end{tabular}
    \par
    \begin{tabular}{*{2}{@{}p{0.46\textwidth}}}
        (c)~#3 & (d)~#4
    \end{tabular}
    \getanswer{#1}{#2}{#3}{#4}{#5}
}

%(A)
%(B)
%(C)
%(D)
\newcommand{\onech}[5]{
    \par
    (a)~#1 \par (b)~#2 \par (c)~#3 \par (d)~#4
    \getanswer{#1}{#2}{#3}{#4}{#5}
}

\newlength\widthcha
\newlength\widthchb
\newlength\widthchc
\newlength\widthchd
\newlength\widthch
\newlength\tabmaxwidth

\setlength\tabmaxwidth{0.96\textwidth}
\newlength\fourthtabwidth
\setlength\fourthtabwidth{0.25\textwidth}
\newlength\halftabwidth
\setlength\halftabwidth{0.5\textwidth}

\newcommand{\choice}[5]{%
\settowidth\widthcha{AM.#1}\setlength{\widthch}{\widthcha}%
\settowidth\widthchb{BM.#2}%
\ifdim\widthch<\widthchb\relax\setlength{\widthch}{\widthchb}\fi%
    \settowidth\widthchb{CM.#3}%
\ifdim\widthch<\widthchb\relax\setlength{\widthch}{\widthchb}\fi%
    \settowidth\widthchb{DM.#4}%
\ifdim\widthch<\widthchb\relax\setlength{\widthch}{\widthchb}\fi%

% These if statements were bypassing the \onech option.
% \ifdim\widthch<\fourthtabwidth
%     \fourch{#1}{#2}{#3}{#4}{#5}
% \else\ifdim\widthch<\halftabwidth
% \ifdim\widthch>\fourthtabwidth
%     \twoch{#1}{#2}{#3}{#4}{#5}
% \else
%      \onech{#1}{#2}{#3}{#4}{#5}
%  \fi\fi\fi}

% Allows for the \onech option.
\ifdim\widthch>\halftabwidth
    \onech{#1}{#2}{#3}{#4}{#5}
\else\ifdim\widthch<\halftabwidth
\ifdim\widthch>\fourthtabwidth
    \twoch{#1}{#2}{#3}{#4}{#5}
\else
    \fourch{#1}{#2}{#3}{#4}{#5}
\fi\fi\fi}


\begin{document}

\begin{table}[h]
\centering
\begin{tabular}{lllll}
\textbf{\large SYLHET CADET COLLEGE} &  &  &  &  \\ \cline{4-5} 
PROGRESS TEST EXAMINATION - 2023 &  & \multicolumn{1}{l|}{} & \multicolumn{1}{l|}{Set} & \multicolumn{1}{l|}{D} \\ \cline{4-5} 
CLASS: XII &  &  &  &  \\ \cline{3-5} 
MULTIPLE CHOICE QUESTIONS & \multicolumn{1}{l|}{\textbf{Subject Code:}} & \multicolumn{1}{l|}{1} & \multicolumn{1}{l|}{2} & \multicolumn{1}{l|}{9} \\ \cline{3-5} 
STATISTICS SECOND PAPER &  &  &  &  \\
TIME – 25 minutes &  &  &  &  \\
FULL MARKS – 25 &  &  &  & 
\end{tabular}
\end{table}
%  \normalfont\normalsize
 % 11.45a.m.~--~1.45p.m.
\hrule

\begin{center}
[N.B. – Answer all the questions. Each question carries ONE mark. Block fully, with a black ball- point pen, the circle of the letter that stands for the correct/best answer in the “Answer sheet” for the Multiple Choice Questions Examination.]\\

  
  \textbf{Candidates are asked not to leave any mark or spot on the question paper.}
\end{center}
\begin{questions}

\question \textbf{$P(A\cup B) = P(A) + P(B)$ implies A \& B are --}
\choice{Disjoint}{Independent}{Joint}{Independent}{a}

\question \textbf{The characteristics of binomial distribution--}

i. $E(X) > V(X)$ \\
ii. $E(X) = V(X)$ \\
iii. $E(X) = np$

\textbf{Which one is correct?}

\choice{i and ii}{i and iii}{ii and iii}{i, ii and iii}{b}

\question \textbf{How many additive laws of probability are there?}
\choice{1}{2}{3}{4}{b}

\question \textbf{If A and B are independent, which formula is correct?}
\choice{$P(A \cap B) = P(A) \cdot P(B)$}{$P(A \cap B) = P(\bar A) \cdot P(B)$}{$P(A \cap B) = P(A) \cdot P(\bar B)$}{$P(A \cap \bar B) = P(A) \cdot P(B)$}{a}

\question \textbf{There are 3 red, 4 black, and 5 white balls in an urn. If two balls are randomly taken, what is the probability that both are red?}
\choice{$\displaystyle  \frac{1}{66}$}{$\displaystyle  \frac{1}{22}$}{$\displaystyle  \frac{2}{22}$}{$\displaystyle  \frac{3}{11}$}{b}

\textbf{Answer the next two questions based on the following information}

\begin{center}
$\displaystyle  f(x) = kx; 0 < x < 5$
\end{center}

\question \textbf{What is the value of $P(2 <x<3)$}
\choice{$\displaystyle  \frac45$}{$\displaystyle  \frac35$}{$\displaystyle  \frac25$}{$\displaystyle  \frac15$}{d}

\question \textbf{$P(X>0)$}
\choice{0.99}{0.5}{1}{0}{c}

\question \textbf{Which is a discrete random variable?}
\choice{Age of students}{Amount of Production in a factory}{Height of workers}{Page size in word processing softwares}{d}

\question \textbf{What is $F(-\infty)$ for a distribution function $F(x)$?}
\choice{$-\infty$}{-1}{0}{1}{c}

\textbf{Answer the next two questions based on the following information}

\begin{center}
For two exhaustive evenst A \& B, P(A) = 0.7 and P(B) = 0.4
\end{center}

\question \textbf{$P(A\cap B) = ?$}
\choice{0.1}{0.3}{0.6}{1}{a}

\question \textbf{The events A \& B are --}

i. independent \\
ii. dependent \\
iii. not mutually exclusive

\textbf{Which one is correct?}

\choice{i and ii}{i and iii}{ii and iii}{i, ii and iii}{c}

\question \textbf{If $\displaystyle  P(x) = \frac 1{15}; x = 1,2,3, \cdots 15$, what is the value of the expectation?}
\choice{8.5}{7.5}{7}{8}{d}

\question \textbf{If $\displaystyle  P(x)= \frac{3-|4-x|}{k}; x=2,3,4, \cdots 6$, what is the value of k?}
\choice{6}{9}{10}{40}{b}

\question \textbf{If the variance of X is 3, what is the variance of V(3)?}
\choice{1}{2}{3}{0}{d}

\question \textbf{If $V(X) = 5,$, what is $V(X+5)?$}
\choice{0}{5}{10}{25}{b}

\question \textbf{E(X) + E(Y) = ?}
\choice{E(X) - E(Y)}{E(X) + E(Y)}{2E(X) - E(Y)}{$E(X) \times E(Y)$}{b}

\question \textbf{What is true of binomial distribution?}
\choice{There is one parameter}{Number of trial is fixed}{Mean is greater than variance}{Skewness is negative}{c}

\question \textbf{What is the skewness of binomial distribution?}
\choice{$\displaystyle \frac{(q-p)^2}{np}$}{$\displaystyle \frac{(q-p)^2}{np}$}{$\displaystyle \frac{(p+1)^2}{npq}$}{$\displaystyle \frac{(q+p)^2}{npq}$}{a}

\question \textbf{When is a binomial distribution positively skewed?}
\choice{p > q}{p = q}{p < q}{p+q < 1}{b}

\textbf{Answer the next two questions based on the following information}

\begin{center}
In a binomial distribution, $\displaystyle  P(x=4) = \frac12 P(x=5); n  = 10$
\end{center}

\question \textbf{What is the mean?}
\choice{6.25}{5.15}{8.52}{5.22}{a}

\question \textbf{$P(x=2) =$ ---}
\choice{0.0053}{0.0069}{0.0085}{0.94}{b}

\question \textbf{The no. of parameters in a Poisson distribution is ---}
\choice{1}{2}{3}{4}{a}

\question \textbf{On average, 1 in 1000 houses in a city gets a fire-burn in a year.If there are 2000 houses, what is the probability that, in a certain year, exactly 5 house will be burnt?}
\choice{0.036}{0.040}{0.027}{0.091}{a}

\textbf{Answer the next two questions based on the following information}

\begin{table}[h]
\centering
\begin{tabular}{c|c|c|c|c}
Year & 1 & 2 & 3 & 4 \\ \hline
Population & 100 & 110 & 120 & 130
\end{tabular}
\end{table}

\question \textbf{Which type of growth is seen here?}
\choice{Arithmetic growth}{Geometric growth}{Exponential growth}{None}{a}

\question \textbf{What is the rate of increase?}
\choice{1}{0.1}{10}{1\%}{b}


%\question \textbf{To complete the song, the last answer should be
%\choice{a}{b}{c}{d}{e} % Invalid answer choice

\end{questions}

 \vspace{2.5cm}

\begin{center}
Why speculate when you can calculate? - John C. Baez
\end{center}

\pagebreak
%\newpage  %Uncomment to put on new age
\bigskip

\begin{multicols}{3}
[
Answer Key
]
\showallanswers % Phil Hirschorn
\end{multicols}


\end{document}