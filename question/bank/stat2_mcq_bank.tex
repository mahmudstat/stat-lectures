\documentclass{exam}
%\documentclass[11pt,a4paper]{exam}
\usepackage{amsmath,amsthm,amsfonts,amssymb,dsfont}
\usepackage{ifthen}
\usepackage{enumerate}% http://ctan.org/pkg/enumerate
\usepackage{multicol}
\usepackage{graphicx}
\usepackage{float}
\usepackage{xcolor}
\usepackage{lipsum}
\usepackage{hyperref}

% Accumulate the answers. Unmodified from Phil Hirschorn's answer
% https://tex.stackexchange.com/questions/15350/showing-solutions-of-the-questions-separately/15353
\newbox\allanswers
\setbox\allanswers=\vbox{}

\newenvironment{answer}
{%
    \global\setbox\allanswers=\vbox\bgroup
    \unvbox\allanswers
}%
{%
    \bigbreak
    \egroup
}

\newcommand{\showallanswers}{\par\unvbox\allanswers}
% End Phil's answer


% Is there a better way?
\newcommand*{\getanswer}[5]{%
    \ifthenelse{\equal{#5}{a}}
    {\begin{answer}\thequestion. (a)~#1\end{answer}}
    {\ifthenelse{\equal{#5}{b}}
        {\begin{answer}\thequestion. (b)~#2\end{answer}}
        {\ifthenelse{\equal{#5}{c}}
            {\begin{answer}\thequestion. (c)~#3\end{answer}}
            {\ifthenelse{\equal{#5}{d}}
                {\begin{answer}\thequestion. (d)~#4\end{answer}}
                {\begin{answer}\textbf{\thequestion. (#5)~Invalid answer choice.}\end{answer}}}}}
}

\setlength\parindent{0pt}
%usage \choice{ }{ }{ }{ }
%(A)(B)(C)(D)
\newcommand{\fourch}[5]{
    \par
    \begin{tabular}{*{4}{@{}p{0.23\textwidth}}}
        (a)~#1 & (b)~#2 & (c)~#3 & (d)~#4
    \end{tabular}
    \getanswer{#1}{#2}{#3}{#4}{#5}
}

%(A)(B)
%(C)(D)
\newcommand{\twoch}[5]{
    \par
    \begin{tabular}{*{2}{@{}p{0.46\textwidth}}}
        (a)~#1 & (b)~#2
    \end{tabular}
    \par
    \begin{tabular}{*{2}{@{}p{0.46\textwidth}}}
        (c)~#3 & (d)~#4
    \end{tabular}
    \getanswer{#1}{#2}{#3}{#4}{#5}
}

%(A)
%(B)
%(C)
%(D)
\newcommand{\onech}[5]{
    \par
    (a)~#1 \par (b)~#2 \par (c)~#3 \par (d)~#4
    \getanswer{#1}{#2}{#3}{#4}{#5}
}

\newlength\widthcha
\newlength\widthchb
\newlength\widthchc
\newlength\widthchd
\newlength\widthch
\newlength\tabmaxwidth

\setlength\tabmaxwidth{0.96\textwidth}
\newlength\fourthtabwidth
\setlength\fourthtabwidth{0.25\textwidth}
\newlength\halftabwidth
\setlength\halftabwidth{0.5\textwidth}

\newcommand{\choice}[5]{%
\settowidth\widthcha{AM.#1}\setlength{\widthch}{\widthcha}%
\settowidth\widthchb{BM.#2}%
\ifdim\widthch<\widthchb\relax\setlength{\widthch}{\widthchb}\fi%
    \settowidth\widthchb{CM.#3}%
\ifdim\widthch<\widthchb\relax\setlength{\widthch}{\widthchb}\fi%
    \settowidth\widthchb{DM.#4}%
\ifdim\widthch<\widthchb\relax\setlength{\widthch}{\widthchb}\fi%

% These if statements were bypassing the \onech option.
% \ifdim\widthch<\fourthtabwidth
%     \fourch{#1}{#2}{#3}{#4}{#5}
% \else\ifdim\widthch<\halftabwidth
% \ifdim\widthch>\fourthtabwidth
%     \twoch{#1}{#2}{#3}{#4}{#5}
% \else
%      \onech{#1}{#2}{#3}{#4}{#5}
%  \fi\fi\fi}

% Allows for the \onech option.
\ifdim\widthch>\halftabwidth
    \onech{#1}{#2}{#3}{#4}{#5}
\else\ifdim\widthch<\halftabwidth
\ifdim\widthch>\fourthtabwidth
    \twoch{#1}{#2}{#3}{#4}{#5}
\else
    \fourch{#1}{#2}{#3}{#4}{#5}
\fi\fi\fi}



\begin{document}
\begin{titlepage}
    \begin{center}
        \vspace*{1cm}
            
        \Huge
        \textbf{Statistics MCQ Question Bank}
            
        \vspace{0.5cm}
        \LARGE
        Second Paper \\


            
        \vspace{1.5cm}
            

            
        \vfill
            
            
        \vspace{0.8cm}
            
                    \textbf{Abdullah Al Mahmud} \\
        \Large
        www.statmania.info\\
            
    \end{center}
\end{titlepage}

\tableofcontents

\newpage

 \section{Introduction to Probability}


\begin{questions}

\subsection {Permutation-Combination}

\question \textbf{Three objects can be placed in 2 positions in -- ways.}
\choice{3}{4}{6}{8}{c}

\question \textbf{In how many ways can a team of 2 be formed from 4 people?}
\choice{4}{6}{8}{12}{b}

\question \textbf{$\displaystyle ^np_r=$}
\choice{$\displaystyle \frac {n!}{(n-r)!}$}{$\displaystyle \frac {n!}{(n+r)!}$}{$\displaystyle \frac {n!}{r!}$}{$\displaystyle \frac {n!}{(r-n)!}$}{a}

\question \textbf{$\displaystyle ^nC_r=$}
\choice{$\displaystyle \frac {n!}{(n-1)!(n+r)!}$}{$\displaystyle \frac {r!}{n!(n-r)!}$}{$\displaystyle \frac {n!(n-1)!}{r!}$}{$\displaystyle \frac {n!}{(r-n)!}$}{a}

%-----------------------------------------------------------------
\subsection {Conceptual Questions}
%-----------------------------------------------------------------

\question \textbf{What is the probability that at least one item in a sample space will occurr?}
\choice{0}{0.5}{1}{Undefined}{c}

\question \textbf{The probability of two disjoint sets happening together is:}
\choice{0.5}{0}{1}{$0  \leq x < 1$}{b}

\question \textbf{How many additive laws of probability are there?}
\choice{1}{2}{3}{4}{b}

\question \textbf{$P(A\cup B) = P(A) + P(B)$ implies A \& B are --}
\choice{Disjoint}{Independent}{Joint}{Independent}{a}

\question \textbf{$P(A\cap B) = P(A) \times P(B)$ implies A \& B are --}
\choice{Disjoint}{Independent}{Joint}{Independent}{b}

\question \textbf{Which is the formula of classical approach of probability?}
\choice{$P=\frac{\text{No. of favorable outcomes}}{\text{Total no. of possible outcomes}}$}{$P=\frac{\text{No. of total outcomes}}{\text{No. of favorable outcomes}}$}{$P = \lim_{n(S)\to\infty} \frac{n(A)}{n(S)}$}{$P = \lim_{n(A)\to\infty} \frac{n(A)}{n(S)}$}{a}

\question \textbf{Which is the formula of empirical/relative frequency approach of probability?}
\choice{$P=\frac{\text{No. of favorable outcomes}}{\text{Total no. of possible outcomes}}$}{$P=\frac{\text{No. of total outcomes}}{\text{No. of favorable outcomes}}$}{$\displaystyle P = \lim_{n(S)\to\infty} \frac{n(A)}{n(S)}$}{$\displaystyle P = \lim_{n(A)\to\infty} \frac{n(A)}{n(S)}$}{a}

\question \textbf{What is the correct formula for conditional probability?}
\choice{$P(A|B) = \frac{P(A \cap B)}{P(B|A)}$}{$P(A|B) = \frac{P(A \cap B)}{P(A)}$}{$P(A|B) = \frac{P(A \cap B)}{P(B)}$}{$P(A|B) = \frac{P(B|A)}{P(B|A)}$}{a}

\question \textbf{The third axiom of probability is --}
\choice{$0 \le P(A) \le 1$}{$P(S) = 1$}{$\displaystyle P(A_1 U A_2 U \cdots U
A_n) = \sum_{i=1}^{\infty}P(A_i)$}{$P(A) = 1 - P(A)$}{c}

\question \textbf{Possible value of probability}

i. -1 \quad
ii. 0.5 \quad
iii. 0

\textbf{Which one is correct?}

\choice{i and ii}{i and iii}{ii and iii}{i, ii and iii}{c}

\question \textbf{An act repeated under some specific conditions is called --}
\choice{Event}{Experiment}{Sample}{Sample space}{b}

\question \textbf{$P(0)$ implies -- }
\choice{A certain event}{An uncertain event}{An impossible event}{A probable event}{c}

\question \textbf{Events having some common elements are called --}
\choice{Complementary events}{Mutually exclusive events}{Exhaustive events}{Non-Mutually exclusive events events}{a}

\question \textbf{The minimum value of probability is}
\choice{$-\alpha$}{1}{0}{-1}{c}

\question \textbf{Each element of sample space is called--}
\choice{Trial}{Experiment}{Variable}{Sample Point}{d}

\question \textbf{Two events not ocurring together are called--}
\choice{dependent Events}{Independent Events}{Mutually Exclusive Events}{Marginal Events}{c}

\question \textbf{If A and B are independent, which formula is correct?}
\choice{$P(A \cap B) = P(A) \cdot P(B)$}{$P(A \cap B) = P(\bar A) \cdot P(B)$}{$P(A \cap B) = P(A) \cdot P(\bar B)$}{$P(A \cap \bar B) = P(A) \cdot P(B)$}{a}

\question \textbf{Which of the following are disjoint events?}
\choice{$A = \{1, 2, 3\}, B = \{4, 5\}$}{$A = \{a, b\}, B = \{b, c\}$}{$A = \{0\}, B = \{0, 1\}$}{$A = \{x, y\}, B = \{x, y\}$}{a}

\question \textbf{Which of the following are disjoint events?}
\choice{$P = \{1, 2\}, Q = \{2, 3\}$}{$P = \{x\}, Q = \{x, y\}$}{$P = \{1, 3\}, Q = \{3, 5\}$}{$P = \{m, n\}, Q = \{p, q\}$}{d}

\question \textbf{Let the sample space be $S = \{1, 2, 3, \dots, 10\}$. Which of the following pairs of events are disjoint?}

i. $A$: Number is prime, \quad $B$: Number is greater than 3 \\
ii. $A$: Number is even, \quad $B$: Number is divisible by 3 \\
iii. $A$: Number is less than 5, \quad $B$: Number is greater than 6

\textbf{Which one is correct?}

\choice{i and ii}{i and iii}{ii and iii}{i, ii and iii}{c}

\question \textbf{Let $S = \{1, 2, \dots, 10\}$. Which of the following event pairs are disjoint?}

i. $A$: Number is divisible by 2, \quad $B$: Number is divisible by 5 \\
ii. $A$: Number is less than 7, \quad $B$: Number is odd \\
iii. $A$: Number is a prime, \quad $B$: Number is a multiple of 4

\textbf{Which one is correct?}

\choice{i and iii}{i and ii}{ii and iii}{i, ii and iii}{a}

\question \textbf{Let the sample space be $S = \{1, 2, 3, \dots, 10\}$. Which of the following pairs of events are disjoint?}

i. $A$: Number is a multiple of 4, \quad $B$: Number is odd \\
ii. $A$: Number is less than 4, \quad $B$: Number is greater than 8 \\
iii. $A$: Number is a square, \quad $B$: Number is even

\textbf{Which one is correct?}

\choice{i and ii}{i and iii}{ii and iii}{i, ii and iii}{a}

\question \textbf{Let $S = \{1, 2, 3, \dots, 10\}$. Which of the following pairs of events are disjoint?}

\choice{$A$: Multiples of 3, $B$: Multiples of 5}{%
$A$: Prime numbers, $B$: Even numbers greater than 2}{%
$A$: Numbers less than 4, $B$: Numbers greater than 6}{%
All of the above}{d}


\subsection{Numbers}

\question \textbf{A number is randomly chosen from a list of 10 consecutive positive
integers. What is the probability that the number selected is greater than the
average (arithmetic mean) of all 10 integers?}
\choice{$\frac 13$}{$\frac 34$}{$\frac 4{10}$}{$\frac 12$}{d}


\question \textbf{10 out of each 100 people in a city walk to the office. If one is picked randomly, what is the probability s/he does not walk to the office?}
\choice{0.95}{0.10}{0.90}{0.01}{c}

\question \textbf{In a school, 15 out of 100 students prefer online classes over in-person classes. If a student is selected randomly, what is the probability that they prefer in-person classes?}  
\choice{0.15}{0.85}{0.75}{0.25}{b}  

\question \textbf{A factory reports that 8 out of every 100 manufactured items are defective. If an item is chosen at random, what is the probability that it is not defective?}  
\choice{0.08}{0.92}{0.80}{0.12}{b}  

\question \textbf{A hospital study finds that 12\% of patients do not prefer evening appointments. If a patient is selected at random, what is the probability that they prefer evening appointments?}  
\choice{0.12}{0.78}{0.88}{0.18}{c}  

\question \textbf{A survey shows that 5 out of every 200 customers in a store pay with cash. If a customer is picked randomly, what is the probability that they pay using another method?}  
\choice{0.050}{0.500}{0.975}{0.025}{c}  

%-----------------------------------------------------------------
\subsection {Coin-Die}
%-----------------------------------------------------------------

\question \textbf{Tossing a die r times generates how many outcomes?}
\choice{$6\times r$}{$r^6$}{$6^r$}{$2^r$}{c}

\question \textbf{Tossing a coin r times generates how many outcomes?}
\choice{$2\times r$}{$r^2$}{$2^r$}{$6^r$}{c}

\question \textbf{A coin is thrown thrice. How many outcomes are generated?}
\choice{3}{4}{8}{9}{c}

\question \textbf{A coin is thrown twice. What is the probability of getting 2 heads?}
\choice{$\frac14$}{$\frac13$}{$\frac12$}{$\frac24$}{a}

\question \textbf{A fair coin is tossed twice. What is the probability of getting at least one tail?}  
\choice{$\frac14$}{$\frac12$}{$\frac34$}{$\frac13$}{c}  

\question \textbf{Two fair coins are tossed simultaneously. What is the probability of getting exactly one head?}  
\choice{$\frac14$}{$\frac12$}{$\frac34$}{$\frac13$}{b}  

\question \textbf{A coin is flipped twice. What is the probability of getting heads first and tails second?}  
\choice{$\frac14$}{$\frac13$}{$\frac12$}{$\frac24$}{a}  

\question \textbf{If two fair coins are tossed together, what is the probability of getting at least one head?}  
\choice{$\frac12$}{$\frac13$}{$\frac34$}{$\frac14$}{c}  

\question \textbf{A fair coin is tossed twice. What is the probability of getting two tails?}  
\choice{$\frac12$}{$\frac13$}{$\frac14$}{$\frac24$}{c}  

\question \textbf{Two fair coins are tossed. What is the probability that at least one of them lands on tails?}  
\choice{$\frac34$}{$\frac12$}{$\frac14$}{$\frac13$}{a}  


\question \textbf{A die is thrown twice. This is called --}
\choice{An experiment}{sample space}{A random experiment}{A trial}{a}

\question \textbf{If a neutral die is thrown, the probability of having a digit greater than 6 is}
\choice{$\frac 1 6$}{$\frac 0 6$}{$\frac 2 3$}{$\frac 3 6$}{b}

\question \textbf{Tossing a coin twice generates how many outcomes?}
\choice{4}{16}{8}{2}{a}

\question \textbf{A die is rolled twice. How many possible outcomes are there?}
\choice{6}{12}{36}{18}{c}


%-----------------------------------------------------------------
\subsection {Balls-Cards}
%-----------------------------------------------------------------

\question \textbf{There are 3 red, 4 black, and 5 white balls in an urn. If two balls are randomly taken, what is the probability that both are red?}
\choice{$\frac{1}{66}$}{$\frac{1}{22}$}{$\frac{2}{22}$}{$\frac{3}{11}$}{b}

\question \textbf{There are 3 red, 4 black, and 5 white balls in an urn. If two balls are randomly taken, what is the probability that neither is red?}  
\choice{$\frac{5}{11}$}{$\frac{6}{11}$}{$\frac{3}{11}$}{$\frac{5}{22}$}{b}  

\question \textbf{A jar contains 6 blue and 4 green marbles. If two marbles are drawn at random, what is the probability that both are blue?}  
\choice{$\frac{5}{18}$}{$\frac{1}{3}$}{$\frac{1}{2}$}{$\frac{1}{4}$}{b}  

\question \textbf{A box has 7 black and 5 white balls. If one ball is drawn at random, what is the probability that it is not black?}  
\choice{$\frac{7}{12}$}{$\frac{5}{12}$}{$\frac{1}{2}$}{$\frac{1}{3}$}{b}  

\question \textbf{A bag contains 8 red and 6 white balls. If two balls are drawn at random, what is the probability that they are of different colors?}  
\choice{$\frac{24}{91}$}{$\frac{58}{91}$}{$\frac{48}{91}$}{$\frac{72}{91}$}{c}  

\question \textbf{A box contains 9 blue and 3 red balls. If two balls are randomly picked, what is the probability that at least one is red?}  
\choice{$\frac{3}{11}$}{$\frac{1}{3}$}{$\frac{18}{33}$}{$\frac{5}{11}$}{d}  

\textbf{Answer the next questions based on the following information.}

A card is drawn from of pack of playing cards.

\question \textbf{What is the probability that the card is a King?}
\choice{0.0192}{0.25}{0.5}{0.0769}{d}

\question \textbf{P(The card is not from Diamonds)--}
\choice{$\frac12$}{$0$}{$\frac34$}{$\frac14$}{c}

\question \textbf{P(The card is red or Clubs)}
\choice{$\frac14$}{$\frac12$}{$\frac23$}{$\frac34$}{d}

\textbf{Answer the next TWO questions based on the following information.}

An urn contains 5 red, 7 blue, and 8 green balls.

\question \textbf{What is the probability that the ball drawn is red?}
\choice{0.26}{0.25}{0.2}{0.4}{a}

\question \textbf{P(The ball drawn is not blue)--}
\choice{$\frac{13}{20}$}{$0.5$}{$\frac{7}{20}$}{$\frac{8}{20}$}{a}


%-----------------------------------------------------------------
\subsection {Set-Problems}
%-----------------------------------------------------------------

\question \textbf{For two independent events $A$ and $B$, which one is correct?}
\choice{$P(A\cap B) = P(A) \times P(B)$}{$P(A\cup B) = P(A) + P(B)$}{$P(A\cap B) = P(A) - P(B)$}{$P(A\cup B) = P(A) \times P(B)$}{a}

\question \textbf{For two mutually exclusive events $A$ and $B$, which one is correct?}
\choice{$P(A\cap B) = P(A) \times P(B)$}{$P(A\cup B) = P(A) + P(B)$}{$P(A\cap B) = P(A) - P(B)$}{$P(A\cup B) = P(A) \times P(B)$}{b}

\question \textbf{Which of the following correct?}
\choice{$\displaystyle \frac{P(A)}{P(B)}= \frac{P(B|A)}{P(A|B)}$}{$\displaystyle \frac{P(A)}{P(A|B)}= \frac{P(B|A)}{P(B)}$}{$\displaystyle \frac{P(A)}{P(B)}= \frac{P(B|A)}{P(B)}$}{$\displaystyle \frac{P(A)}{P(B)}= \frac{P(A|B)}{P(B|A)}$}{d}

\question \textbf{The probability of rain is $\frac 16$ for any given day next week. What is the probability that it will rain on both Monday and Tuesday?}
\choice{$\frac 16$}{$\frac 1{36}$}{$\frac 56$}{$\frac 1{17}$}{b}

\question \textbf{Given \( P(A \cup B) = 0.7 \), \( P(A \cap B) = 0.2 \), what are \( P(A) \) and \( P(B) \)?}  
\choice{$P(A) = 0.5 \text{ and } P(B) = 0.4
$}{$P(A) = 0.4 \text{ and } P(B) = 0.6$}{$P(A) = 0.4 \text{ and } P(B) = 0.3$}{$P(A) = 0.7 \text{ and } P(B) = 0.3$}{a}

\question \textbf{If \( P(A) = 0.4 \), \( P(B) = 0.5 \), and \( P(A \cup B) = 0.7 \), what is \( P(A \cap B) \)?}  
\choice{$0.2$}{$0.1$}{$0.3$}{$0.4$}{a}

\question \textbf{Given \( P(A) = 0.3 \), \( P(A \cup B) = 0.6 \), and \( P(A \cap B) = 0.1 \), what is \( P(B) \)?}  
\choice{$0.6$}{$0.4$}{$0.3$}{$0.2$}{b}

\question \textbf{If \( P(A) = 0.5 \), \( P(B) = 0.6 \), and \( P(A \cap B) = 0.3 \), what is \( P(A \cup B) \)?}  
\choice{$0.8$}{$0.9$}{$0.7$}{$1$}{c}

\question \textbf{If \( P(A) = 0.2 \), \( P(B) = 0.3 \), and \( P(A \cup B) = 0.4 \), what is \( P(A \cap B) \)?}  
\choice{$0.9$}{$0.2$}{$0.3$}{$0.1$}{d}

\question \textbf{Given \( P(A) = 0.7 \), \( P(A \cup B) = 0.9 \), and \( P(A \cap B) = 0.5 \), what is \( P(B) \)?}  
\choice{$0.8$}{$0.6$}{$0.7$}{$0.5$}{c}


\textbf{Answer the next two questions based on the following information}

\begin{center}
For two exhaustive evenst A \& B, P(A) = 0.7 and P(B) = 0.4
\end{center}

\question \textbf{$P(A\cap B) = ?$}
\choice{0.1}{0.3}{0.6}{1}{a}

\question \textbf{The events A \& B are --}

i. independent \\
ii. dependent \\
iii. not mutually exclusive

\textbf{Which one is correct?}

\choice{i and ii}{i and iii}{ii and iii}{i, ii and iii}{c}

\textbf{Answer the next three questions using the following information}

$P(A) = \frac 1 3, P(B) = \frac 1 2 \space \& \space P(A\cup B) = \frac 7 {12}$

\question \textbf{$P(A \cap B) = ?$}
\choice{$\frac {5}{12}$}{$\frac12$}{$\frac{1}{4}$}{$\frac{15}{16}$}{c}

\question \textbf{$P(A \cap \bar B)=?$}
\choice{$\frac{1}{4}$}{$\frac{3}{4}$}{$\frac{5}{6}$}{$\frac{1}{12}$}{a}

\question \textbf{What is the probability that B occurs or A does not occur?}
\choice{$\frac{3}{4}$}{$\frac{7}{12}$}{$\frac{5}{12}$}{$\frac{11}{12}$}{d}


\textbf{Answer the next three questions using the following information}

$P(C) = \frac 2 5, P(D) = \frac 3 4 \space \& \space P(C \cup D) = \frac{9}{10}$

\question \textbf{$P(C \cap D) = ?$}
\choice{$\frac{1}{10}$}{$\frac{1}{4}$}{$\frac{7}{20}$}{$\frac{4}{5}$}{b}

\question \textbf{$P(C \cap \bar D)=?$}
\choice{$\frac{1}{10}$}{$\frac{2}{5}$}{$\frac{2}{20}$}{$\frac{3}{10}$}{c}

\question \textbf{What is the probability that D occurs or C does not occur?}
\choice{$\frac{17}{20}$}{$\frac{7}{10}$}{$\frac{3}{4}$}{$\frac{11}{20}$}{a}

% Situation Set Starts
\textbf{Answer the next three questions using the following information:}

$P(E) = \frac 1 3, P(F) = \frac 1 4 \space \& \space P(E \cap F) = \frac{1}{10}$

\question \textbf{$P(E \cup F) = ?$}
\choice{$\frac{1}{58}$}{$\frac{3}{10}$}{$\frac{58}{60}$}{$\frac{58}{120}$}{c}

\question \textbf{$P(E \cap \bar F)=?$}
\choice{$\frac{7}{40}$}{$\frac{7}{30}$}{$\frac{3}{10}$}{$\frac{1}{30}$}{b}

\question \textbf{What is the probability that F occurs or E does not occur?}
\choice{$\frac{11}{30}$}{$\frac{19}{30}$}{$\frac{13}{40}$}{$\frac{23}{30}$}{d}
% Situation Set Ends

\question \textbf{An un contains 10 red and 5 black balls. Two balls are drawn; what is the probability of getting two red balls?}
\choice{$\frac 37$}{$\frac 47$}{$\frac {20}{21}$}{$\frac 2{21}$}{a}

\subsection{Situation Set}

% Situation Set Starts
\textbf{Answer the next two questions based on the following information}

\begin{center}
For two comprehensive events $A$ and $B, P(A) = 0.8, \text{ and } P(B) = 0.6$; 
\end{center}
\question \textbf{What is the value of $P(A \cap B)?$}
\choice{0.1}{0.2}{0.3}{0.4}{d}

\question \textbf{The events $A$ and $B$ are --}

i. independent \\
ii. dependent \\
iii. non-disjoint

\textbf{Which one is correct?}

\choice{i and ii}{i and iii}{ii and iii}{i, ii and iii}{c}
% Situation Set Ends

\subsection{Multiple Completion}

% Multiple Completion Starts
\question \textbf{$P(A) = 0$ implies}

i. $A$ is an impossible event \\
ii. $A$ would ocurr in extreme cases \\
iii. $P(\bar A)$ is a certain event

\textbf{Which one is correct?}

\choice{i and ii}{i and iii}{ii and iii}{i, ii and iii}{b}
% Multiple Completion Ends

% Multiple Completion Starts
\question \textbf{If $A$ is an uncertain event, which one is possible?}

i. $0 < P(A) < 1$ \\
ii. $P(A) = 0.1$ \\
iii. $P(A) = 0$

\textbf{Which one is correct?}

\choice{i and ii}{i and iii}{ii and iii}{i, ii and iii}{a}
% Multiple Completion Ends

% Multiple Completion Starts
\question \textbf{If a die is thrown once, the probability of getting even numbers is --}

i. A certain event \\
ii. A composite event  \\
iii. An uncertain event

\textbf{Which one is correct?}

\choice{i and ii}{i and iii}{ii and iii}{i, ii and iii}{b}
% Multiple Completion Ends

%---------------------------------------------------------------------------
\section{Random Variables}
%---------------------------------------------------------------------------

\subsection{Concept of Random Variable}

\question \textbf{Which is a discrete random variable?}
\choice{Age of students}{Amount of Production in a factory}{Height of workers}{Page size in word processing softwares}{d}

\question \textbf{A set of sample points tabulated along with their respective probabilities is an example of -- }
\choice{Probability distribution}{Probability function}{Frequency distribution}{Marginal probability distribution}{a}

\question \textbf{How many conditions does a probability density function have?}
\choice{2}{3}{4}{5}{b}

\question \textbf{A coin is tossed twice and no. of heads appeared is denoted by X. How many possible values of X are there?}
\choice{1}{2}{0}{3}{d}

\question \textbf{A die is thrown thrice and the number of times a 6 appears is denoted by $X$. How many possible values can $X$ take?}
\choice{1}{2}{3}{4}{d}


\question \textbf{Which one is a property of marginal probability density function?}
\choice{$\displaystyle \int_{x} f(x^2) \,dx=1$}{$\displaystyle  \int_{x} f(x^2) \,dx=0.5$}{$\displaystyle  \int_{x} f(x) \,dx=1$}{$P(x \ge 1)$}{c}

\question \textbf{Which one is NOT an example of a continuous random variable -- }
\choice{Weight}{Height}{Time}{Size of television}{d}

\question \textbf{Integrated value of $\frac 14 x^4$ --}
\choice{$\frac 1{20} x^5$}{$\frac 1{20} x^5+c$}{$\frac 1{5} x^4$}{$\frac 5{4} x^5$}{b}

\question \textbf{The conditions of a probability distribution are--}

i. $\sum P(X) = 1$

ii. $\sum P(X) = 0$

iii. $0 \le P(X) \le 1$

\textbf{Which one is correct?}

\choice{i and ii}{i and iii}{ii and iii}{i, ii and iii}{b}

\question \textbf{The conditions for a cumulative distribution function (CDF) are--}

i. $F(x)$ is non-decreasing.

ii. $0 \le F(x) \le 1$

iii. $\displaystyle \lim_{x \to \infty} F(x) = 1$


\textbf{Which one is correct?}

\choice{i and ii}{ii and iii}{i and iii}{i, ii, and iii}{d}

\question \textbf{The properties of a discrete probability distribution table are--}

i. $\sum P(X) = 1$

ii. $P(X) \ge 0$ for all $X$

iii. Each probability corresponds to a discrete value.

\textbf{Which one is correct?}

\choice{i and ii}{ii and iii}{i and iii}{i, ii, and iii}{d}


\question \textbf{What is $F(\infty)$ for a distribution function $F(x)$?}
\choice{$-\infty$}{-1}{0}{1}{d}

\question \textbf{What is $F(-\infty)$ for a distribution function $F(x)$?}
\choice{$-\infty$}{-1}{0}{1}{c}

\question \textbf{How many types of random variables are there?}
\choice{2}{3}{4}{5}{a}

\question \textbf{Which of the following is not a discrete random variable?}
\choice{umber of students}{Weight}{Number of heads in coin toss}{Population}{b}

\question \textbf{Which one is a property of a probability distribution?}
\choice{$P(x_i) = 0$}{$P(x_i \ne 1)$}{$\Sigma P(x_i) = 1$}{$\int_x P(X) dx \le 1$}{c}

\question \textbf{Which one is not a discrete random variable?}
\choice{Summation two die throw outcome}{Weight}{Number of heads in five coin tosses}{Released version number of a software}{d}

\question \textbf{Which one is not a discrete random variable?}
\choice{Number of students in a class}{Weight of a package}{Shoe size}{Total goals scored in a match}{b}

\question \textbf{Which variable type can skip certain whole numbers?}
\choice{Number of chapters read in a day}{Weight of a person}{Number of floors in a building}{Number of people boarding a train}{c}

\question \textbf{Which one is an example of a discrete random variable?}
\choice{The amount of liquid in a glass}{Temperature readings at noon}{Number of defective items in a batch}{Exact age in years}{c}

\question \textbf{Identify which one is not a discrete variable.}
\choice{Number of cookies eaten}{Height of students}{Total cars in a parking lot}{Number of siblings}{b}


\question \textbf{Which one is a property of joint probability distribution?}
\choice{$P(X_i,Y_j)<1$}{$P(X_i,Y_j)=0$}{$P(X_i,Y_j)<0$}{$0 \leq P(X_i,Y_j)\leq 1$}{d}

%-----------------------------------------------------------------------------
\subsection{Situation Set} % Random variable
%-----------------------------------------------------------------------------

\textbf{Answer the next two questions based on the following information}

\begin{table}[h]
\centering
\begin{tabular}{cccc}
X & 0 & 1 & 2 \\ \hline
P(x) & $\frac 12$ & $\frac14$ & $\frac14$
\end{tabular}
\end{table}

\question \textbf{What is F(1)}
\choice{$0.65$}{$0.75$}{$0.5$}{$1$}{b}

\question \textbf{$P(X \le 1 \le 3) =$--}
\choice{0.75}{0.70}{0.95}{1}{a}

% Group Starts
\textbf{Answer the next three questions based on the following information}

\begin{table}[H]
\centering
\begin{tabular}{c|c|c|c|c}
X & 0 & 1 & 2 & 3 \\ \hline
P(X) & $\frac 14$ & m & $\frac 13$ & $\frac 16$
\end{tabular}
\end{table}

\question \textbf{What is the value of m?}
\choice{$\frac 1 3$}{$\frac 5 {12}$}{$\frac 1 4$}{$\frac 1 6$}{c}

\question \textbf{Find $F(2)$.}
\choice{$\frac 1 2$}{$\frac 3 4$}{$\frac 5 6$}{$\frac 2 3$}{c}

\question \textbf{What is $P(X > 1)$?}
\choice{$\frac 1 2$}{$\frac 5 {12}$}{$\frac 1 3$}{$\frac 7 {12}$}{a}
% Group Ends

% Group Starts
\textbf{Answer the next three questions based on the following information}

\begin{table}[H]
\centering
\begin{tabular}{c|c|c|c|c|c}
X & 1 & 2 & 3 & 4 & 5 \\ \hline
P(X) & $\frac 1 5$ & c & $\frac 1 4$ & $\frac 1 6$ & $\frac 1 3$
\end{tabular}
\end{table}

\question \textbf{What is the value of c?}
\choice{$\frac 1 3$}{$\frac 1 4$}{$\frac 1 {20}$}{$\frac 1 6$}{c}

\question \textbf{Find $P(2 < X \leq 4)$.}
\choice{$\frac 5 {12}$}{$\frac 1 2$}{$\frac 5 6$}{$\frac 2 3$}{a}

\question \textbf{What is $P(X \leq 3)$?}
\choice{$\frac 9 {20}$}{$\frac 7 {10}$}{$\frac 1 2$}{$\frac 3 4$}{c}
% Group Ends

\textbf{Answer the next three questions based on the following information}

\begin{table}[H]
\centering
\begin{tabular}{c|c|c|c}
x & 1 & 2 & 3 \\ \hline
P(x) & $\frac 13$ & a & $\frac16$
\end{tabular}
\end{table}

\question \textbf{What is the value of a?}
\choice{$\frac 23$}{$\frac 56$}{$\frac 12$}{$1$}{c}

\question \textbf{Find $P(2 < X \leq 3)$}
\choice{$\frac 56$}{$\frac 23$}{$\frac 12$}{$\frac 16$}{d}

\question \textbf{What is P(X<3)?}
\choice{$\frac 56$}{$\frac 25$}{$\frac 19$}{$\frac 17$}{a}

\textbf{Answer the next two questions based on the following information}

\begin{table}[H]
\centering
\begin{tabular}{c|c|c|c}
x & 1 & 2 & 3 \\ \hline
P(x) & $\frac 13$ & $\frac12$ & $\frac16$
\end{tabular}
\end{table}

\question \textbf{What is $F(2)$?}
\choice{$\frac 23$}{$\frac 56$}{$\frac 12$}{$1$}{b}

\question \textbf{$P(1 < X \leq 2)$}
\choice{$\frac 56$}{$\frac 23$}{$\frac 12$}{$\frac 16$}{c}

\textbf{Answer the next two questions based on the following information}

\begin{center}
$f(x) = kx; 0 < x < 5$
\end{center}

\question \textbf{What is the value of $P(2 <x<3)$}
\choice{$\frac45$}{$\frac35$}{$\frac25$}{$\frac15$}{d}

\question \textbf{$P(X>0)$}
\choice{0.99}{0.5}{1}{0}{c}

\textbf{Answer the next two questions using the following information}

\begin{table}[h]
	    \centering
\begin{tabular}{ccccccl}
x    & 1 & 2  & 3  & 4  & 5  & 6  \\ \hline
P(x) & k & 2k & 3k & 4k & 5k & 6k
\end{tabular}
\end{table}

\question \textbf{What is the value of k?}
\choice{$\frac{7}{21}$}{$\frac{5}{21}$}{$\frac{1}{21}$}{$1$}{c}

\question \textbf{What is the type of variable X?}
\choice{Discrete}{Discrete random}{Continuous}{Continuous random}{b}


\textbf{Answer the next THREE questions using the following information}

\begin{center}
$\displaystyle P(x) = \frac{x+1}{k}; x = 1,2,3,4$
\end{center}

\question \textbf{What is the value of k?}
\choice{10}{11}{14}{15}{c}

\question \textbf{$F(2)=-$}
\choice{$\frac{2}{14}$}{$\frac{3}{11}$}{$\frac{5}{14}$}{$\frac{5}{11}$}{c}

\question \textbf{$P(x)$ is a --}
\choice{Joint probability distribution}{Cumulative probability distribution}{Probability mass function}{Probability Density function}{c}

\question \textbf{The example of a discrete random variable is--}

i. Binomial variate 

ii. Poisson variate

iii. Normal variate 

\textbf{Which one is correct?}

\choice{i and ii}{i and iii}{ii and iii}{i, ii and iii}{a} 

\question \textbf{$f(x) = 2x; 0 <X<3$; What is F(3)?}
\choice{3}{0}{1}{0}{c}

 \question \textbf{$f(x) = 3x; 0 < X < 2$; What is $F(2)$?}  
\choice{6}{3}{1}{0}{c}

\question \textbf{$f(x) = x^2; 0 < X < 4$; What is $F(4)$?}  
\choice{16}{0}{4}{1}{d}

\question \textbf{$f(x) = 4 - x; 1 < X < 5$; What is $F(5)$?}  
\choice{3}{0}{1}{4}{c}


\textbf{Answer the next two questions based on the following information:}

$P(x,y) = \frac 1{21}(x+y); x = 1,2,3$ and $y=1,2$

\question \textbf{P(x)=?}
\choice{$P(x) = \frac{2x+3}{21}$}{$P(x) = \frac{x+3}{27}$}{$P(x) = \frac{4x+3}{21}$}{$P(x) = \frac{2x+5}{21}$}{a}

\question \textbf{P(y)=?}
\choice{$\frac{y+2}{7}$}{$\frac{y+3}{7}$}{$\frac{3y+2}{7}$}{$\frac{y+2}{9}$}{c}

\question \textbf{If $f(x) = kx^3; -1 \leq x \leq 1$, then k is}

i) positive \\
ii) negative  \\
iii) lies from -1 to 1

\choice{i}{ii}{iii}{i and ii}{a}

\textbf{Answer the next two questions based on the following information.}

\begin{table}[h]
\begin{center}
\begin{tabular}{|l|l|l|l|l|l|l|}
\hline
x    & 4         & 5         & 6         & 3         & 2         & 1         \\ \hline
P(X) & $\frac16$ & $\frac16$ & $\frac16$ & $\frac16$ & $\frac16$ & $\frac16$ \\ \hline
\end{tabular}
\end{center}
\end{table}

\question \textbf{The value of $P(3<X<5)$ is:}
\choice{$\frac12$}{$\frac16$}{$\frac13$}{0}{b}

\question \textbf{$P(x \neq 2) is:$}
\choice{$\frac56$}{$0$}{1}{Can't be found from this information}{a}

\subsection{Multiple Completion}

% Multiple Completion Starts
\question \textbf{For a continuous random variable \( X \) with PDF \( f(x) = 2x \), defined on \( [0, 1] \):}

i. \( f(x) \geq 0 \) for all \( x \in [0,1] \) \\
ii. \( \int_0^1 f(x) \, dx = 1 \) \\
iii. \( P(X > 1) = 0 \)

\textbf{Which one is correct?}

\choice{i and ii}{i and iii}{ii and iii}{i, ii and iii}{d}
% Multiple Completion Ends

% Multiple Completion Starts
\question \textbf{For a continuous random variable \( X \) with PDF \( f(x) = k(2 - x) \) defined on \( 0 \leq x \leq 2 \):}

i. The value of \( k \) is 1. \\
ii. The cumulative distribution function \( F(x) = x - \frac{x^2}{4} \) for \( 0 \leq x \leq 2 \). \\
iii. \( P(1 < X < 2) = \frac{3}{8} \)

\textbf{Which one is correct?}

\choice{i}{i and ii}{ii}{i, ii and iii}{c}
% Multiple Completion Ends


\newpage

%--------------------------------------------------------------------------
\section{Mathematical Expectation}
%--------------------------------------------------------------------------

\question \textbf{E(X+Y) = ?}
\choice{E(X) - E(Y)}{E(X) + E(Y)}{2E(X) - E(Y)}{$E(X) \times E(Y)$}{b}

\question \textbf{E(4x+2Y) = ?}
\choice{E(X) - E(Y)}{4E(X) + 2E(Y)}{2E(X) + 4E(Y)}{$E(X) \times E(Y)$}{b}


\question \textbf{What is the expected value of of the squared deviation of 
the value of the random variable from their mean?}
\choice{Arithmetic Mean}{Expectation}{Variance}{Co-variance}{c}

\question \textbf{What is the minimum value of variance a random variable?}
\choice{$-\infty$}{1}{0}{-1}{c}

\question \textbf{If $y=ax+b$, what is the value of $V(y)?$}
\choice{$aV(X)$}{$a^2V(X)$}{$V(X)$}{$a^2$}{b}

\question \textbf{If $y=ax+b$, what is the value of $E(y)?$}
\choice{$aE(X) + b$}{$a^2E(X)$}{$E(X)$}{$b$}{a}

\question \textbf{What is the value of $V(5)$?}
\choice{0}{25}{5}{1}{a}

\question \textbf{If $P(x) = \frac 1n; x = 1,2,3,\cdots ,n$, what is the value of $E(X)?$}
\choice{$\frac n2$}{$\frac{n-1}{2}$}{$\frac{n+1}{2}$}{$n+1$}{c}

\question \textbf{If $\displaystyle P(x)= \frac{4-|5-x|}{k}; x=2,3,4, \cdots 8$, what is the value of k?}
\choice{5}{8}{16}{24}{c}

\question \textbf{Expected value of a constant a is --}
\choice{1}{Variance}{a}{a+1}{c}

\question \textbf{The variance of a constant m is --}
\choice{0}{1}{m}{$m^2$}{a}

\question \textbf{What is $V(X-Y)$  equal to?}
\choice{$V(X)+V(Y)$}{$V(X)+V(Y)-2 Cov(X,Y)$}{$V(X)-V(Y)$}{$V(X)+V(Y)+2Cov(X,Y)$}{c}

\question \textbf{What is the value of V(2X+5)?}
\choice{$4V(X)-5$}{20}{$4V(X)$}{0}{c}

\question \textbf{If $P(x) = \frac 1{20}; x=1,2,3, \cdots,20,$ what is the 
standard deviation?}
\choice{1}{5.77}{7.75}{12.57}{a}

\question \textbf{Expectation measures --}
\choice{Dispersion}{Skewness}{Kurtosis}{Central tendency}{d}

\question \textbf{If $E(X) = -0.5$, then $E(1-2X) = $?}
\choice{0}{-1}{2}{1}{c}

\question \textbf{If $P(X) = \frac{1}{10}; x = 1,2,\cdots 10$, then $E(X) =$?}
\choice{10}{5.5}{0}{11}{b}

\question \textbf{Which formula of variance is correct?}
\choice{$V(X+Y) = V(X)+V(Y)-2Cov(X,Y)$}{$V(X+Y) = V(X)+V(Y)+2Cov(X,Y)$}{$V(X+Y) = V(X)+V(Y)-2Cov(X,Y)$}{$V(X+Y) = V(X)-V(Y)+2Cov(X,Y)$}{b}

\question \textbf{X is a constant; what is the value of $V(\frac X2)$?}

i) 0 \\
ii) $\frac12$ \\
iii) $\frac14$

\choice{ii}{i}{iii}{i and iii}{b}

\question \textbf{If $E(X)=2, E(X^2) = 8, V(X)= --$}
\choice{0}{2}{4}{8}{c}

\question \textbf{If $E(X)=3, E(X^2) = 11, V(X) = --$}
\choice{2}{5}{6}{4}{a}

\question \textbf{If $E(X) = 4$, what is $E(3X-2)$?}
\choice{10}{8}{7}{6}{a}

\question \textbf{If $E(X)=5, E(X^2) = 30, V(X) = --$}
\choice{3}{5}{4}{6}{b}

\question \textbf{If $E(X)=6$, what is $E(\frac{X}{2} + 1)$?}
\choice{4}{3}{2}{5}{a}

\question \textbf{If $E(X)=2, E(X^2) = 10, V(X) = --$}
\choice{5}{6}{7}{4}{b}

\question \textbf{If $E(X) = 7$, what is $E(4X+3)$?}
\choice{28}{30}{31}{29}{c}

\question \textbf{If $E(X) = 3$, what is $E(5 - X)$?}
\choice{2}{3}{4}{5}{a}

\question \textbf{If $E(X) = 4$ and $V(X) = 5$, what is $E(X^2)$?}  
\choice{9}{16}{21}{25}{c}

\question \textbf{If $E(X) = 3$ and $V(X) = 7$, what is $E(X^2)$?}  
\choice{9}{10}{16}{18}{c}

\question \textbf{If $E(X) = 5$ and $E(X^2) = 34$, what is $V(X)$?}  
\choice{6}{9}{10}{7}{b}

\question \textbf{If $E(X) = 2$ and $E(X^2) = 14$, what is $V(X)$?}  
\choice{10}{9}{8}{7}{a}

\question \textbf{If $E(X) = 6$ and $V(X) = 12$, what is $E(X^2)$?}  
\choice{36}{40}{48}{50}{c}



\question \textbf{If $P(x)= \frac{4-|5-x|}{k}; x=2,3,4, \cdots 8$, what is the value of $E(X)$?}
\choice{3}{8}{16}{5}{d}

\question \textbf{If $P(x)= \frac{6-|7-x|}{k}; x=2,3,4, \cdots 12$, what is the value of $E(X)$?}
\choice{6}{9}{13}{36}{d}

\question \textbf{If $P(x)= \frac{3-|4-x|}{k}; x=2,3,4, \cdots 6$, what is the value of k?}
\choice{6}{9}{10}{40}{b}

\question \textbf{If the variance of X is 3, what is the variance of V(3)?}
\choice{1}{2}{3}{0}{d}

\question \textbf{If $V(X) = 5,$, what is $V(X+5)?$}
\choice{0}{5}{10}{25}{b}

\question \textbf{If $V(X) = 5,$, what is $V(2X+5)?$}
\choice{20}{5}{10}{25}{a}

\question \textbf{If $E(X) = 2$ and $E(X^2)=8$, then the value of the $V(X) = ?$}
\choice{0}{2}{4}{8}{c}

\question \textbf{If $E(X^2) = 20$ and $V(X) = 11$, what is $E(X)$?}  
\choice{3}{4}{5}{6}{a}

\question \textbf{If $E(X^2) = 50$ and $V(X) = 14$, what is $E(X)$?}  
\choice{5}{6}{7}{8}{b}

\question \textbf{If $E(X^2) = 25$ and $V(X) = 9$, what is $E(X)$?}  
\choice{2}{3}{4}{5}{c}

\question \textbf{If $E(X^2) = 45$ and $V(X) = 21$, what is $E(X)$?}  
\choice{$4 \sqrt{3}$}{$2 \sqrt{6}$}{$6 \sqrt{2}$}{$7 \sqrt{2}$}{b}

\question \textbf{If $E(X^2) = 13$ and $V(X) = 4$, what is $E(X)$?}  
\choice{2}{3}{4}{5}{c}

\question \textbf{If $E(X) = 3$, what is $E(2X - 5)$?}  
\choice{1}{3}{5}{7}{a}  

\question \textbf{If $E(X) = 4$, what is $E(\frac{X}{2} + 3)$?}  
\choice{4}{5}{6}{7}{b}  

\question \textbf{If $E(X) = -2$, what is $E(3X + 7)$?}  
\choice{1}{-1}{-2}{4}{a}  

\question \textbf{If $E(X) = 6$, what is $E(5 - X)$?}  
\choice{1}{0}{-1}{2}{c}  

\question \textbf{If $E(X) = 10$, what is $E(4X - 8)$?}  
\choice{12}{40}{28}{32}{d}  


\question \textbf{If $\displaystyle  P(x) = \frac 1{15}; x = 1,2,3, \cdots 15$, what is the value of the expectation?}
\choice{8.5}{7.5}{7}{8}{d}

\subsection{Situation Set}

% Situation Set Starts
\textbf{Answer the next THREE questions based on the following information}

\begin{table}[h]
\centering
\begin{tabular}{cccc}
X & 0 & 1 & 2 \\ \hline
P(x) & $\frac 13$ & $\frac14$ & $\frac5{12}$
\end{tabular}
\end{table}

\question \textbf{What is the value of $E(X)$}
\choice{$\frac{15}{12}$}{$\frac{13}{12}$}{$\frac{1}{12}$}{$\frac{11}{13}$}{b}

\question \textbf{What is the value of $E(X^2)$}
\choice{$\frac{25}{12}$}{$\frac{13}{12}$}{$\frac{23}{12}$}{$\frac{25}{13}$}{b}

\question \textbf{What is $V(2X)$?}
\choice{2.93}{2.91}{1.97}{2.97}{d}
% Situation Set ENDS

% Situation Set Starts
\textbf{Answer the next THREE questions based on the following information}

\begin{table}[h]
\centering
\begin{tabular}{c|c|c|c}
X     & 1           & 2           & 3           \\ \hline
P(x)  & $\frac{1}{6}$ & $\frac{1}{2}$ & $\frac{1}{3}$
\end{tabular}
\end{table}


\question \textbf{What is the value of $E(X)$?}
\choice{$2.00$}{$2.17$}{$2.33$}{$2.50$}{b}

\question \textbf{What is the value of $E(X^2)$?}
\choice{$5.17$}{$4.83$}{$5.00$}{$5.33$}{a}

\question \textbf{What is $V(3X)$?}
\choice{$9.67$}{$11.33$}{$12.67$}{$4.25$}{d}
% Situation Set Ends

% Situation Set Starts
\textbf{Answer the next two questions based on the following information}

\begin{center}
The probability function of random variable $x$ is given below:

\( P(x) = \frac{x}{k}; x = 1, 2, 3, 4 \)
\end{center}

\question \textbf{What is the value of $k$?}
\choice{6}{10}{15}{20}{b}

\question \textbf{What is $E(X)$?}
\choice{2.25}{3.5}{2.5}{3.0}{d}
% Situation Set Ends

% Situation Set Starts
\textbf{Answer the next three questions based on the following information}

\begin{center}
The probability function of random variable $x$ is given below:

\( P(x) = \frac{2x + 1}{k}; x = 1, 2, 3, 4 \)
\end{center}

\question \textbf{What is the value of $k$?}
\choice{18}{25}{12}{24}{d}

\question \textbf{What is $E(X)$?}
\choice{1.75}{2.92}{3.25}{2.25}{b}

\question \textbf{What is $V(X)$?}
\choice{1.05}{3.0}{1.5}{1.25}{a}
% Situation Set Ends

% Situation Set Starts
\textbf{Answer the next two questions based on the following information}

\begin{center}
The probability function of random variable x is given below

\( P(x) = \frac{x-1}{k}; x = 2, 3, 4, 5 \)
\end{center}

\question \textbf{What is the value of k?}
\choice{2}{5}{10}{25}{c}

\question \textbf{What is $E(X)$?}
\choice{0.425}{0.525}{0.725}{0.625}{c}
% Situation Set Ends

\subsection{Multiple Completion}

% Multiple Completion Starts
\question \textbf{The possible relationship between $E(X) and E(X^2)$}

i. $E(X) \ge E(X^2)$ \\
ii. $E(X) \le E(X^2)$ \\
iii. $E(X) = E(X^2)$

\textbf{Which one is correct?}

\choice{i and ii}{i and iii}{ii and iii}{i, ii and iii}{b}
% Multiple Completion Ends

%--------------------------------------------------------------------------
\section{Binomial Distribution}
%--------------------------------------------------------------------------

\question \textbf{How many parameters are there in a binomial distribution?}
\choice{1}{2}{3}{4}{b}

\question \textbf{What is the Mean of Binomial Distribution?}
\choice{np}{npq}{nq}{$\sqrt{npq}$}{a}

\question \textbf{What is the Variance of Binomial Distribution?}
\choice{np}{npq}{nq}{$\sqrt{npq}$}{b}

\question \textbf{What is the Standard Deviation of Binomial Distribution?}
\choice{np}{npq}{nq}{$\sqrt{npq}$}{d}

\question \textbf{What is the Coefficient of Variation of Binomial Distribution?}
\choice{np}{npq}{$\frac{q}{np}$}{$\sqrt{npq}$}{c}

\question \textbf{Which is true of mean (np) of Binomial Distribution?}
\choice{$np=0$}{$np<0$}{$np>0$}{$np\ne0$}{c}

\question \textbf{In a Binomial distribution, how are mean and variance related?}
\choice{$Mean > Variance$}{$Mean < Variance$}{$Mean = Variance$}{$Mean =2 \times Variance$}{a}

\question \textbf{When does Binomial distribution tend to Poisson distribution?}
\choice{$n \rightarrow \infty$ and $p \rightarrow \infty$}{$n \rightarrow 0$ and $p \rightarrow 0$}{$n \rightarrow \infty$ and $p \rightarrow 0$}{$n \rightarrow 0$ and $p \rightarrow \infty$}{c}

\textbf{Answer the next two questions based on the following information.}

X is a binomial variate with expectation 4 and standard deviation $\sqrt 3$.

\question \textbf{What are the values of the parameters (mean and probability)?}
\choice{$16, \frac 14$}{$16, \frac 34$}{$15, \frac 14$}{$10, \frac 14$}{a}

\question \textbf{What is $P(X \neq 0)?$}
\choice{0}{0.01}{0.99}{1}{c}

\question \textbf{The characteristics of binomial distribution--}

i. $E(X) > V(X)$ \\
ii. $E(X) = V(X)$ \\
iii. $E(X) = np$

\textbf{Which one is correct?}

\choice{i and ii}{i and iii}{ii and iii}{i, ii and iii}{b}

\question \textbf{What is true of binomial distribution?}
\choice{There is one parameter}{Number of trial is fixed}{Mean is greater than variance}{Skewness is negative}{c}

\question \textbf{What is the skewness of binomial distribution?}
\choice{$\displaystyle \frac{(q-p)^2}{np}$}{$\displaystyle \frac{(q-p)^2}{np}$}{$\displaystyle \frac{(p+1)^2}{npq}$}{$\displaystyle \frac{(q+p)^2}{npq}$}{a}

\question \textbf{When is a binomial distribution positively skewed?}
\choice{p > q}{p = q}{p < q}{p+q < 1}{c}

\textbf{Answer the next two questions based on the following information}

\begin{center}
In a binomial distribution, $P(x=4) = \frac12 P(x=5); n  = 10$
\end{center}

\question \textbf{What is the mean?}
\choice{6.25}{5.15}{8.52}{5.22}{a}

\question \textbf{$P(x=2) =$ ---}
\choice{0.0053}{0.0069}{0.0085}{0.94}{b}

\question \textbf{In a binomial distribution with $p = 0.3$ and $n = 10$, what is $P(2)$?}  
\choice{0.2335}{0.2668}{0.3828}{0.1211}{c}  

\question \textbf{In a binomial distribution with $p = 0.4$ and $n = 12$, what is $P(3)$?}  
\choice{0.0896}{0.2131}{0.1419}{0.2942}{c}  

\question \textbf{In a binomial distribution with $p = 0.5$ and $n = 8$, what is $P(4)$?}  
\choice{0.2734}{0.3125}{0.2070}{0.0898}{a}  

\question \textbf{In a binomial distribution with $p = 0.2$ and $n = 15$, what is $P(5)$?}  
\choice{0.1789}{0.1887}{0.1032}{0.2413}{c}  

\question \textbf{In a binomial distribution with $p = 0.6$ and $n = 9$, what is $P(6)$?}  
\choice{0.2007}{0.2508}{0.2311}{0.7682}{b}  

\question \textbf{In a binomial distribution with $p = 0.3$ and $P(x) = 0.2508, n=9, x=?$}  
\choice{18}{10}{13}{6}{b}  

\question \textbf{In a binomial distribution with $p = 0.4$ and $P(x) = 0.1419$, what is $n$?}  
\choice{5}{6}{12}{15}{c}  

\question \textbf{In a binomial distribution with $p = 0.5$ and $P(2) = 0.1093$, what is $n$?}  
\choice{15}{1}{8}{12}{c}
 

\question \textbf{In a binomial distribution with $p = 0.2$ and $P(x) = 0.9389$, $n=?$}  
\choice{7}{12}{11}{15}{d}  

\question \textbf{In a binomial distribution with $p = 0.6$ and $P(5) = 0.02449$, $n=?$}  
\choice{3}{9}{10}{15}{b}  

\subsection{Situation Set}

% Situation Set Starts
\textbf{Answer the next THREE questions based on the following information}

\begin{center}
The mean of a Binomial distribution is 40 and standard deviation 6. 
\end{center}

\question \textbf{What is the value of $n$?}
\choice{200}{300}{400}{500}{c}

\question \textbf{What is the value of $1-q$?}
\choice{0.5}{0.2}{0.3}{0.1}{d}

\question \textbf{What is the value of $P(X\le 40)$?}
\choice{0.52}{0.54}{0.45}{0.91}{b}
% Situation Set Ends

\subsection{Multiple Completion}

\question \textbf{In a binomial distribution with parameters \( n \) and \( p \):}

i. The expected value is given by \( E(X) = np \). \\
ii. The variance is given by \( V(X) = np(1 - p) \). \\
iii. The standard deviation is given by \( \sqrt{np} \).

\textbf{Which one is correct?}

\choice{i and ii}{i and iii}{ii and iii}{i, ii and iii}{a}

\question \textbf{Which of the following statements about a binomial distribution are true?}

i. The probability of success remains constant for each trial. \\
ii. The trials are dependent on each other. \\
iii. The number of trials is fixed in advance.

\textbf{Which one is correct?}

\choice{i and ii}{i and iii}{ii and iii}{i, ii and iii}{b}

\question \textbf{Consider a binomial experiment. Which of the following statements is/are true?}  

i. Each trial results in exactly one of two possible outcomes. \\  
ii. The expected value is always greater than the variance. \\  
iii. The probability mass function of a binomial distribution can be computed using the binomial formula.  

\textbf{Which one is correct?}  

\choice{i and ii}{i and iii}{ii and iii}{i, ii and iii}{d}  

\question \textbf{Which of the following is/are correct about the binomial distribution?}  

i. The variance is maximized when \( p = 0.5 \). \\  
ii. If \( p = 1 \), the distribution becomes degenerate. \\  
iii. The standard deviation is given by \( \sqrt{np(1-p)} \).  

\textbf{Which one is correct?}  

\choice{i and ii}{i and iii}{ii and iii}{i, ii and iii}{d}  


%--------------------------------------------------------------------------
\section{Poisson Distribution}
%--------------------------------------------------------------------------

\question \textbf{The no. of parameters in a Poisson distribution is ---}
\choice{1}{2}{3}{4}{a}

\question \textbf{What is the mean of Poisson distribution}
\choice{$\frac 1{\sqrt m}$}{$m$}{$\frac 1m$}{$1+\frac 1m$}{b}

\question \textbf{Which relationship between mean and variance of Poisson Distribution is correct?}
\choice{$Mean > Variance$}{$Mean < Variance$}{$Mean = Variance$}{$Mean \ne Variance$}{c}

\question \textbf{What is the Variance of Poisson Distribution(with parameter m)?}
\choice{$\frac1{\sqrt{m}}$}{$\frac1m$}{$m$}{$\frac1{m+1}$}{c}

\question \textbf{What is the Standard Deviation of Poisson Distribution(with parameter m)?}
\choice{$\frac1{\sqrt{m}}$}{$\frac1m$}{$\sqrt{m}$}{$\frac1{m+1}$}{c}

\question \textbf{Which one is true of the parameter (m) of Poisson Distribution?}
\choice{$m=0$}{$m<0$}{$m>0$}{$m=1$}{c}

\question \textbf{The parameter of a Poisson Distribution is 5. What is its mean?}
\choice{2}{5}{2.24}{25}{b}

\question \textbf{When does Binomial Distribution tend to Poisson Distribution?}
\choice{$n \rightarrow \infty, p \rightarrow 0$ \& $np$ is finite}{$n \rightarrow \infty, p \rightarrow 0$ \& $np$ is infinite}{$n \rightarrow \infty, p \rightarrow  \infty$ \& $np$ is finite}{$n \rightarrow 0, p \rightarrow \infty$ \& $np$ is infinite}{a}

\question \textbf{The parameter of a Poisson variate is 2. What is its variance?}
\choice{0}{4}{$\sqrt 2$}{2}{d}

\question \textbf{The parameter of a Poisson variate is 5. What is its variance?}
\choice{10}{5}{$\sqrt{5}$}{25}{b}

\question \textbf{A Poisson distribution has a mean of 3. What is the variance?}
\choice{9}{3}{$\sqrt{3}$}{0}{b}

\question \textbf{X is a Poisson variate. P(2) = P(4). What is the value of the parameter?}
\choice{12}{3.46}{3.6}{4}{b}

\question \textbf{X is a Poisson variate. P(3) = P(5). What is the value of the parameter?}
\choice{4.5}{5}{2.3}{4.1}{a}

\question \textbf{For a Poisson variate X, if P(1) = P(3), what is the variance?}
\choice{2.5}{3.2}{2.45}{4.5}{c}

\question \textbf{For a Poisson variate $X$, if $P(2) = P(3)$, what is the variance?}  
\choice{3}{4}{5}{6}{a}


\textbf{Answer the next two questions based on the following information}

For a Poisson variate X, if P(2) = P(5).

\question \textbf{What is standard deviation?}
\choice{1.978}{1.998}{1.989}{1.889}{a}

\question \textbf{What is the value of P(2)?}
\choice{0.25}{0.14}{0.15}{0.02}{c}

\question \textbf{The standard deviation of a poisson distribution is 2. 
What is the parameter?}
\choice{2}{3}{4}{5}{c}

\question \textbf{Mean of a Poisson variate is a. What is its standard deviation?}
\choice{0}{a}{$a^{\frac 12}$}{$a^2$}{c}

\question \textbf{The standard deviation of a Poisson distribution is 3. What is the parameter?}
\choice{6}{9}{3}{4}{b}

\question \textbf{For a Poisson distribution with a mean of 5, what is the variance?}
\choice{5}{10}{25}{15}{a}

\question \textbf{If the variance of a Poisson distribution is 4, what is \( P(2) \)?}
\choice{0.1465}{0.1954}{0.1839}{0.2184}{a}

\question \textbf{If the variance of a Poisson distribution is 3.5, what is \( P(1) \)?}
\choice{0.1465}{0.1057}{0.1839}{0.2184}{b}

\question \textbf{A Poisson distribution has a mean of 7. What is the standard deviation?}
\choice{3.2}{4.1}{2.65}{1.78}{c}

\question \textbf{If \( P(2) \) in a Poisson distribution with parameter 
\(\lambda\) equals 0.2240, what is the parameter \(\lambda\)?}
\choice{2.4551}{1.2515}{1.2115}{2.5112}{b}

\question \textbf{A Poisson distribution has a mean of 4. What is \( P(3) \)?}
\choice{0.1465}{0.1954}{0.1839}{0.2381}{b}

\question \textbf{If the variance of a Poisson distribution is 3, what is the 
mean?}
\choice{3}{\(\sqrt{3}\)}{2}{6}{a}

\question \textbf{For a Poisson distribution with mean 6, what is the
probability of \( P(0) \)?}
\choice{0.0895}{0.012}{0.0454}{0.0024}{d}

\question \textbf{The mean of a Poisson distribution is 10. What is its 
standard deviation?}
\choice{5}{\(\sqrt{10}\)}{10}{\(\sqrt{20}\)}{b}

\question \textbf{Given that the parameter of a Poisson distribution is 8, 
what is the variance?}
\choice{4}{8}{\(\sqrt{8}\)}{16}{b}

\subsection{Multiple Completion}

\question \textbf{For a Poisson-distributed variable with mean $\lambda = 4$, which of the following is true?}

i. $E(X) = 4$ \\  
ii. $V(X) = 2$ \\  
iii. $E(X^2) = 18$

\textbf{Which one is correct?}  

\choice{i and ii}{i and iii}{ii and iii}{i, ii and iii}{b}  

\question \textbf{If $X \sim \text{Poisson}(m = 3)$, which of the following holds?}

i. $E(X) = 3$ \\  
ii. $V(X) = 3$ \\  
iii. $E(X^2) = 12$  

\textbf{Which one is correct?}  

\choice{i and ii}{i and iii}{ii and iii}{i, ii and iii}{d}  

\question \textbf{For a Poisson distribution, which of the following statements are true?}

i. The mean and variance are always equal. \\  
ii. The distribution is always symmetric. \\  
iii. The probability of zero occurrences is given by $e^{-m}$.  

\textbf{Which one is correct?}  

\choice{i and ii}{i and iii}{ii and iii}{i, ii and iii}{b}  

\question \textbf{A Poisson-distributed random variable has mean $\lambda = 6$. Among the following properties ---}

i. $E(X) = 6$ \\  
ii. $V(X) = 6$ \\  
iii. $P(X = 0) = e^{-6}$  

\textbf{Which one is correct?}  

\choice{i and ii}{i and iii}{ii and iii}{i, ii and iii}{d}  

\question \textbf{For a Poisson process with $\lambda = 5$, which of the following is true?}

i. The standard deviation is $\sqrt{5}$. \\  
ii. $P(X \geq 1) = 1 - e^{-5}$. \\  
iii. $E(X^2) = 30$  

\textbf{Which one is correct?}  

\choice{i and ii}{i and iii}{ii and iii}{i, ii and iii}{d}  

%--------------------------------------------------------------------------
\subsection{Problems}
%--------------------------------------------------------------------------

\question \textbf{On average, 1 in 1000 houses in a city gets a fire-burn in a year.If there are 2000 houses, what is the probability that, in a certain year, exactly 5 house will be burnt?}
\choice{0.036}{0.040}{0.027}{0.091}{a}

%--------------------------------------------------------------------------
\section{Vital Statistics}
%--------------------------------------------------------------------------

\question \textbf{What is the called the ratio of the dependent population to the earning population?}
\choice{Dependency ratio}{Sex ration}{Population density}{Growth rate}{a}

\question \textbf{Which of the following best describes the dependency ratio?}
\choice{The ratio of the elderly population to the working-age population}
{The ratio of the combined non-working (0-14 and 65+) population to the working-age (15-64) population}
{The proportion of young dependents (0-14) in the population}
{The total population divided by the number of children (0-14)}
{b}

\question \textbf{City A has 12,000 individuals aged 0-14, 35,000 aged 15-64, and 5,000 aged 65+. What is the dependency ratio?}
\choice{0.31}{0.48}{0.60}{0.25}{b}

\question \textbf{A City has a dependency ratio of 0.52. If its working-age population (15-64) is 50,000, what is the total number of dependents (0-14 and 65+)?}
\choice{15,600}{20,000}{26,000}{30,000}{c}

\textbf{Answer the following 2 questions based on the information given below.}  

\begin{table}[h]
\centering
\begin{tabular}{|c|c|c|}
\hline
\textbf{City} & \textbf{Population (in thousands)} & \textbf{Area (in km\(^2\)} \\ \hline
Gamma         & 1200                              & 400                        \\ \hline
Delta         & 800                               & 320                        \\ \hline
\end{tabular}
\end{table}  

\question \textbf{What is the population density of City Delta?}  
\choice{2 people/km\(^2\)}{4 people/km\(^2\)}{2.5 people/km\(^2\)}{2.2 people/km\(^2\)}{b}  

\question \textbf{Which city is less densely populated?}  
\choice{Gamma}{Delta}{Both are equal}{Cannot be determined}{b}  

\textbf{Answer the following two questions based on the information given below.}  

In a city, the total number of live births in a year was 2,400. The number of women aged 15-49 years in the population was 48,000.  

\question \textbf{Calculate the General Fertility Rate (GFR) for the city.}  
\choice{40 per 1,000 women}{50 per 1,000 women}{60 per 1,000 women}{30 per 1,000 women}{b}  

\question \textbf{If live births increase to 3,000 while the number of women aged 15-49 remains the same, what is the new GFR?}  
\choice{55 per 1,000 women}{65 per 1,000 women}{50 per 1,000 women}{62.5 per 1,000 women}{d}  

\question \textbf{The population of a city is 500,000, and the number of 
live births recorded in a year is 8,000. What is the Crude Birth Rate (CBR)?}
\choice{12 per 1,000}{16 per 1,000}{20 per 1,000}{22 per 1,000}{b}

\question \textbf{What is the formula of population density?}
\choice{$\frac{M}{F}\times 100$}{$\frac{F}{M}\times 100$}{$\frac{B}{P}\times 100$}{$\frac{P}{A}$}{d}

\question \textbf{In the following data, what is the dependency ratio?}

\begin{table}[h]
\centering
\begin{tabular}{c|ccccccc}
Age          & 0-14   & 15-24  & 25-34  & 35-44  & 45-54  & 55-64  & 65+    \\ \hline
Populatation & 31,500 & 40,000 & 48,000 & 41,000 & 32,000 & 25,000 & 16,000
\end{tabular}
\end{table}

\choice{35.54\%}{25.54\%}{23.24\%}{31.25\%}{b}

\question \textbf{Crude Birth Rate (CBR) is:}
\choice{$\frac BP \times 100$}{$\frac BP \times 1000$}{$\frac PB \times 100$}{$\frac FP \times 100$}{b}

\question \textbf{Which one is a measure of reproduction?}

i) CBR \\
ii) CDR \\
iii) NRR

\choice{i}{ii}{iii}{i and ii}{c}

\question \textbf{The number of people living per unit area is called--}
\choice{Population Index}{Population Density}{Human Development Index}{Dependency Ratio}{b}

\question \textbf{Which formula of GFR is accurate?}
\choice{$GFR = \frac{B}{P}\times 1000$}{$GFR = \frac{B}{F_{15-49}}\times 1000$}{$GFR = \frac{B_i}{F_i}\times 1000$}{$GFR = \frac{G_i}{F{15-49}}\times 1000$}{b}

\question \textbf{Total number of children born to each 1000 people in any country or region is called --}
\choice{TFR}{GFR}{CBR}{GRR}{c}

\question \textbf{A city has a dependency ratio of 0.48. If the working-age population (15–64) is 62,500, what is the number of dependents (ages 0–14 and 65+)?}  
\choice{30,000}{25,000}{22,000}{20,000}{a}

\question \textbf{The dependency ratio of a town is 0.60. If there are 40,000 people aged 15–64, how many individuals are considered dependents?}  
\choice{22,000}{26,500}{24,000}{25,000}{c}


\subsection{Growth Rates}

\question 
\textbf{If $n$ in $P_n = P_o(1+r)^n$ is split into infinite parts and $r$ adjusted accordingly, what type of growth do we have?}

\choice{Simple growth}{Arithmetic growth}{Exponential growth}{Geometric growth}{c}

\textbf{Answer the next two questions based on the following information}

\begin{table}[h]
\centering
\begin{tabular}{c|c|c|c|c}
Year & 1 & 2 & 3 & 4 \\ \hline
Population & 100 & 110 & 120 & 130
\end{tabular}
\end{table}

\question \textbf{Which type of growth is seen here?}
\choice{Arithmetic growth}{Geometric growth}{Exponential growth}{None}{a}

\question \textbf{What is the rate of increase?}
\choice{1}{0.1}{10}{1\%}{b}

\question \textbf{In exponential growth, when is a population doubled?}
\choice{$\frac{\log_{10} 2}{r}$}{$\frac{\log_e 2}{r}$}{$\frac{\log_e 2}{r^2}$}{$\frac{\log_e 3}{r}$}{b}

\question \textbf{If a population exponentially declines, when is it reduced to half?}  
\choice{$\frac{\log_{10} 2}{r}$}{$\frac{\log_e 2}{r}$}{$\frac{\log_e 2}{r^2}$}{$\frac{\log_e 3}{r}$}{b}  

\question \textbf{How long does it take for a population to triple in exponential growth?}  
\choice{$\frac{\log_{10} 3}{r}$}{$\frac{\log_e 3}{r}$}{$\frac{\log_e 3}{r^2}$}{$\frac{\log_e 2}{r}$}{b}  

\textbf{Answer the next two questions based on the following information}

\begin{center}
Statement
\end{center}

\question \textbf{Vital statistics records --}

i. marriage \\
ii. birth \\
iii. sickness and death

\textbf{Which one is correct?}

\choice{i and ii}{i and iii}{ii and iii}{i, ii and iii}{d}

\end{questions}

\newpage  %Uncomment to put on new age
%\bigskip

\begin{multicols}{3}
[
\textbf{Answer Key:}
]
\showallanswers % Phil Hirschorn
  \end{multicols}


\end{document}