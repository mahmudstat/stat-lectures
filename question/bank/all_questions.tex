\documentclass[12pt]{exam}
\usepackage{amsmath,amsthm,amsfonts,amssymb,dsfont} % Changed dsmath to dsfont for broader compatibility
\usepackage{float}
\usepackage{ifthen}
\usepackage{array}
\usepackage{geometry}
\geometry{
legalpaper, total={177.8mm, 290mm},left=10mm, right=10mm,
top=7mm, bottom=10mm,
}
\usepackage{enumerate}% http://ctan.org/pkg/enumerate
\usepackage{multicol} %as questions will be in a single column
\usepackage{hhline}
\usepackage[table]{xcolor}

% Accumulate the answers. Unmodified from Phil Hirschorn's answer
% https://tex.stackexchange.com/questions/15350/showing-solutions-of-the-questions-separately/15353
\newbox\allanswers
\setbox\allanswers=\vbox{}

\newenvironment{answer}
{%
  \global\setbox\allanswers=\vbox\bgroup
  \unvbox\allanswers
}%
{%
  \bigbreak
  \egroup
}

\newcommand{\showallanswers}{\par\unvbox\allanswers}
% End Phil's answer


% Is there a better way?
\newcommand*{\getanswer}[5]{%
  \ifthenelse{\equal{#5}{a}}
  {\begin{answer}\thequestion. (a)~#1\end{answer}}
  {\ifthenelse{\equal{#5}{b}}
    {\begin{answer}\thequestion. (b)~#2\end{answer}}
    {\ifthenelse{\equal{#5}{c}}
      {\begin{answer}\thequestion. (c)~#3\end{answer}}
      {\ifthenelse{\equal{#5}{d}}
        {\begin{answer}\thequestion. (d)~#4\end{answer}}
        {\begin{answer}\textbf{\thequestion. (#5)~Invalid answer choice.}\end{answer}}}}}
}

\setlength\parindent{0pt}
%usage \choice{ }{ }{ }{ }
%(A)(B)(C)(D)
\newcommand{\fourch}[5]{
  \par
  \begin{tabular}{*{4}{@{}p{0.23\textwidth}}}
    (a)~#1 & (b)~#2 & (c)~#3 & (d)~#4
  \end{tabular}
  \getanswer{#1}{#2}{#3}{#4}{#5}
}

%(A)(B)
%(C)(D)
\newcommand{\twoch}[5]{
  \par
  \begin{tabular}{*{2}{@{}p{0.46\textwidth}}}
    (a)~#1 & (b)~#2
  \end{tabular}
  \par
  \begin{tabular}{*{2}{@{}p{0.46\textwidth}}}
    (c)~#3 & (d)~#4
  \end{tabular}
  \getanswer{#1}{#2}{#3}{#4}{#5}
}

%(A)
%(B)
%(C)
%(D)
\newcommand{\onech}[5]{
  \par
  (a)~#1 \par (b)~#2 \par (c)~#3 \par (d)~#4
  \getanswer{#1}{#2}{#3}{#4}{#5}
}

\newlength\widthcha
\newlength\widthchb
\newlength\widthchc
\newlength\widthchd
\newlength\widthch
\newlength\tabmaxwidth

\setlength\tabmaxwidth{0.96\textwidth}
\newlength\fourthtabwidth
\setlength\fourthtabwidth{0.25\textwidth}
\newlength\halftabwidth
\setlength\halftabwidth{0.5\textwidth}

\newcommand{\choice}[5]{%
\settowidth\widthcha{AM.#1}\setlength{\widthch}{\widthcha}%
\settowidth\widthchb{BM.#2}%
\ifdim\widthch<\widthchb\relax\setlength{\widthch}{\widthchb}\fi%
  \settowidth\widthchb{CM.#3}%
\ifdim\widthch<\widthchb\relax\setlength{\widthch}{\widthchb}\fi%
  \settowidth\widthchb{DM.#4}%
\ifdim\widthch<\widthchb\relax\setlength{\widthch}{\widthchb}\fi%

\ifdim\widthch>\halftabwidth
  \onech{#1}{#2}{#3}{#4}{#5}
\else\ifdim\widthch<\halftabwidth
\ifdim\widthch>\fourthtabwidth
  \twoch{#1}{#2}{#3}{#4}{#5}
\else
  \fourch{#1}{#2}{#3}{#4}{#5}
\fi\fi\fi}

\begin{document}

\begin{center}
\textbf{\Huge Statistics Question Bank}
\end{center}
\hrule
\vspace{1cm} % Add some space after the header

\tableofcontents % Automatically generates the table of contents

\newpage % Start the questions on a new page after the TOC



\part{Probability}\begin{questions}
\section{Probability}
\subsection{Permutation}
\subsubsection{Single}
\question \textbf{Three objects can be placed in 2 positions in -- ways.}
\choice{3}{4}{6}{8}{c}
\subsection{Combination}
\subsubsection{Single}
\question \textbf{In how many ways can a team of 2 be formed from 4 people?}
\choice{4}{6}{8}{12}{b}
\subsection{Permutation}
\subsubsection{Single}
\question \textbf{$\displaystyle ^np_r=$}
\choice{$\displaystyle \frac {n!}{(n-r)!}$}{$\displaystyle \frac {n!}{(n+r)!}$}{$\displaystyle \frac {n!}{r!}$}{$\displaystyle \frac {n!}{(r-n)!}$}{a}
\subsection{Combination}
\subsubsection{Single}
\question \textbf{$\displaystyle ^nC_r=$}
\choice{$\displaystyle \frac {n!}{(n-1)!(n+r)!}$}{$\displaystyle \frac {r!}{n!(n-r)!}$}{$\displaystyle \frac {n!(n-1)!}{r!}$}{$\displaystyle \frac {n!}{(r-n)!}$}{a}
\subsection{Conceptual}
\subsubsection{Single}
\question \textbf{What is the probability that at least one item in a sample space will occurr?}
\choice{0}{0.5}{1}{Undefined}{c}
\question \textbf{The probability of two disjoint sets happening together is:}
\choice{0.5}{0}{1}{$0  \leq x < 1$}{b}
\question \textbf{How many additive laws of probability are there?}
\choice{1}{2}{3}{4}{b}
\question \textbf{$P(A\cup B) = P(A) + P(B)$ implies A \& B are --}
\choice{Disjoint}{Independent}{Joint}{Independent}{a}
\question \textbf{$P(A\cap B) = P(A) \times P(B)$ implies A \& B are --}
\choice{Disjoint}{Independent}{Joint}{Independent}{b}
\question \textbf{Which is the formula of classical approach of probability?}
\choice{$P=\frac{\text{No. of favorable outcomes}}{\text{Total no. of possible outcomes}}$}{$P=\frac{\text{No. of total outcomes}}{\text{No. of favorable outcomes}}$}{$P = \lim_{n(S)\to\infty} \frac{n(A)}{n(S)}$}{$P = \lim_{n(A)\to\infty} \frac{n(A)}{n(S)}$}{a}
\question \textbf{Which is the formula of empirical/relative frequency approach of probability?}
\choice{$P=\frac{\text{No. of favorable outcomes}}{\text{Total no. of possible outcomes}}$}{$P=\frac{\text{No. of total outcomes}}{\text{No. of favorable outcomes}}$}{$\displaystyle P = \lim_{n(S)\to\infty} \frac{n(A)}{n(S)}$}{$\displaystyle P = \lim_{n(A)\to\infty} \frac{n(A)}{n(S)}$}{a}
\question \textbf{What is the correct formula for conditional probability?}
\choice{$P(A|B) = \frac{P(A \cap B)}{P(B|A)}$}{$P(A|B) = \frac{P(A \cap B)}{P(A)}$}{$P(A|B) = \frac{P(A \cap B)}{P(B)}$}{$P(A|B) = \frac{P(B|A)}{P(B|A)}$}{a}
\question \textbf{The third axiom of probability is --}
\choice{$0 \le P(A) \le 1$}{$P(S) = 1$}{$\displaystyle P(A_1 U A_2 U \cdots U
A_n) = \sum_{i=1}^{\infty}P(A_i)$}{$P(A) = 1 - P(A)$}{c}
\subsubsection{Multiple Completion}
\question \textbf{Possible value of probability}

i. -1 \quad
ii. 0.5 \quad
iii. 0

\textbf{Which one is correct?}

\choice{i and ii}{i and iii}{ii and iii}{i, ii and iii}{c}
\subsubsection{Single}
\question \textbf{An act repeated under some specific conditions is called --}
\choice{Event}{Experiment}{Sample}{Sample space}{b}
\question \textbf{$P(0)$ implies -- }
\choice{A certain event}{An uncertain event}{An impossible event}{A probable event}{c}
\question \textbf{Events having some common elements are called --}
\choice{Complementary events}{Mutually exclusive events}{Exhaustive events}{Non-Mutually exclusive events events}{a}
\question \textbf{The minimum value of probability is}
\choice{$-\alpha$}{1}{0}{-1}{c}
\question \textbf{Each element of sample space is called--}
\choice{Trial}{Experiment}{Variable}{Sample Point}{d}
\question \textbf{Two events not ocurring together are called--}
\choice{dependent Events}{Independent Events}{Mutually Exclusive Events}{Marginal Events}{c}
\question \textbf{If A and B are independent, which formula is correct?}
\choice{$P(A \cap B) = P(A) \cdot P(B)$}{$P(A \cap B) = P(\bar A) \cdot P(B)$}{$P(A \cap B) = P(A) \cdot P(\bar B)$}{$P(A \cap \bar B) = P(A) \cdot P(B)$}{a}
\question \textbf{Which of the following are disjoint events?}
\choice{$A = \{1, 2, 3\}, B = \{4, 5\}$}{$A = \{a, b\}, B = \{b, c\}$}{$A = \{0\}, B = \{0, 1\}$}{$A = \{x, y\}, B = \{x, y\}$}{a}
\question \textbf{Which of the following are disjoint events?}
\choice{$P = \{1, 2\}, Q = \{2, 3\}$}{$P = \{x\}, Q = \{x, y\}$}{$P = \{1, 3\}, Q = \{3, 5\}$}{$P = \{m, n\}, Q = \{p, q\}$}{d}
\question \textbf{Let the sample space be $S = \{1, 2, 3, \dots, 10\}$. Which of the following pairs of events are disjoint?}

i. $A$: Number is prime, \quad $B$: Number is greater than 3 \\
ii. $A$: Number is even, \quad $B$: Number is divisible by 3 \\
iii. $A$: Number is less than 5, \quad $B$: Number is greater than 6

\textbf{Which one is correct?}

\choice{i and ii}{i and iii}{ii and iii}{i, ii and iii}{c}\end{questions}
\vspace{.3cm}

\begin{center}

“The only thing that’s certain is uncertainty.” --- John Allen Paulos

  --- Abdullah Al Mahmud ---
\end{center}

\pagebreak
\bigskip

\begin{multicols}{3}
[
Answer Key
]
\showallanswers % Phil Hirschorn
\end{multicols}

\end{document}
