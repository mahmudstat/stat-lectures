\documentclass[a4paper,oneside, margin=1.4in]{book}

\usepackage{lipsum}
\usepackage[hidelinks]{hyperref}
\usepackage{titletoc}
\usepackage{amsmath}
\usepackage{geometry}
\usepackage{graphicx}
\graphicspath{ {.} }
\geometry{a4paper, margin=1in}
\titlecontents*{chapter}
  [0pt]% <left>
  {}
  {\chaptername\ \thecontentslabel\quad}
  {}
  {\bfseries\hfill\contentspage}    

\usepackage{bookmark}
\usepackage{etoolbox}

\makeatletter
\newcommand*{\AddChapterPrefixInBookmarks}{%
  \if@mainmatter
    \ifnum\bookmarkget{level}=0 %
      \preto\bookmark@text{\@chapapp\space}%
    \fi
  \fi
}
\makeatother

\bookmarksetup{
  numbered,
  addtohook=\AddChapterPrefixInBookmarks,
}

% Workaround for numbered sections in unnumbered
% chapter "Introduction" to avoid chapter number
% zero.
\renewcommand*{\thesection}{%
  \ifcase\value{chapter}%
  \else
    \thechapter.%
  \fi
  \arabic{section}%
}

\title{My document}

\begin{document}
\frontmatter

\begin{titlepage}
    \begin{center}
        \vspace*{1cm}
            
        \Huge
        \textbf{Statistics Question Bank}
            
        \vspace{0.5cm}
        \huge
        Second Paper
            
        \vspace{1.5cm}
            
        \textbf{Abdullah Al Mahmud}

     \vspace{1.5cm}

	\Large 
	Updated on: \today
            
        \vfill
            

            
        \vspace{0.8cm}
            
\includegraphics[width=1cm]{logo.png}
            
        \Large
        www.statmania.info\\
            
    \end{center}
\end{titlepage}


\tableofcontents


\mainmatter
\chapter{Probability} 
\section{Creative Questions}

\begin{enumerate}

     \item
	  \textbf{Events that do not depend on each other are called independent events, and events that cannot occurr simulataneously are called disjoint events.} 
  
  \begin{enumerate}
    \item
	Provide an example of disjoint events, using the set theory. \hfill 1
    \item
	Prove that $P(A\cap \bar B) = P(A) - P(A\cap B)$ \hfill 2
    \item  
	If there are k mutually and exhaustive events, prove $\displaystyle \sum_{i=1}^k P(A_i) = 1$ \hfill 3
    \item
	Prove that two events cannot be simulataneously independent and mutually exclusive. \hfill 4
  \end{enumerate}

 \item
	  \textbf{A quality control analyst in an industry tracks the no. of defective items produced per day. He observes 150 successive days and then prepares a table.} 
	  
	  \begin{table}[h]
	  \centering
\begin{tabular}{c|c|c|c|c|c} \hline
No. of items & 0 & 1 & 2 & 3 & 4 \\ \hline
Frequency & 30 & 32 & 40 & 28 & 20 \\ \hline
\end{tabular}
\end{table}
  
  \begin{enumerate}
    \item
	What is the formula of classical probability? \hfill 1
    \item
	Explain the difference between Priori Approach and Empirical Approach of probability. \hfill 2
    \item  
	What is the probability that less than 2 defective items would be produced on a particular day? \hfill 3
    \item
	Explain the relationship between independency and mutual excluvity in the light of the stem. \hfill 4
  \end{enumerate}
  
     \item
	  \textbf{Ratul and Tomal both have an unbiased die. Both have randomly thrown their dice once. } 
  
  \begin{enumerate}
    \item
	What are equally likely events? \hfill 1
    \item
	If a die is thrown once, what is the probability of getting a prime number? \hfill 2
    \item  
	From the stem, what is the probability that the sum of numbers appearing \\ on the dice is greater than 6. \hfill 3
    \item
	Examine: the probabilities of getting the sum less than 6 and greater than 6 are equal. \hfill 4
  \end{enumerate}

  
   \item
	  \textbf{It is observed that 50\% of mails are spam. A software filters spam mail before reaching the inbox. Its accuracy for detecting a spam mail is 99\% and chances of tagging a non-spam mail as spam mail is 5\%.} 
  
  \begin{enumerate}
    \item
	What is a disjoint event? \hfill 1
    \item
	For two independent events, what does the Bayes' theorem reduce to? \hfill 2
    \item  
	What is the probability that a mail is tagged as spam?  \hfill 3
    \item
	If a certain mail is tagged as spam, find the probability that it is not a spam mail. \hfill 4
  \end{enumerate}
  
  \item
	  \textbf{A dope test correctly identifies a drug user as positive 90\%  of the time, but incorrectly identifes 20\% non-users as users. The probability of drug use is 0.05.} 
  
  \begin{enumerate}
    \item
	Write down the formula of conditional probability. \hfill 1
    \item
	Express $P(A|B)$ in terms of $P(B|A)$. \hfill 2
    \item  
	Find the probability of testing positive in the test. \hfill 3
    \item
	If the test shows a user positive, what is the probability that the person is actually a user? \hfill 4
  \end{enumerate}

 \item
	  \textbf{A red and a blue dice are thrown once. The dice are absolutely neutral and independent.} 
  
  \begin{enumerate}
    \item
	What is a simple event? \hfill 1
    \item
	Give an example of a certain event using set theory. \hfill 2
    \item  
	Find the probability that the difference of two digits from two dices is less than 3.\hfill 3
    \item
	Are the probabilities of getting greater digit from the blue die and that from the red die equal? Justify. \hfill 4
  \end{enumerate}
  
  \item
	  \textbf{An unbiased coin is tossed 10 times.} 
  
  \begin{enumerate}
    \item
	If a coin is flung 3 times, how many outcomes are generated? \hfill 1
    \item
	If a coin is flung n times, show how many outcomes are generated. \hfill 2
    \item  
	What is the probability of getting a) at least 3 heads, b) at most 3 heads? \hfill 3
    \item
	Are these probabilities equal? a) Getting at least 2 heads \& b) Getting at least 2 tails. \\ Also justify logically. \hfill 4
  \end{enumerate}
  
  \item
  \textbf{It is observed that in a college, there are 100 students, of whom 30 play football, 40 play cricket, and 20 play both.}
 
  \begin{enumerate}
    \item
	What is the range of probability? \hfill 1
    \item
    	What is the relationship between independence and mutual excluvity?  \hfill 2
    \item
    	Are the probabilities of playing cricket and that of football independent? Prove. \hfill 3
     \item
     	If a student is selected randomly, and if he does not play cricket, what is the probability that \\ he plays football? \hfill 4
  \end{enumerate}

 \item
	  \textbf{A box contains four blue and 6 green balls. 3 balls are drawn randomly.} 
  
  \begin{enumerate}
    \item
	What is the value of $^nC_r$? \hfill 1
    \item
	Illustrate the difference between permutation and combination with an example. \hfill 2
    \item  
	What is the probability that all balls are green? \hfill 3
    \item
	What is the probabilith that one ball has a different color? \hfill 4
  \end{enumerate}

\item
	  \textbf{Sadman has an urn with 5 red and 4 white balls. He has randomly drawn two balls from the urn.} 
  
  \begin{enumerate}
    \item
	What is the probability of an uncertain event? \hfill 1
    \item
	Write the third axiom of probability. \hfill 2
    \item  
	What is the probability that both the balls drawn by Sadman are white? \hfill 3
    \item
	Are the probabilities of both balls being same color and different color equal? Analyze. \hfill 4
  \end{enumerate}

 \item
	  \textbf{Two dice are thrown together. The dice are named A and B.} 
  
  \begin{enumerate}
    \item
	What is P(A=7)? \hfill 1
    \item
	Create the sample space. \hfill 2
    \item  
	What is the probability that the outcomes of A \& B are different? \hfill 3
    \item
	Determine the probability that the summation of outcome of two dice is a prime number. \hfill 4
  \end{enumerate}

 \item
	  \textbf{A magician draws two cards from a pack (i) with replacement and then (ii) without replacement. The cards were well-shuffled before drawing.} 
  
  \begin{enumerate}
    \item
	What is the probability of an impossible event? \hfill 1
    \item
	How to determine the probability of a joint event?  \hfill 2
    \item  
	As per (i), what is the probability that the cards have different color? \hfill 3
    \item
	As per (ii), what is the probability that the cardsare aces of same color?  \hfill 4
  \end{enumerate}

 \item
	  \textbf{$P(A) = \frac{3}{10}, P(B) = \frac 25, P(B\cup A) = \frac12$} 
  
  \begin{enumerate}
    \item
	What is an independent event? \hfill 1
    \item
	What is the relationship between independency and mutual excluvity? \hfill 2
    \item  
	Find $P(A \vert B)$ and $P(B \vert A)$ \hfill 3
    \item
	Verify the equality mathematically \& empirically: $P(B) = P(A) \cdot P(B \vert A) + P(\bar A) \cdot P(B \vert \bar A)$ \hfill 4
  \end{enumerate}
  
   \item
  \textbf{$P(A\vert B) = \frac 1 8, P(A) = \frac 12, P(B) = \frac 15$}
 
  \begin{enumerate}
    \item
	Write down the range of probability. \hfill 1
    \item
    	Find $P(A\cap B)$.  \hfill 2
    \item
    	Find $P(A\vert \bar B)$. \hfill 3
     \item
     	Are the probabilities $P(A\vert B)$ and $P(B\vert A)$ equal? Justify\hfill 4
  \end{enumerate}

 \item
  \textbf{Sakib has recently graduated from the University of Dhaka. he applies to two firms - EduCube \& Digic- for a Data Analyst job. The probability of hiring by EduCube is 0.8 and by Digic is 0.4. The probability that none hires is 0.5.} 
  
  \begin{enumerate}
    \item
	What is a sample space? \hfill 1
    \item
	Explain how to find $P(\bar A \cap B)$ using Venn Diagram. \hfill 2
    \item  
	Find the probability of hiirng by by Digic but not by EduCube. \hfill 3
    \item
	Find the probability that no firm will reject him. \hfill 4
  \end{enumerate}

 \item
	  \textbf{Recently there is an increase in the number of electronic medias in Bangladesh. A professor stated in the class room that very few people now resort to print media for news. A research indicates 70\% people collect news from electronic media, 60\% from print media, and 50\% from both.} 
  
  \begin{enumerate}
    \item
	What is an impossible event? \hfill 1
    \item
	Write the event "None of the two occurs" in two different notations. \hfill 2
    \item  
	What is the probability of getting news from at most one type of media? \hfill 3
    \item
	Is the professor correct in his/her statement? Analyze. \hfill 4
  \end{enumerate}
  


  \end{enumerate}

\section{Short Questions}

  \begin{enumerate}
    \item
	Question \hfill 1
    \item
	Question \hfill 2
    \item  
	Question \hfill 3
    \item
	Question \hfill 4
  \end{enumerate}

%---------------------------------------------------------------------------------------
\chapter{Random Variable and Probability Function} 
%---------------------------------------------------------------------------------------

\section{Creative Questions}

  \begin{enumerate}
  
 \item
	  \textbf{A deck of 52 card is well-shuffled and three cards are drawn from them at random. The number of kings obtained is denoted by x.} 
  
  \begin{enumerate}
    \item
	What are equaly likely events? \hfill 1
    \item
	Differentiate between with replacement and without replacement drawings. \hfill 2
    \item  
	Form the probability fucntion using the above information and then form the distribution.\hfill 3
    \item
	Examine the statement: $P(1 \le x \le 3) = F(3)-F(1)$ \hfill 4
  \end{enumerate}
  
     \item
  \textbf{The joint probability function of two random variables X and Y is given below:}
  
  \begin{center}
  $\displaystyle P(X,Y) = \frac {x+2y}{16}; x = 0, 1; y = 0 ,1,2,3$
 \end{center}
 
  \begin{enumerate}
    \item
	Write down the formula of conditional proibability. \hfill 1
    \item
    	What is the relationship between marginal and joint probability? \hfill 2
    \item
    	Find P(X). \hfill 3
     \item
     	Find $P(X\vert Y)$ and $P(X\vert 0)$. \hfill 4
  \end{enumerate}
  
   \item
	  \textbf{The probability distributions of a random variable X in two different cases are given below:} 
	  
	  \begin{table}[h]
	  \caption {\textbf{Distribution - A}}
	   \begin{center}
\begin{tabular}{llllllll}
x    & 0    & 1    & 2    & 3 & 4    & 5    & 6    \\  \hline
P(x) & 0.20 & 0.10 & 0.08 & w & 0.02 & 0.10 & 0.30
\end{tabular}
 \end{center}
\end{table}

\begin{table}[h]
	  \caption {\textbf{Distribution - B}}
	   \begin{center}
\begin{tabular}{llllll}
x    & 0    & 1    & 2    & 3    & 4 \\ \hline
P(x) & 0.20 & 0.10 & 0.30 & 0.50 & 0.20   
\end{tabular}
\end{center}
\end{table}
  
  \begin{enumerate}
    \item
	What is a probability mass function? \hfill 1
    \item
	Can we dtermine the probability of a certain value of a discrete random variable? \hfill 2
    \item  
	What is the value of w? \hfill 3
    \item
	Which table is a proper probability distribution? Justify with mathematical reasoning. \hfill 4
  \end{enumerate}
  
   \item
	  \textbf{A continuos random variable X follows the following probability density function (pdf).} 
	  \begin{center}
	  $f(x) = 6x(1-x); 0\le x\le 1$
  \end{center}
  
  \begin{enumerate}
    \item
	Give an example of a continous random variable. \hfill 1
    \item
	Examine whether the given function is a pdf. \hfill 2
    \item  
	If $P(X>a) = P(X<a)$, find the value of a. \hfill 3
    \item
	Should $P(0.5 \le X \le 1)$  be equal to 0.5? \hfill 4
  \end{enumerate}
  
     \item
	  \textbf{The probability mass function (pmf) of a football striker scoring no. of hattricks during the course of a league season is given below}
	  
	  \begin{center}
	  $\displaystyle P(x)  = \frac{|2-x|}{k}; x = 0, 1, 2, 3, 4, 5$
	  \end{center}
  
  \begin{enumerate}
    \item
    What is a random variable? \hfill 1
    \item
		Is probability a discrete variable? Explain in brief. \hfill 2
    \item  
	Find the value of k.  \hfill 3
    \item
	Find the probability that the no. of hattricks would be less than the expectation. \hfill 4
  \end{enumerate}
  
   \item
	  \textbf{A fair coin is tossed five times. Number of heads appearing are noted, considering it a discrete random variable.} 
  
  \begin{enumerate}
    \item
	Give a real life example of a discrete random variable. \hfill 1
    \item
	Can discrete variable have infinite number of possible outcomes? \hfill 2
    \item  
	Find the probability distribution from the stem. \hfill 3
    \item
	Construct thedistribution function and hence find $F(X \le 3)$. \hfill 4
  \end{enumerate}
  
  \item
  \textbf{The probability density function of a continuous random variable is}

$$
  f(x) =
\begin{cases}
k(x+1),  & 0 \le x \le 1 \\
0, & otherwise
\end{cases}
$$

  \begin{enumerate}
    \item
	What is a random variable? \hfill 1
    \item
    	Find the value of k \hfill 2
    \item
    	Find the probability that the values of x would lie between 0 and 0.5. \hfill 3
     \item
     	What is the probability that X is greater than 0.8?  \hfill 4
  \end{enumerate}
  
    \item
  \textbf{The probability density function of a continuous random variable is}

$$
  f(x) =
\begin{cases}
kx(x-1),  & 1 \le x \le 4 \\
0, & otherwise
\end{cases}
$$

  \begin{enumerate}
    \item
	What is the range of probability? \hfill 1
    \item
    	Find the value of k \hfill 2
    \item
    	Justify the pdf property of the fucntion. \hfill 3
     \item
     	What is the probability that X is greater than 3?  \hfill 4
  \end{enumerate}


 \item
	  \textbf{The probability distribution of a discrete random variable X is given below:} 

	  \begin{table}[h]
	  \begin{center}
\begin{tabular}{c|cccccc}
x    & -2   & -1 & 0   & 1 & 3 & 4   \\ \hline
P(x) & 0.1 & k & 2k & 3k & 4k & 0.2
\end{tabular} 
\end{center}	
\end{table}
  
  \begin{enumerate}
    \item
	What is $\Sigma P(x)$? \hfill 1
    \item
	Find the value of k. \hfill 2
    \item  
	Find $P(X \geq 0)$ and $P(X < 1)$. \hfill 3
    \item
	Find the cumulative distribution function, F(X) and F(2) and explain. \hfill 4
  \end{enumerate}

 \item
	  \textbf{The joint probability function of two random variables X \& Y is given below:}

\begin{center}
$P(x,y) = \frac{1}{21}(x+y); x=1,2,3$ \& $y = 1,2$ 
\end{center}
  
  \begin{enumerate}
    \item
	What is a probability density function (pdf)? \hfill 1
    \item
	What is P(X=a) in a pdf, where a is an aribitrary number? \hfill 2
    \item  
	Find the marginal probabilities. \hfill 3
    \item
	Find $P(x \vert y), P(x \vert 1)$ and $P(y|4)$ \hfill 4
  \end{enumerate}
  
  \item
	  \textbf{The probability density function of a continuos random variable X is given as:}
	  
	  $$f(x) = \frac 1 {b-a}; a\le x\le b$$
  
  \begin{enumerate}
    \item
	In this distribution, what is P(a)? \hfill 1
    \item
    What is the shape of the distribution? \hfill 2	
    \item
	Find $P( a\le x\le b)$. \hfill 3
      \item
	Find and explain the median of the distribution. \hfill 4
  \end{enumerate}

 \item
  \textbf{The probability density function of a continuous random variable is}

$$
  f(x) =
\begin{cases}
kx^2+kx+ \frac 18,  & 0 \le x \le 2 \\
0, & otherwise
\end{cases}
$$

  \begin{enumerate}
    \item
	What is a continuous random variable? \hfill 1
    \item
    	Find the value of k \hfill 2
    \item
    	Find the probability that the values of x would lie between1 and 3. \hfill 3
     \item
     	Find the 40th percentile of the distribution and explain.  \hfill 4
  \end{enumerate}
  \end{enumerate}

\section{Short Questions}

  \begin{enumerate}
    \item
		What is a continuous random variable?  \hfill 1
    \item
	Question \hfill 1
    \item  
	Question \hfill 1
    \item
	Question \hfill 1
  \end{enumerate}

%---------------------------------------------------------------------------
\chapter{Mathematical Expectation} 
%---------------------------------------------------------------------------

\section{Creative Questions}
  \begin{enumerate}
  
   \item
	  \textbf{The probability distribution of a random X is provided below:} 
	  
	  \begin{table}[h]
	  \centering
\begin{tabular}{c|ccccc}
X & -1 & 0 & 1 & 2 & 3 \\ \hline
P(x) & $\frac 3{20}$ & $\frac 15$ & $\frac 14$ & $\frac 14$ & $\frac 3{20}$
\end{tabular}
\end{table}
  
  \begin{enumerate}
    \item
	What is the expectation of a constant m? \hfill 1
    \item
	Find $E(X).$ \hfill 2
    \item  
	Find $E(Y)$, where $Y = \frac X2$  \hfill 3
    \item
	Find Variance of (2X+3). \hfill 4
  \end{enumerate}
  
        \item \textbf{A random variable is distributed as below:}
        
        \begin{center}
  \textbf{$P(X) = \frac{3-\vert 4-x\vert}{k}; x=2,3,4,5,6$}
  \end{center}

  \begin{enumerate}
    \item
	What is the Expectation equivalent to? \hfill 1
    \item
    	Find the value of k. \hfill 2
    \item
    	Determine the value of the expectation. \hfill 3
     \item
     	Find $V(2X-1)$ \hfill 4
  \end{enumerate}
  
   \item
	  \textbf{The probability distributions of demand of mobile phones of two operating systems (OS) Android (X) and iPhone OS (iOS) (Y) are:} 
	  
	    \begin{table}[h]
	    	  \begin{center}
\begin{tabular}{c|c|c|c|c|c}
Demand & 100  & 200  & 300  & 400  & 500  \\ \hline
P(X)   & 0.1  & 0.4  & m    & 0.15 & 0.1  \\ \hline
P(Y)   & 0.09 & 0.45 & 0.32 & 0.11 & 0.03
\end{tabular}
	  \end{center}
\end{table}
  
  \begin{enumerate}
    \item
	What is Expectation? \hfill 1
    \item
	Can Expectation be negative? \hfill 2
    \item  
	Find m from the table. \hfill 3
    \item
	Which OS has higher demand? Analyze. \hfill 4
  \end{enumerate}

\item
	  \textbf{An umbrella seller earns a revenue of BDT. 5000 if it rains. If it does not rain, he loses BDT. 1000. The probability that it rains on a given day is 0.04.} 
  
  \begin{enumerate}
    \item
	Write down the formula of Expectation for a continuous random variable. \hfill 1
    \item
	Can the value of Expectation be zero? \hfill 2
    \item  
	What is the umbrella seller's expected revenue? \hfill 3
    \item
	What should be the minimum probability of raining for him to achieve revenue greater than zero? \hfill 4
  \end{enumerate}
  
     \item
	  \textbf{A box contains 5 red and 6 white balls. 3 balls are drawn at random. X is the number of white balls drawn.} 
  
  \begin{enumerate}
    \item
	What does variance measure? \hfill 1
    \item
	Can the variance be smaller than standard deviation? \hfill 2
    \item  
	Find the E(X) from the stem. \hfill 3
    \item
	Find the variance from the stem assuming X is the number of red balls drawn. \hfill 4
  \end{enumerate}
  
   \item
	  \textbf{A professor showed a probability distribution in a class:}
	  
	  \begin{table}[h]
	  \begin{center}
\begin{tabular}{llllll}
x    & 1   & 2 & 3   & 4 & 5   \\ \hline
p(x) & 0.1 & a & 0.3 & b & 0.2 
\end{tabular} \\
\textbf{The value of the arithmetic mean of the distribution is 3.}
\end{center}	
\end{table}
    
  \begin{enumerate}
    \item
	What is the formula of expectation? \hfill 1
    \item
	What is the variance of a constant? Explain logically. \hfill 2
    \item  
	What are the values of a \& b? \hfill 3
    \item
	Find and explain the variance of the distribution. \hfill 4
  \end{enumerate}
  
       \item
	  \textbf{X is a random variable having the below functional form:} 
  
  \begin{center}
  $P(X) = \frac{6-|7-x|}{k}; x = 1, 2, \cdots,10$ \\
  \end{center}  
  
  Y is another variable having the relationship y = 3x+5
  
  \begin{enumerate}
    \item
	What is joint probability? \hfill 1
    \item
	What is the minimum possible value of variance? Why? \hfill 2
    \item  
	Find the value of k. \hfill 3
    \item
	Find E(X) and E(Y). Why are they different? \hfill 4
  \end{enumerate}
  
       \item
  \textbf{Various sales and their probabilities of a grocery store is given below}
  
  \begin{table}[h]
  \centering
\begin{tabular}{llllll}
Sales & 200 & 250 & 275 & 310 & 350 \\
Probability & 0.10 & 0.20 & 0.40 & 0.25 & 0.05
\end{tabular}
\end{table}

  \begin{enumerate}
    \item
	Can the expectation of a random variable be negative? \hfill 1
    \item
    	Find the expected sales of the store on a given day. \hfill 2
    \item
    	Compute the dispersion of sales f the store. \hfill 3
     \item
     	To make the expected sale 280, what sale does the store need in place of 200? \hfill 4
  \end{enumerate}
  
     \item
	  \textbf{A survey of Television (TV) users at Gulshan in Dhaka was conducted to find how many sets each family use. The following data were obtained:} 
	  
	  	  \begin{table}[h]
	  \begin{center}
\begin{tabular}{llllll}
No of TV set    & 0 & 1  & 2 & 3    \\ \hline
No of family & 10 & 75 & 10 & 5
\end{tabular}
\end{center}	
\end{table}
  
  \begin{enumerate}
    \item
	What is Expectation equivalent to? \hfill 1
    \item
	Can Variance be negative? Why or why not? \hfill 2
    \item  
	Find the variance of the number of TV sets. \hfill 3
    \item
	Find and compoare between arithmetic mean and expectation. \hfill 4
  \end{enumerate}
  \end{enumerate}
\section{Short Questions}

\chapter{Binomial Distribution} 
\section{Creative Questions}
  \begin{enumerate}
  
     \item
	  \textbf{A farmer selected a paddy field for seed collection. He found out that 10 out of each 25 paddies are damaged. He collected a sample of 15 paddies.} 
  
  \begin{enumerate}
    \item
	What is a Bernoulli trial? \hfill 1
    \item
	IF a Bernoulli trial is repeated n times, in how many ways are outcomes generated? Explain. \hfill 2
    \item  
	Find the probability that at least one paddy is damaged? \hfill 3
    \item
	Comment on the skewness of the data. \hfill 4 
	
		[Hint: For a binomial distribution, $\gamma_1 = \frac{q-p}{\sqrt{npq}}$]
  \end{enumerate}
  
 \item
	  \textbf{A farmer plans to store rice seeds for future use. It was found that 8 out of 20 seeds are rotten. He then collected a sample of 15 seeds.} 
  
  \begin{enumerate}
    \item
	What is Bernoulli trial? \hfill 1
    \item
	How are Bernoulli and Binomial distributions related? \hfill 2
    \item  
	What is the probability that at least one seed is rotten out of 15? \hfill 3
    \item
	What is the probability that the number of rotten seeds is greater than the arithmetic mean? \hfill 4
  \end{enumerate}

 \item
	  \textbf{The number of defective pen produced by a company follows a binomial distribution with expectation 1.5 and variance 1.125.}. 
  
  \begin{enumerate}
    \item
	What is the mean of binomial distribution \hfill 1
    \item
	Can variance be greater than mean in binomial distribution? \hfill 2
    \item  
	Determine the probability function of the number of defective items produced by the company. \hfill 3
    \item
	What is the probability that the number of defective items is no less than 3? \hfill 4
  \end{enumerate}
  
  \end{enumerate}

\section{Short Questions}

\chapter{Poisson Distribution} 
\section{Creative Questions}

\begin{enumerate}

   \item
	  \textbf{Between 1000hrs and 1700 hrs, the average number of phonce calls per minute received by a power distribution company is 2.5. } 
  
  \begin{enumerate}
    \item
	Give an example where Poisson distribution is applicable. \hfill 1
    \item
	Find the relationship between expectationa and standard deviation of Poisson distribution. \hfill 2
    \item  
	Find the probability that the number of calls is between 1 and 3 (inclusive). \hfill 3
    \item
	What is the probability that the number of calls received is greater than the average? \hfill 4
  \end{enumerate}
  
     \item
	  \textbf{The frequency distribution of defective items in packets of key rings is given below.} 
	  
	  \begin{table}[h]
	  \centering
\begin{tabular}{c|c|c|c|c|c|c}
Number of defective items & 0 & 1 & 2 & 3 & 4 & 5 \\ \hline
Number of packets & 76 & 74 & 29 & 17 & 3 & 1
\end{tabular}
\end{table}
  
  \begin{enumerate}
    \item
	What is another way to write $P(X \ge 1)$? \hfill 1
    \item
	Can the mean of Poisson distribution be negative? \hfill 2
    \item  
	From the given stem, find mean and variance. \hfill 3
    \item
	Find the expected frequencies and comment. \hfill 4
  \end{enumerate}
  
 \item
	  \textbf{A can manufacturer observes that 0.1\% of the produced cans are faulty. The cans are packaged in carton boxes, with 500 cans in each box. A wholeseller purchases 100 boxes from the manufacturer.} 
  
  \begin{enumerate}
    \item
	What is shape of Poisson distribution? \hfill 1
    \item
	For a Poisson distribution, P(2) = P (3). What is P(2)? \hfill 2
    \item  
	Find the probability of exactly one defective can. \hfill 3
    \item
	Find the expected number of boxes with no defective cans. \hfill 4
  \end{enumerate}
  
  \item
  \textbf{In winter, the probability that it rains on a particular day is 0.015. An analyst observes \\ 100 winter days.}
 
  \begin{enumerate}
    \item
	What is an experiment? \hfill 1
    \item
    	When can the Poisson distribution be approximated by the Binomial distribution? \hfill 2
    \item
    	Find, using Binomial distribution, the probability that  it would not rain at all on the \\ observed days. \hfill 3
     \item
     	Find the probability in 3(c) using Poisson distribution.  \hfill 4
  \end{enumerate}

  \item
	  \textbf{BTCL receives 2.5 telephone calls on average from 4 pm to 6 pm. The number of calls received is a random variable. } 
  
  \begin{enumerate}
    \item
	When is Poisson variate applicable? \hfill 1
    \item
	Show conversion criteria and method from Binomial to Poisson distribution. \hfill 2
    \item  
	Find the probability of receiving no more than 3 calls. \hfill 3
    \item
	Find the pattern of calls and show on graph paper.  \hfill 4 \\
	Hint: Find probabilities: P(0), P(1), $\cdots$
  \end{enumerate}
  
  \item
	  \textbf{The number of customers coming at a shop per minute follows a Poisson distribution,  whose mean is 3.} 
  
  \begin{enumerate}
    \item
	What is a Poisson variate? \hfill 1
    \item
	Can the mean of Poisson distribution be negative? \hfill 2
    \item  
	Find the probability that the number of customers coming is between 1 and 2. \hfill 3
    \item
	Are the probabilities of coming to 2 and 3 customers equal?  \hfill 4
  \end{enumerate}

 \item
	  \textbf{A random variable is distributed as follows:} 
	  
	  \begin{table}[h]
	  \centering
\begin{tabular}{ccccccc}
Value & 0 & 1 & 2 & 3 & 4 & 5 \\ \hline
Frequency & 70 & 73 & 27 & 15 & 4 & 1
\end{tabular}
\end{table}
  
  \begin{enumerate}
    \item
	What is the mean of Poisson distribution? \hfill 1
    \item
	What is the relationship between mean and standard deviation of Poisson distribution? \hfill 2
    \item  
	Find the mean and variance of the given distribution. \hfill 3
    \item
	Compare the observed and expected frequencies, assuming a Possion distribution. \hfill 4
  \end{enumerate}

  \end{enumerate}

\section{Short Questions}

\chapter{Normal Distribution} 
\section{Creative Questions}


\section{Short Questions}

\chapter{Index Number} 
\section{Creative Questions}


\section{Short Questions}

\chapter{Sampling} 
\section{Creative Questions}


\section{Short Questions}

\chapter{Vital Statistics} 
\section{Creative Questions}

  \begin{enumerate}
  
   \item
	  \textbf{A reseracher uses the following data to know about some demographic characterisics.} 
  
  \begin{enumerate}
    \item
	What is General Fertility Rate? \hfill 1
    \item
	What is the difference between GRR and NRR? \hfill 2
    \item  
	Compute the population density. \hfill 3
    \item
	Are TFR and GRR same for this data? \hfill 4
  \end{enumerate}
  
 \item
  \textbf{For projection of population in a future time period, demographers use simple, \\ geometric or exponential growth technique. Each method has its advantages and \\ disadvantages.}

  \begin{enumerate}
    \item
	What is geometric growth? \hfill 1
    \item
    	In geometric growth method, obtain the formula for time required for the population to get \\ doubled [denote rate as r]. \hfill 2
    \item
    	In exponential method, how much unit of time is required for the population to get tripled?  \hfill 3
     \item
     	For projecting (predicting future values), is geometric growth method better than the \\ exponential method? Justify.  \hfill 4
  \end{enumerate}
  
     \item
	  \textbf{Population of Dhaka and Sylhet by different age groups and areas are given below:} 
	  
	  \begin{table}[h]
	  \begin{center}
\begin{tabular}{lllll}
Division & \multicolumn{3}{c}{Age}         & Area ($km^2$) \\
         & 0-14      & 15-64    & 65+      &               \\
Dhaka    & 10,000,00 & 5,00,000 & 5,80,000 & 1,880         \\
Sylhet   & 7,00,000  & 2,70,000 & 4,70,000 & 2,319        
\end{tabular}
\end{center}
\end{table}
  
  \begin{enumerate}
    \item
	Write down the formula of dependency ratio. \hfill 1
    \item
	What is meant by NRR = 0.983? \hfill 2
    \item  
	Find and compare between the dependency ratios of the cities. \hfill 3
    \item
	Based on data, which city is more comfortable for living? \hfill 4
  \end{enumerate}
  
   \item
	  \textbf{As part of an analysis, a researcher collected data on women and live births.} 
	  \begin{table}[h]
	  \centering
\begin{tabular}{c|c|c|c|c|c|c|c}
Age & 15-19 & 20-24 & 25-29 & 30-34 & 35-39 & 40-44 & 45-49 \\ \hline
No. of Women & 540 & 760 & 530 & 495 & 450 & 505 & 430 \\ \hline
No. of live births & 109 & 198 & 86 & 90 & 65 & 76 & 60
\end{tabular}
\end{table}
  
  \begin{enumerate}
    \item
	What is the formula of death rate? \hfill 1
    \item
	Write down the uses of vital statistics. \hfill 2
    \item  
	Find teh Age Specific Birth Rates (ASFR). \hfill 3
    \item
	Find the GFR and compare its concept and value with ASFRs. \hfill 4
  \end{enumerate}
  
  \end{enumerate}

\section{Short Questions}

\backmatter
\chapter{Conclusion}
\lipsum[8]

\tableofcontents
\end{document}