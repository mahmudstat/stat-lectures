\documentclass[a4paper,oneside, margin=1.4in]{book}
\usepackage{lipsum}
\usepackage[hidelinks]{hyperref}
\usepackage{titletoc}
\usepackage{amsmath}
\usepackage{geometry}
\usepackage{graphicx}
\graphicspath{ {.} }
\geometry{a4paper, margin=1in}
\titlecontents*{chapter}
  [0pt]% <left>
  {}
  {\chaptername\ \thecontentslabel\quad}
  {}
  {\bfseries\hfill\contentspage}    

\usepackage{bookmark}
\usepackage{etoolbox}

\makeatletter
\newcommand*{\AddChapterPrefixInBookmarks}{%
  \if@mainmatter
    \ifnum\bookmarkget{level}=0 %
      \preto\bookmark@text{\@chapapp\space}%
    \fi
  \fi
}
\makeatother

\bookmarksetup{
  numbered,
  addtohook=\AddChapterPrefixInBookmarks,
}

% Workaround for numbered sections in unnumbered
% chapter "Introduction" to avoid chapter number
% zero.
\renewcommand*{\thesection}{%
  \ifcase\value{chapter}%
  \else
    \thechapter.%
  \fi
  \arabic{section}%
}

\title{My document}

\begin{document}
\frontmatter

\begin{titlepage}
    \begin{center}
        \vspace*{1cm}
            
        \Huge
        \textbf{Statistics Question Bank}
            
        \vspace{0.5cm}
        \huge
        Second Paper
            
        \vspace{1.5cm}
            
        \textbf{Abdullah Al Mahmud}

     \vspace{1.5cm}

	\Large 
	Updated on: \today
            
        \vfill
            

            
        \vspace{0.8cm}
            
\includegraphics[width=1cm]{logo.png}
            
        \Large
        www.statmania.info\\
            
    \end{center}
\end{titlepage}


\tableofcontents


\mainmatter
\chapter{Probability} 
\section{Creative Questions}

\begin{enumerate}

     \item
	  \textbf{Events that do not depend on each other are called independent 
	  events, and events that cannot occurr simulataneously are called disjoint events.} 
  
  \begin{enumerate}
    \item
	Provide an example of disjoint events, using the set theory. \hfill 1
    \item
	Prove that $P(A\cap \bar B) = P(A) - P(A\cap B)$ \hfill 2
    \item  
	If there are k mutually and exhaustive events, prove 
	$\displaystyle \sum_{i=1}^k P(A_i) = 1$ \hfill 3
    \item
	Prove that two events cannot be simulataneously independent and mutually 
	exclusive. \hfill 4
  \end{enumerate}

 \item
	  \textbf{A quality control analyst in an industry tracks the no. 
	  of defective items produced per day. He observes 150 successive days 
	  and then prepares a table.} 
	  
	  \begin{table}[h]
	  \centering
\begin{tabular}{c|c|c|c|c|c} \hline
No. of items & 0 & 1 & 2 & 3 & 4 \\ \hline
Frequency & 30 & 32 & 40 & 28 & 20 \\ \hline
\end{tabular}
\end{table}
  
  \begin{enumerate}
    \item
	What is the formula of classical probability? \hfill 1
    \item
	Explain the difference between Priori Approach and Empirical Approach 
	of probability. \hfill 2
    \item  
	What is the probability that less than 2 defective items would be produced 
	on a particular day? \hfill 3
    \item
	Explain the relationship between independency and mutual excluvity in the 
	light of the stem. \hfill 4
  \end{enumerate}
  
     \item
	  \textbf{Ratul and Tomal both have an unbiased die. Both have randomly thrown
	  their die once. } 
  
  \begin{enumerate}
    \item
	What are equally likely events? \hfill 1
    \item
	If a die is thrown once, what is the probability of getting a prime 
	number? \hfill 2
    \item  
	From the stem, what is the probability that the sum of numbers appearing
	\\ on the dice is greater than 6. \hfill 3
    \item
	Examine: the probabilities of getting the sum less than 6 and greater 
	than 6 are equal. \hfill 4
  \end{enumerate}

  
   \item
	  \textbf{It is observed that 50\% of mails are spam. A software 
	  filters spam mail before reaching the inbox. Its accuracy for detecting a spam mail is 99\% and chances of tagging a non-spam mail as spam mail is 5\%.} 
  
  \begin{enumerate}
    \item
	What is a disjoint event? \hfill 1
    \item
	For two independent events, what does the Bayes' theorem reduce to? \hfill 2
    \item  
	What is the probability that a mail is tagged as spam?  \hfill 3
    \item
	If a certain mail is tagged as spam, find the probability that it is not a spam mail. \hfill 4
  \end{enumerate}
  
  \item
  \textbf{A company receives 60\% of its job applications from applicants with
  the required qualifications. A hiring software screens applications for minimum qualifications. It correctly identifies qualified applications 97\% of the time, but it also incorrectly marks 4\% of unqualified applications as qualified.}
  
  \begin{enumerate}
   \item  
    What is the probability that an application is marked as qualified? \hfill 3
   \item
    If an application is marked as qualified, find the probability that it actually does not meet the required qualifications. \hfill 4
  \end{enumerate}

\item
  \textbf{In a survey of a town's population of 500 people, it was found that 150 people read the local newspaper daily, 200 people listen to the radio daily, and 80 people do both.}
 
  \begin{enumerate}
    \item
    	What is the probability that a randomly selected person reads the newspaper given that they listen to the radio? \hfill 3
    \item
    	Calculate the probability that a randomly selected person neither reads the newspaper nor listens to the radio. \hfill 4
  \end{enumerate}

\item
  \textbf{In a school with 200 students, 60 students participate in the science 
  club, 80 participate in the math club, and 30 participate in both.}
 
  \begin{enumerate}
    \item
    	What is the probability that a randomly selected student participates in both clubs? \hfill 3
    \item
    	If a student is chosen at random, what is the probability that they are in exactly one of the clubs? \hfill 4
  \end{enumerate}

\item
  \textbf{In a community of 300 residents, it was found that 90 people use public transportation regularly, 120 use bicycles, and 40 use both.}
 
  \begin{enumerate}
    \item
    	What is the probability that a randomly selected resident uses either public transportation or bicycles? \hfill 3
    \item
    	What is the conditional probability that a resident uses public transportation given that they use a bicycle? \hfill 4
  \end{enumerate}

  
  \item
	  \textbf{A dope test correctly identifies a drug user as positive 90\%  of the time, but incorrectly identifes 20\% non-users as users. The probability of drug use is 0.05.} 
  
  \begin{enumerate}
    \item
	Write down the formula of conditional probability. \hfill 1
    \item
	Express $P(A|B)$ in terms of $P(B|A)$. \hfill 2
    \item  
	Find the probability of testing positive in the test. \hfill 3
    \item
	If the test shows a user positive, what is the probability that the person is actually a user? \hfill 4
  \end{enumerate}

 \item
	  \textbf{A red and a blue dice are thrown once. The dice are absolutely neutral and independent.} 
  
  \begin{enumerate}
    \item
	What is a simple event? \hfill 1
    \item
	Give an example of a certain event using set theory. \hfill 2
    \item  
	Find the probability that the difference of two digits from two dices is less than 3.\hfill 3
    \item
	Are the probabilities of getting greater digit from the blue die and that from the red die equal? Justify. \hfill 4
  \end{enumerate}
  
  \item
	  \textbf{An unbiased coin is tossed 10 times.} 
  
  \begin{enumerate}
    \item
	If a coin is flung 3 times, how many outcomes are generated? \hfill 1
    \item
	If a coin is flung n times, show how many outcomes are generated. \hfill 2
    \item  
	What is the probability of getting a) at least 3 heads, b) at most 3 heads? \hfill 3
    \item
	Are these probabilities equal? a) Getting at least 2 heads \& b) Getting at least 2 tails. \\ Also justify logically. \hfill 4
  \end{enumerate}
  
  \item
  \textbf{It is observed that in a college, there are 100 students, of whom 30 play football, 40 play cricket, and 20 play both.}
 
  \begin{enumerate}
    \item
	What is the range of probability? \hfill 1
    \item
    	What is the relationship between independence and mutual excluvity?  \hfill 2
    \item
    	Are the probabilities of playing cricket and that of football independent? Prove. \hfill 3
     \item
     	If a student is selected randomly, and if he does not play cricket, what is the probability that \\ he plays football? \hfill 4
  \end{enumerate}

 \item
	  \textbf{A box contains four blue and 6 green balls. 3 balls are drawn randomly.} 
  
  \begin{enumerate}
    \item
	What is the value of $^nC_r$? \hfill 1
    \item
	Illustrate the difference between permutation and combination with an example. \hfill 2
    \item  
	What is the probability that all balls are green? \hfill 3
    \item
	What is the probabilith that one ball has a different color? \hfill 4
  \end{enumerate}
  
  \item
	  \textbf{A jar contains 5 red marbles and 7 yellow marbles. 
	  Three marbles are drawn at random.} 
  
  \begin{enumerate}
    \item  
	What is the probability that all marbles are yellow? \hfill 3
    \item
	What is the probability that a marble has a different color? \hfill 4
  \end{enumerate}


\item
	  \textbf{Sadman has an urn with 5 red and 4 white balls. He has randomly drawn two balls from the urn.} 
  
  \begin{enumerate}
    \item
	What is the probability of an uncertain event? \hfill 1
    \item
	Write the third axiom of probability. \hfill 2
    \item  
	What is the probability that both the balls drawn by Sadman are white? \hfill 3
    \item
	Are the probabilities of both balls being same color and different color equal? Analyze. \hfill 4
  \end{enumerate}

 \item
	  \textbf{Two dice are thrown together. The dice are named A and B.} 
  
  \begin{enumerate}
    \item
	What is P(A=7)? \hfill 1
    \item
	Create the sample space. \hfill 2
    \item  
	What is the probability that the outcomes of A \& B are different? \hfill 3
    \item
	Determine the probability that the summation of outcome of two dice is a prime number. \hfill 4
  \end{enumerate}

 \item
	  \textbf{A magician draws two cards from a pack (i) with replacement and then (ii) without replacement. The cards were well-shuffled before drawing.} 
  
  \begin{enumerate}
    \item
	What is the probability of an impossible event? \hfill 1
    \item
	How to determine the probability of a joint event?  \hfill 2
    \item  
	As per (i), what is the probability that the cards have different color? \hfill 3
    \item
	As per (ii), what is the probability that the cardsare aces of same color?  \hfill 4
  \end{enumerate}

 \item
	  \textbf{$P(A) = \frac{3}{10}, P(B) = \frac 25, P(B\cup A) = \frac12$} 
  
  \begin{enumerate}
    \item  
	Find $P(A \vert B)$ and $P(B \vert A)$ \hfill 3
    \item
	Verify the equality mathematically \& empirically: $P(B) = P(A) \cdot P(B \vert A) + P(\bar A) \cdot P(B \vert \bar A)$ \hfill 4
  \end{enumerate}
  
   \item
  \textbf{$P(A\vert B) = \frac 1 8, P(A) = \frac 12, P(B) = \frac 15$}
 
  \begin{enumerate}
    \item
    \item
    	Find $P(A\cap B)$.  \hfill 2
    \item
    	Find $P(A\vert \bar B)$. \hfill 3
     \item
     	Are the probabilities $P(A\vert B)$ and $P(B\vert A)$ equal? Justify\hfill 4
  \end{enumerate}
  
  \item \textbf{\( P(B \vert A) = \frac{1}{4}, P(B) = \frac{1}{2}, P(A) = \frac{1}{3} \)}

\begin{enumerate}
    \item Find \( P(B \vert \bar{A}) \). \hfill 3
    \item Find $P(\bar A \cap \bar B)$ \hfill 4
\end{enumerate}

\item \textbf{\( P(C \vert D) = \frac{2}{5}, P(C) = \frac{3}{4}, P(D) = \frac{1}{2} \)}

\begin{enumerate}
    \item Find \( P(C \vert \bar{D}) \). \hfill 3
    \item Examine the following statements: \\ i) A and B are independent and \\
    ii) A and B are mutually exclusive. \hfill 4
\end{enumerate}

\item \textbf{\( P(X \vert Y) = \frac{1}{6}, P(X \cap Y) = \frac{1}{10}, P(Y) = \frac{3}{4} \)}

\begin{enumerate}
    \item Find \( P(X \vert \bar{Y}) \). \hfill 3
    \item Are $P(\bar X)$ and $P(Y)$ independent? \hfill 4
\end{enumerate}

\item \textbf{\( P(M \vert N) = \frac{2}{9}, P(M \cup N) = \frac{5}{7}, P(N) = \frac{4}{7} \)}

\begin{enumerate}
    \item Calculate \( P(M \cap \bar{N}) \). \hfill 3
    \item Examine whether   \hfill 4
    
    i. $P(M \vert N) = P(N \vert M)$ \\
    ii. $P(M \cap \bar N) = P(\bar M \cap N)$

\end{enumerate}

\item \textbf{\( P(G \vert H) = \frac{3}{8}, P(G \cap H) = \frac{1}{6}, P(H) = \frac{2}{5} \)}

\begin{enumerate}
    \item Find \( P(G \cup \bar{H}) \). \hfill 3
	\item Verify the equality mathematically \& empirically: \\ $P(G) = P(H) \cdot P(G \vert H) + P(\bar H) \cdot P(G \vert \bar H)$ \hfill 4
\end{enumerate}

\item \textbf{\( P(T \vert U) = \frac{5}{12}, P(T \cup U) = \frac{7}{10}, P(U) = \frac{1}{2} \)}

\begin{enumerate}
    \item Determine \( P(T \cap \bar{U}) \). \hfill 3
    \item What conditions must hold for \( T \) and \( U \) to be mutually exclusive? \hfill 4
\end{enumerate}



 \item
  \textbf{Sakib has recently graduated from the University of Dhaka. he applies to two firms - EduCube \& Digic- for a Data Analyst job. The probability of hiring by EduCube is 0.8 and by Digic is 0.4. The probability that none hires is 0.5.} 
  
  \begin{enumerate}
    \item
	What is a sample space? \hfill 1
    \item
	Explain how to find $P(\bar A \cap B)$ using Venn Diagram. \hfill 2
    \item  
	Find the probability of hiirng by by Digic but not by EduCube. \hfill 3
    \item
	Find the probability that no firm will reject him. \hfill 4
  \end{enumerate}

 \item
	  \textbf{Recently there is an increase in the number of electronic medias in Bangladesh. A professor stated in the class room that very few people now resort to print media for news. A research indicates 70\% people collect news from electronic media, 60\% from print media, and 50\% from both.} 
  
  \begin{enumerate}
    \item
	What is an impossible event? \hfill 1
    \item
	Write the event "None of the two occurs" in two different notations. \hfill 2
    \item  
	What is the probability of getting news from at most one type of media? \hfill 3
    \item
	Is the professor correct in his/her statement? Analyze. \hfill 4
  \end{enumerate}
  


  \end{enumerate}

\section{Short Questions}

  \begin{enumerate}
    \item What is a trial in the context of probability? \hfill 1

\item What is an experiment in probability. \hfill 1

\item What is a sample space? \hfill 1

\item What is a sample point in probability? \hfill 1

\item Explain what an event is in probability. \hfill 1

\item What is a simple event? \hfill 1

\item Define a compound event. \hfill 1

\item What is an impossible event? \hfill 1

\item What is a certain event? \hfill 1

\item Describe an uncertain event in probability. \hfill 1

\item What does it mean when events are mutually exclusive? \hfill 2

\item What is a complementary event? \hfill 1

\item What are equally likely events. \hfill 1

\item What is the difference between a permutation and a combination? \hfill 2

\item In how many different ways can 5 books be arranged on a shelf? \hfill 2

\item In how many ways can a committee of 3 people be selected from a group of 8 people? \hfill 2

\item If there are 4 different letters, how many unique 2-letter permutations can be formed? \hfill 1

\item Calculate the number of 3-letter combinations that can be formed from 7 different letters. \hfill 2

\item In how many ways can a president and a vice president be chosen from a group of 10 candidates? \hfill 2

\item How many different 4-digit passwords can be created using the digits 1 to 9 if repetition is not allowed? \hfill 2

\item What is the number of ways to arrange the letters in the word “APPLE”? \hfill 1

\item If 10 people are at a meeting, how many ways can 2 people be chosen to speak? \hfill 2

\item How many different teams of 4 players can be formed from a group of 12 players? \hfill 2

\item What is the formula for calculating the number of permutations of n objects taken r at a time? \hfill 1

\item How many different 3-digit combinations can be formed using the digits 2, 4, 6, 8, and 9 if repetition is allowed? \hfill 2

\item In how many ways can 6 people be seated in a row? \hfill 2

\item If a deck of cards is shuffled, in how many ways can 5 cards be selected from the deck? \hfill 2

\item What is the value of 5!? \hfill 1

\item Expand ${}^nP_r$

\item Expand ${}^nC_r$

\item What is the classical definition of probability? \hfill 1

\item What is the range of probability? \hfill 1

\item Briefly explain empirical probability with an example. \hfill 2

\item How does the classical definition of probability differ from the empirical definition? \hfill 2

\item Which definition of probability does this formula belong to: 

\[
P(E) = \lim_{n \to \infty} \frac{\text{n(A)}}{n(S)}
\]

\item What are the three axioms of probability in the axiomatic approach? \hfill 2

\item In the axiomatic approach, if \( P(S) = 1 \), where \( S \) is the sample space, what does this imply? \hfill 1

\item How does the axiomatic approach define the probability of the union of two mutually exclusive events \( A \) and \( B \)? \hfill 2

\item What is the third axiom of probability? \hfill 1

\item In the third axiom, what is the value of $\displaystyle \sum_{i=1}^n P(A_ii)$

\item What does it mean when \( P(A) = 0 \) in probability theory? \hfill 1

\item When \( P(A) = 1 \), what does this signify about the event \( A \)? \hfill 1

\item What is the formula for conditional probability \( P(A|B) \)? \hfill 1

\item What is the value of $P(A \cap B)$ if two events \( A \) and \( B \) are independent? \hfill 1

\item If events \( A \) and \( B \) are independent, what is the value of \( P(A|B) \)? \hfill 1

\item What is an independent event? \hfill 1

\item What is the additive law of probability for two events \( A \) and \( B \)? \hfill 1

\item How does the additive law of probability apply when events \( A \) and \( B \) are mutually exclusive? \hfill 2

\item What is the additive law of probability for \( n \) events, and how is it expressed mathematically? \hfill 2

\item What is the multiplicative law of probability for two events \( A \) and \( B \)? \hfill 1

\item How does the multiplicative law of probability apply when events \( A \) and \( B \) are independent? \hfill 2

\item If two events \( A \) and \( A^c) \) are complementary, what is the relationship between them? \hfill 1

\item Prove using Venn diagram $P(A\cap \bar B) = P(A) - P(A\cap B)$ \hfill 2

 \item 	What is the relationship between independency and mutual excluvity? \hfill 2

\item What is the range of values that probability can take for any event? \hfill 1

\item Why is the probability of any event always between 0 and 1, inclusive? \hfill 2

\item Can the value of probability be 1.2? \hfill 1

\item Can the value of probability be -0.2? \hfill 1

\item How can the expression \( P(A \cap B) \) be expanded in terms of conditional probability? \hfill 2

\item Expand \( P(A' \cap B) \) \hfill 2

\item Expand \( P(A \cap B') \) \hfill 2

\item Can two events be independent and mutually exclusive at once? \hfill 2

\item What is the probability of getting a head on a fair coin toss? \hfill 1

\item If a fair die is rolled, what is the probability of getting a number greater than 4? \hfill 1

\item If a coin is tossed 4 times, how many outcomes are generated? \hfill 1

\item If a die is thrown 3 times, how many outcomes are generated? \hfill 1

\item If 2 coins and a die are thrown together, how many outcomes are generated? \hfill 1

\item Is there any difference between tossin a coin thrice and tossing 3 coins together \hfill 2

\item Write down the formula of $P(\bar A | \bar B)$ \hfill 1

\item Write down and expand the formula of $P(\bar A | B)$ \hfill 2


  \end{enumerate}

%---------------------------------------------------------------------------------------
\chapter{Random Variable and Probability Function} 
%---------------------------------------------------------------------------------------

\section{Creative Questions}

  \begin{enumerate}
  
 \item
	  \textbf{A deck of 52 card is well-shuffled and three cards are drawn from them at random. The number of kings obtained is denoted by x.} 
  
  \begin{enumerate}
    \item
	What are equaly likely events? \hfill 1
    \item
	Differentiate between with replacement and without replacement drawings. \hfill 2
    \item  
	Form the probability fucntion using the above information and then form the distribution.\hfill 3
    \item
	Examine the statement: $P(1 \le x \le 3) = F(3)-F(1)$ \hfill 4
  \end{enumerate}
  
\begin{enumerate}

  \item
  \textbf{The joint probability function of two random variables \( X \) and \( Y \) is given by:}
  
  \begin{center}
  \( \displaystyle P(X,Y) = \frac{x + 2y}{28}; \quad x = 0, 1; \quad y = 0, 1, 2, 3 \)
  \end{center}
 
  \begin{enumerate}
    \item
    	Write down the formula for conditional probability. \hfill 1
    \item
    	What is the relationship between marginal and joint probability? \hfill 2
    \item
    	Find \( P(X) \). \hfill 3
    \item
     	Find \( P(X \vert Y) \) and \( P(X \vert Y = 0) \). \hfill 4
  \end{enumerate}

  \item
  \textbf{The joint probability function of two random variables \( X \) and \( Y \) is described by:}
  
  \begin{center}
  \( \displaystyle P(X,Y) = \frac{2x + 3y}{45}; \quad x = 0, 1, 2; \quad y = 0, 1, 2 \)
  \end{center}
 
  \begin{enumerate}
    \item
    	Write down the formula for conditional probability. \hfill 1
    \item
    	What is the relationship between marginal and joint probability? \hfill 2
    \item
    	Find \( P(X) \). \hfill 3
    \item
     	Find \( P(X \vert Y) \) and \( P(X \vert Y = 0) \). \hfill 4
  \end{enumerate}

\end{enumerate}

  
  \item
  \textbf{The joint probability function of two random variables \( X \) and \( Y \) is given by:}
  
  \begin{center}
  \( \displaystyle P(X,Y) = \frac{x + y + 1}{42}; \quad x = 0, 1, 2; \quad y = 0, 1, 2, 3 \)
  \end{center}
 
  \begin{enumerate}
    \item
    	Calculate the marginal probability \( P(Y) \). \hfill 3
    \item
     	Determine \( P(Y \vert X = 1) \) and \( P(Y \vert X = 0) \). \hfill 4
  \end{enumerate}

  \item
  \textbf{The joint probability function of two random variables \( X \) and \( Y \) is described by:}
  
  \begin{center}
  \( \displaystyle P(X,Y) = \frac{2x + y + 1}{52}; \quad x = 1, 2; \quad y = 1, 2, 3, 4 \)
  \end{center}
 
  \begin{enumerate}
    \item
    	Find the marginal distribution \( P(X) \). \hfill 3
    \item
     	Compute \( P(Y \vert X) \) for \( X = 2 \). \hfill 4
  \end{enumerate}
  
  \item
  \textbf{The joint probability function of two random variables \( X \) and \( Y \) is given by:}
  
  \begin{center}
  \( \displaystyle P(X,Y) = \frac{3x + y}{48}; \quad x = 1, 2; \quad y = 0, 1, 2, 3 \)
  \end{center}
 
  \begin{enumerate}
    \item
    	Find \( P(X) \). \hfill 3
    \item
     	Calculate \( P(X \vert Y) \) and \( P(X \vert Y=1) \). \hfill 4
  \end{enumerate}



   \item
	  \textbf{The probability distributions of a random variable X in two different cases are given below:} 
	  
	  \begin{table}[!ht]
	  \caption {\textbf{Distribution - A}}
	   \begin{center}
\begin{tabular}{llllllll}
x    & 0    & 1    & 2    & 3 & 4    & 5    & 6    \\  \hline
P(x) & 0.20 & 0.10 & 0.08 & w & 0.02 & 0.10 & 0.30
\end{tabular}
 \end{center}
\end{table}

\begin{table}[h]
	  \caption {\textbf{Distribution - B}}
	   \begin{center}
\begin{tabular}{llllll}
x    & 0    & 1    & 2    & 3    & 4 \\ \hline
P(x) & 0.20 & 0.10 & 0.30 & 0.50 & 0.20   
\end{tabular}
\end{center}
\end{table}
  
  \begin{enumerate}
    \item
	What is a probability mass function? \hfill 1
    \item
	Can we dtermine the probability of a certain value of a discrete random variable? \hfill 2
    \item  
	What is the value of w? \hfill 3
    \item
	Which table is a proper probability distribution? Justify with mathematical reasoning. \hfill 4
  \end{enumerate}
  
   \item
	  \textbf{A continuos random variable X follows the following probability density function (pdf).} 
	  \begin{center}
	  $f(x) = 6x(1-x); 0\le x\le 1$
  \end{center}
  
  \begin{enumerate}
    \item
	Give an example of a continous random variable. \hfill 1
    \item
	Examine whether the given function is a pdf. \hfill 2
    \item  
	If $P(X>a) = P(X<a)$, find the value of a. \hfill 3
    \item
	Should $P(0.5 \le X \le 1)$  be equal to 0.5? \hfill 4
  \end{enumerate}
  
     \item
	  \textbf{The probability mass function (pmf) of a football striker scoring no. of hattricks during the course of a league season is given below}
	  
	  \begin{center}
	  $\displaystyle P(x)  = \frac{|2-x|}{k}; x = 0, 1, 2, 3, 4, 5$
	  \end{center}
  
  \begin{enumerate}
    \item
    What is a random variable? \hfill 1
    \item
		Is probability a discrete variable? Explain in brief. \hfill 2
    \item  
	Find the value of k.  \hfill 3
    \item
	Find the probability that the no. of hattricks would be less than the expectation. \hfill 4
  \end{enumerate}
  
   \item
	  \textbf{A fair coin is tossed five times. Number of heads appearing are noted, considering it a discrete random variable.} 
  
  \begin{enumerate}
    \item
	Give a real life example of a discrete random variable. \hfill 1
    \item
	Can discrete variable have infinite number of possible outcomes? \hfill 2
    \item  
	Find the probability distribution from the stem. \hfill 3
    \item
	Construct thedistribution function and hence find $F(X \le 3)$. \hfill 4
  \end{enumerate}
  
  \item
  \textbf{The probability density function of a continuous random variable is}

$$
  f(x) =
\begin{cases}
k(x+1),  & 0 \le x \le 1 \\
0, & otherwise
\end{cases}
$$

  \begin{enumerate}
    \item
	What is a random variable? \hfill 1
    \item
    	Find the value of k \hfill 2
    \item
    	Find the probability that the values of x would lie between 0 and 0.5. \hfill 3
     \item
     	What is the probability that X is greater than 0.8?  \hfill 4
  \end{enumerate}
  
    \item
  \textbf{The probability density function of a continuous random variable is}

$$
  f(x) =
\begin{cases}
kx(x-1),  & 1 \le x \le 4 \\
0, & otherwise
\end{cases}
$$

  \begin{enumerate}
    \item
	What is the range of probability? \hfill 1
    \item
    	Find the value of k \hfill 2
    \item
    	Justify the pdf property of the fucntion. \hfill 3
     \item
     	What is the probability that X is greater than 3?  \hfill 4
  \end{enumerate}


 \item
	  \textbf{The probability distribution of a discrete random variable X is given below:} 

	  \begin{table}[h]
	  \begin{center}
\begin{tabular}{c|cccccc}
x    & -2   & -1 & 0   & 1 & 3 & 4   \\ \hline
P(x) & 0.1 & k & 2k & 3k & 4k & 0.2
\end{tabular} 
\end{center}	
\end{table}
  
  \begin{enumerate}
    \item
	What is $\Sigma P(x)$? \hfill 1
    \item
	Find the value of k. \hfill 2
    \item  
	Find $P(X \geq 0)$ and $P(X < 1)$. \hfill 3
    \item
	Find the cumulative distribution function, F(X) and F(2) and explain. \hfill 4
  \end{enumerate}

 \item
	  \textbf{The joint probability function of two random variables X \& Y is given below:}

\begin{center}
$P(x,y) = \frac{1}{21}(x+y); x=1,2,3$ \& $y = 1,2$ 
\end{center}
  
  \begin{enumerate}
    \item
	What is a probability density function (pdf)? \hfill 1
    \item
	What is P(X=a) in a pdf, where a is an aribitrary number? \hfill 2
    \item  
	Find the marginal probabilities. \hfill 3
    \item
	Find $P(x \vert y), P(x \vert 1)$ and $P(y|4)$ \hfill 4
  \end{enumerate}
  
  \item
	  \textbf{The probability density function of a continuos random variable X is given as:}
	  
	  $$f(x) = \frac 1 {b-a}; a\le x\le b$$
  
  \begin{enumerate}
    \item
	In this distribution, what is P(a)? \hfill 1
    \item
    What is the shape of the distribution? \hfill 2	
    \item
	Find $P( a\le x\le b)$. \hfill 3
      \item
	Find and explain the median of the distribution. \hfill 4
  \end{enumerate}

 \item
  \textbf{The probability density function of a continuous random variable is}

$$
  f(x) =
\begin{cases}
kx^2+kx+ \frac 18,  & 0 \le x \le 2 \\
0, & otherwise
\end{cases}
$$

  \begin{enumerate}
    \item
	What is a continuous random variable? \hfill 1
    \item
    	Find the value of k \hfill 2
    \item
    	Find the probability that the values of x would lie between1 and 3. \hfill 3
     \item
     	Find the 40th percentile of the distribution and explain.  \hfill 4
  \end{enumerate}
  \end{enumerate}

\section{Short Questions}

  \begin{enumerate}

\item What is a discrete random variable? \hfill 1

\item What is a continuous random variable? \hfill 1

\item Give an example of a continuous random variable \hfill 1

\item Is the number of cars passing through a toll booth in an hour an example of a discrete or continuous random variable? \hfill 1

\item Is the temperature in a city measured every hour an example of a discrete or continuous random variable? \hfill 1

\item Is the number of students in a classroom an example of a discrete or continuous random variable? \hfill 1

\item Is the amount of time it takes for a light bulb to burn out an example of a discrete or continuous random variable? \hfill 1

\item Is the number of emails received in a day an example of a discrete or continuous random variable? \hfill 1

\item Is the weight of a person an example of a discrete or continuous random variable? \hfill 1

\item What is the integral of \( x^n \), where \( n \neq -1 \)? \hfill 1

\item Compute the integral of \( x^3 \) with respect to \( x \). \hfill 1

\item Find the integral of \( x^5 \) with respect to \( x \). \hfill 1

\item Compute the definite integral of \( x^2 \) from 0 to 3. \hfill 2

\item Find the value of the definite integral \( \int_1^4 x^4 \, dx \). \hfill 2

\item What is the property of a probability distribution regarding the sum of all probabilities? \hfill 1

\item What are the required properties of a probability distribution? \hfill 2

\item What is the formula of cumulative distribution function for a discrete variable? \hfill 1

\item What is the formula of cumulative distribution function for a continuous variable \hfill 1

\item If $a<b, F(b) - F(a) = ?, where F$ is cumulative distribution function? \hfill 2

\item How can you find $f(x)$ from $F(x)$ for a continuous distribution? \hfill 1

\item How can you find $f(x)$ from $F(x)$ for a discrete distribution? \hfill 1

  \item How can calculate $P(X>3)$ using the concept of complementary probability?  \hfill 1




  \end{enumerate}

%---------------------------------------------------------------------------
\chapter{Mathematical Expectation} 
%---------------------------------------------------------------------------

\section{Creative Questions}
  \begin{enumerate}
  
   \item
	  \textbf{The probability distribution of a random X is provided below:} 
	  
	  \begin{table}[h]
	  \centering
\begin{tabular}{c|ccccc}
X & -1 & 0 & 1 & 2 & 3 \\ \hline
P(x) & $\frac 3{20}$ & $\frac 15$ & $\frac 14$ & $\frac 14$ & $\frac 3{20}$
\end{tabular}
\end{table}
  
  \begin{enumerate}
    \item
	What is the expectation of a constant m? \hfill 1
    \item
	Find $E(X).$ \hfill 2
    \item  
	Find $E(Y)$, where $Y = \frac X2$  \hfill 3
    \item
	Find Variance of (2X+3). \hfill 4
  \end{enumerate}
  
        \item \textbf{A random variable is distributed as below:}
        
        \begin{center}
  \textbf{$P(X) = \frac{3-\vert 4-x\vert}{k}; x=2,3,4,5,6$}
  \end{center}

  \begin{enumerate}
    \item
	What is the Expectation equivalent to? \hfill 1
    \item
    	Find the value of k. \hfill 2
    \item
    	Determine the value of the expectation. \hfill 3
     \item
     	Find $V(2X-1)$ \hfill 4
  \end{enumerate}
  
   \item
	  \textbf{The probability distributions of demand of mobile phones of two 
	  operating systems (OS) Android (X) and iPhone OS (iOS) (Y) are:} 
	  
	    \begin{table}[h]
	    	  \begin{center}
\begin{tabular}{c|c|c|c|c|c}
Demand & 100  & 200  & 300  & 400  & 500  \\ \hline
P(X)   & 0.1  & 0.4  & m    & 0.15 & 0.1  \\ \hline
P(Y)   & 0.09 & 0.45 & 0.32 & 0.11 & 0.03
\end{tabular}
	  \end{center}
\end{table}
  
  \begin{enumerate}
    \item
	What is Expectation? \hfill 1
    \item
	Can Expectation be negative? \hfill 2
    \item  
	Find m from the table. \hfill 3
    \item
	Which OS has higher demand? Analyze. \hfill 4
  \end{enumerate}
  
  \item
	  \textbf{The probability distributions of daily sales of two popular coffee brands, Brand A (X) and Brand B (Y), are:} 
	  
	    \begin{table}[h]
	    	  \begin{center}
\begin{tabular}{c|c|c|c|c|c}
Sales (cups) & 50   & 100  & 150  & 200  & 250  \\ \hline
P(X)         & 0.05 & 0.3  & p    & 0.25 & 0.1  \\ \hline
P(Y)         & 0.1  & 0.35 & 0.3  & 0.2  & 0.05
\end{tabular}
	  \end{center}
\end{table}
  
  \begin{enumerate}
       \item  
	Find \( p \) from the table. \hfill 3
    \item
	Which brand has a more consistent daily sales distribution? Justify your answer. \hfill 4
  \end{enumerate}


\item
	  \textbf{An umbrella seller earns a revenue of BDT. 5000 if it rains. If it does not rain, he loses BDT. 1000. The probability that it rains on a given day is 0.04.} 
  
  \begin{enumerate}
    \item
	Write down the formula of Expectation for a continuous random variable. \hfill 1
    \item
	Can the value of Expectation be zero? \hfill 2
    \item  
	What is the umbrella seller's expected revenue? \hfill 3
    \item
	What should be the minimum probability of raining for him to achieve revenue greater than zero? \hfill 4
  \end{enumerate}
  
     \item
	  \textbf{A box contains 5 red and 6 white balls. 3 balls are drawn at random. X is the number of white balls drawn.} 
  
  \begin{enumerate}
    \item
	What does variance measure? \hfill 1
    \item
	Can the variance be smaller than standard deviation? \hfill 2
    \item  
	Find the E(X) from the stem. \hfill 3
    \item
	Find the variance from the stem assuming X is the number of red balls drawn. \hfill 4
  \end{enumerate}
  
   \item
	  \textbf{A professor showed a probability distribution in a class:}
	  
	  \begin{table}[h]
	  \begin{center}
\begin{tabular}{llllll}
x    & 1   & 2 & 3   & 4 & 5   \\ \hline
p(x) & 0.1 & a & 0.3 & b & 0.2 
\end{tabular} \\
\textbf{The value of the arithmetic mean of the distribution is 3.}
\end{center}	
\end{table}
    
  \begin{enumerate}
    \item
	What is the formula of expectation? \hfill 1
    \item
	What is the variance of a constant? Explain logically. \hfill 2
    \item  
	What are the values of a \& b? \hfill 3
    \item
	Find and explain the variance of the distribution. \hfill 4
  \end{enumerate}
  
       \item
	  \textbf{X is a random variable having the below functional form:} 
  
  \begin{center}
  $P(X) = \frac{6-|7-x|}{k}; x = 1, 2, \cdots,10$ \\
  \end{center}  
  
  Y is another variable having the relationship y = 3x+5
  
  \begin{enumerate}
    \item
	What is joint probability? \hfill 1
    \item
	What is the minimum possible value of variance? Why? \hfill 2
    \item  
	Find the value of k. \hfill 3
    \item
	Find E(X) and E(Y). Why are they different? \hfill 4
  \end{enumerate}
  
       \item
  \textbf{Various sales and their probabilities of a grocery store is given below}
  
  \begin{table}[h]
  \centering
\begin{tabular}{llllll}
Sales & 200 & 250 & 275 & 310 & 350 \\
Probability & 0.10 & 0.20 & 0.40 & 0.25 & 0.05
\end{tabular}
\end{table}

  \begin{enumerate}
    \item
	Can the expectation of a random variable be negative? \hfill 1
    \item
    	Find the expected sales of the store on a given day. \hfill 2
    \item
    	Compute the dispersion of sales f the store. \hfill 3
     \item
     	To make the expected sale 280, what sale does the store need in place of 200? \hfill 4
  \end{enumerate}
  
     \item
	  \textbf{A survey of Television (TV) users at Gulshan in Dhaka was conducted to find how many sets each family use. The following data were obtained:} 
	  
	  	  \begin{table}[h]
	  \begin{center}
\begin{tabular}{llllll}
No of TV set    & 0 & 1  & 2 & 3    \\ \hline
No of family & 10 & 75 & 10 & 5
\end{tabular}
\end{center}	
\end{table}
  
  \begin{enumerate}
    \item
	What is Expectation equivalent to? \hfill 1
    \item
	Can Variance be negative? Why or why not? \hfill 2
    \item  
	Find the variance of the number of TV sets. \hfill 3
    \item
	Find and compoare between arithmetic mean and expectation. \hfill 4
  \end{enumerate}
  \end{enumerate}
  
\section{Short Questions}

\begin{enumerate}



\item What is the formula for the expectation \( E(X) \) of a discrete random variable \( X \) \hfill 1

\item What is $E(X^2)$ equal to? \hfill 1

\item What is the relationship between expectation and arithmetic mean?  \hfill 1

\item Derive Expectation from arithmetic mean.  \hfill 2

\item Rewrite $E(2X)$.  \hfill 1

\item Rewrite $E(4X + c)$.  \hfill 1

\item Rewrite $E(4X + 7)$.  \hfill 1

\item Rewrite $E(\frac Y3 - 3)$.  \hfill 1

\item How can you expand $E(x+y)$? \hfill 1

\item Between $\{E(x)\}^2$ and $E(x^2)$, which one is greater?

\item How can you expand $E(xy)$? \hfill 1

\item Between $\displaystyle E\left(\frac{1}{x}\right)$ and $\displaystyle \frac 1{E(x)}$, which one is greater? \hfill 1

 \item Determine the formula of variance in terms of expectation. \hfill 2

\item How is \( E(x^2) \) found using \( V(x) \) and \( E(x) \)? \hfill 1

\item What is the formula for the variance of a discrete random variable \( X \)? \hfill 1

\item How is the variance \( V(X) \) related to the expectation \( E(X) \)? \hfill 1

\item What is the variance of a constant value? \hfill 1

\item How is the variance of the sum of two independent random variables calculated? \hfill 2

\item What is the relationship between variance and standard deviation? \hfill 1

\item $V(a) = ?$, where a is an arbitrary constant?  \hfill 1

\item $V(x-a) = ?$  \hfill 1

\item $V(\frac x3) = ?$  \hfill 1

\item $V(\frac x4 + 2) = ?$  \hfill 1

\item If $V(x) = 1$, what is the value of standard deviation? \hfill 1

\item If $V(x) = -9$, what is the value of standard deviation? \hfill 1

\item If $V(x) = 10$, what is the value of standard deviation? \hfill 1

\item If $V(x) = 1$, what is the value of standard deviation? \hfill 1

\item Can variance be smaller than standard deviation? \hfill 2

\item When can variance and standard deviation be equal? \hfill 2

\item $V(x + y)=?$ \hfill 1

\item $V(x - y)=?$ \hfill 1

\item What covariance? \hfill 1

\item Write down the formula of covariance. \hfill 1


\end{enumerate}

\chapter{Binomial Distribution} 
\section{Creative Questions}
  \begin{enumerate}
  
     \item
	  \textbf{A farmer selected a paddy field for seed collection. He found out that 10 out of each 25 paddies are damaged. He collected a sample of 15 paddies.} 
  
  \begin{enumerate}
    \item
	What is a Bernoulli trial? \hfill 1
    \item
	IF a Bernoulli trial is repeated n times, in how many ways are outcomes 
	generated? Explain. \hfill 2
    \item  
	Find the probability that at least one paddy is damaged. \hfill 3
    \item
	Comment on the skewness of the data. \hfill 4 
	
		[Hint: For a binomial distribution, $\gamma_1 = \frac{q-p}{\sqrt{npq}}$]
  \end{enumerate}
  
  \item
	  \textbf{A biologist is studying a group of plants and notes that 8 out of every 20 plants are infected with a certain disease. She collects a sample of 12 plants.} 
  
  \begin{enumerate}
    \item  
	Find the probability that at least one plant is infected. \hfill 3
    \item
	Determine the median of the distribution and explain its significance. \hfill 4
  \end{enumerate}


 \item
	  \textbf{The electric kettles produced by a certain manufacturer are 12\% 
	  defective on average. The company supplies 20 kettles in a packet. A retailer
	  bought 1000 packets.} 
  
  \begin{enumerate}
    \item  
	What is the probability that no. of defective kettles is at most 2? \hfill 3
    \item
	In how many packtes, there are exactly 3 defective kettles? \hfill 4
  \end{enumerate}
  
  \item
	  \textbf{A company produces smartphones, and it is known that 5\% of the 
	  smartphones have a manufacturing defect on average. The company ships 15 
	  smartphones in each box, and a retailer purchases 500 boxes.} 
  
  \begin{enumerate}
    \item  
	What is the probability that the number of defective smartphones in a box 
	is at least 1? \hfill 3
    \item
	How many boxes are expected to contain exactly 2 defective smartphones? \hfill 4
  \end{enumerate}

  
 \item
	  \textbf{A farmer plans to store rice seeds for future use. It was found 
	  that 8 out of 20 seeds are rotten. He then collected a sample of 15 seeds.} 
  
  \begin{enumerate}
    \item
	What is Bernoulli trial? \hfill 1
    \item
	How are Bernoulli and Binomial distributions related? \hfill 2
    \item  
	What is the probability that at least one seed is rotten out of 15? \hfill 3
    \item
	What is the probability that the number of rotten seeds is greater than the arithmetic mean? \hfill 4
  \end{enumerate}

 \item
	  \textbf{The number of defective pen produced by a company follows a 
	  binomial distribution with expectation 1.5 and variance 1.125.}. 
  
  \begin{enumerate}
    \item
	What is the mean of binomial distribution \hfill 1
    \item
	Can variance be greater than mean in binomial distribution? \hfill 2
    \item  
	Determine the probability function of the number of defective items 
	produced by the company. \hfill 3
    \item
	What is the probability that the number of defective items is no
	less than 3? \hfill 4
  \end{enumerate}
  \item
	  \textbf{The number of faulty light bulbs produced by a factory 
	  follows a binomial distribution with an expectation of 2 and a 
	  variance of 1.6.} 
  
  \begin{enumerate}
    \item  
	Determine the probability function for the number of faulty light 
	bulbs produced by the factory. \hfill 3
    \item
	What is the probability that the number of faulty light bulbs is 
	at least 4? \hfill 4
  \end{enumerate}

  
  \end{enumerate}

\section{Short Questions}

 \begin{enumerate}
 
 \item What is $p$ in binomial distribution? \hfill 1
 
 \item What is a Bernoulli trial?  \hfill 1
 
 \item How many outcomes are there in a Bernoulli trial?  \hfill 1
 
 \item What is the probability function of Bernoulli distribution? \hfill 1
 
 \item Is Bernoulli distribution discrete or continuous? \hfill 1
 
 \item What is the value of n (number of trial) in a Bernoulli distribution? \hfill 1
 
 \item What is relationship between Bernoulli and Binomial distribution? \hfill 1
 
 \item How many parameters does the Binomial distribution have? \hfill 1
  
  \item What are the paremeters of Binomial distribution? \hfill 2
  
  \item What happens if $n=1$ in Binomial distribution? \hfill 2
  
  \item What is denoted by $X$ in Binomial distribution? \hfill 1
  
 \item What is the mean of a binomial distribution with parameters \( n \) and \( p \)? \hfill 1

\item How is the variance of a binomial distribution with parameters \( n \) and \( p \) calculated? \hfill 1

\item What is the skewness of a binomial distribution \( p \)? \hfill 1

\item What is the kurtosis of a binomial distribution? \hfill 1

\item When is the binomial distribution symmetric? \hfill 1

\item How does the probability mass function (PMF) of a binomial distribution change with increasing \( n \)? \hfill 2

\item What is the relationship between the mean, variance, and skewness in a binomial distribution? \hfill 2

\item Is $E(X) < V(X)$ possible in binomial distribution? \hfill 1

\item Can $V(X)$ be equal to $E(X)$ in binomial distribution? Examine. \hfill 2

\item In a symmetric binomial distribution, what is the functional form of probability? \hfill 1


 \end{enumerate}

\chapter{Poisson Distribution} 
\section{Creative Questions}

\begin{enumerate}

   \item
	  \textbf{Between 1000hrs and 1700 hrs, the average number of phonce calls per minute received by a power distribution company is 2.5. } 
  
  \begin{enumerate}
    \item
	Give an example where Poisson distribution is applicable. \hfill 1
    \item
	Find the relationship between expectationa and standard deviation of Poisson distribution. \hfill 2
    \item  
	Find the probability that the number of calls is between 1 and 3 (inclusive). \hfill 3
    \item
	What is the probability that the number of calls received is greater than the average? \hfill 4
  \end{enumerate}
  
     \item
	  \textbf{The frequency distribution of defective items in packets of key rings is given below.} 
	  
	  \begin{table}[h]
	  \centering
\begin{tabular}{c|c|c|c|c|c|c}
Number of defective items & 0 & 1 & 2 & 3 & 4 & 5 \\ \hline
Number of packets & 76 & 74 & 29 & 17 & 3 & 1
\end{tabular}
\end{table}
  
  \begin{enumerate}
    \item
	What is another way to write $P(X \ge 1)$? \hfill 1
    \item
	Can the mean of Poisson distribution be negative? \hfill 2
    \item  
	From the given stem, find mean and variance. \hfill 3
    \item
	Find the expected frequencies and comment. \hfill 4
  \end{enumerate}
  
 \item
	  \textbf{A can manufacturer observes that 0.1\% of the produced cans are faulty. The cans are packaged in carton boxes, with 500 cans in each box. A wholeseller purchases 100 boxes from the manufacturer.} 
  
  \begin{enumerate}
    \item
	What is shape of Poisson distribution? \hfill 1
    \item
	For a Poisson distribution, P(2) = P (3). What is P(2)? \hfill 2
    \item  
	Find the probability of exactly one defective can. \hfill 3
    \item
	Find the expected number of boxes with no defective cans. \hfill 4
  \end{enumerate}
  
  \item
  \textbf{In winter, the probability that it rains on a particular day is 0.015. An analyst observes \\ 100 winter days.}
 
  \begin{enumerate}
    \item
	What is an experiment? \hfill 1
    \item
    	When can the Poisson distribution be approximated by the Binomial distribution? \hfill 2
    \item
    	Find, using Binomial distribution, the probability that  it would not rain at all on the \\ observed days. \hfill 3
     \item
     	Find the probability in 3(c) using Poisson distribution.  \hfill 4
  \end{enumerate}

  \item
	  \textbf{BTCL receives 2.5 telephone calls on average from 4 pm to 6 pm. The number of calls received is a random variable. } 
  
  \begin{enumerate}
    \item
	When is Poisson variate applicable? \hfill 1
    \item
	Show conversion criteria and method from Binomial to Poisson distribution. \hfill 2
    \item  
	Find the probability of receiving no more than 3 calls. \hfill 3
    \item
	Find the pattern of calls and show on graph paper.  \hfill 4 \\
	Hint: Find probabilities: P(0), P(1), $\cdots$
  \end{enumerate}
  
   \item
	  \textbf{The number of customers coming at supershop follows a Poisson 
	  distribution with mean 3.} 
  
  \begin{enumerate}
    \item  
	Determine the probability that in a particular minute, at least 1 customer 
	arrives. \hfill 3
    \item
	If $P(X = a) = P(X= b)$, find the value of $a$. What pattern do you get?
	\hfill 4
  \end{enumerate}
  
  \item
	  \textbf{The number of cars passing through a toll booth follows a 
	  Poisson distribution with a mean of 5 cars per minute.} 
  
  \begin{enumerate}
    \item  
	Determine the probability that exactly 3 cars pass through the toll 
	booth in a minute. \hfill 3
    \item
	If $P(X = a) = P(X = b)$, find the value of $a$ and $b$. What pattern do 
	you observe? \hfill 4
  \end{enumerate}

  
  \item
	  \textbf{The number of customers coming at a shop per minute follows a 
	  Poisson distribution,  whose mean is 3.} 
  
  \begin{enumerate}
    \item
	What is a Poisson variate? \hfill 1
    \item
	Can the mean of Poisson distribution be negative? \hfill 2
    \item  
	Find the probability that the number of customers coming is between 1 and 2. \hfill 3
    \item
	Are the probabilities of coming to 2 and 3 customers equal?  \hfill 4
  \end{enumerate}

 \item
	  \textbf{A random variable is distributed as follows:} 
	  
	  \begin{table}[h]
	  \centering
\begin{tabular}{c|c|c|c|c|c|c}
Value & 0 & 1 & 2 & 3 & 4 & 5 \\ \hline
Frequency & 70 & 73 & 27 & 15 & 4 & 1
\end{tabular}
\end{table}
  
  \begin{enumerate}
    \item
	What is the mean of Poisson distribution? \hfill 1
    \item
	What is the relationship between mean and standard deviation of Poisson 
	distribution? \hfill 2
    \item  
	Find the mean and variance of the given distribution. \hfill 3
    \item
	Compare the observed and expected frequencies, assuming a Possion 
	distribution. \hfill 4
  \end{enumerate}
  
  \item  
\textbf{A random variable is distributed as follows:}  

\begin{table}[h]
\centering
\begin{tabular}{c|c|c|c|c|c|c}
Value & 0 & 1 & 2 & 3 & 4 & 5 \\ \hline
Frequency & 60 & 80 & 50 & 20 & 6 & 2
\end{tabular}
\end{table}

\begin{enumerate}
    \item  
    Find the mean and standard deviation of the given distribution. \hfill 3
    \item  
    Compare the observed and expected frequencies, assuming a Poisson. \hfill 4
\end{enumerate}


  \end{enumerate}

\section{Short Questions}

\begin{enumerate}

  \item What is a Poisson variate? \hfill 1
  
  \item What is a Poisson process?  \hfill 1
  
  \item What is the mean of a Poisson distribution with parameter \( \lambda \)? \hfill 1

\item What is the variance of a Poisson distribution with parameter \( \lambda \)? \hfill 1

\item How is the probability mass function (PMF) of a Poisson distribution expressed? \hfill 1

\item When can the Poisson distribution be used as an approximation to the binomial distribution? \hfill 1

\item What is $e$ in Poisson distribution? \hfill 1

\item What is the mean of Poisson distribution? \hfill 1

\item What is the variance of Poisson distribution? \hfill 1

\item What is the relationship between mean and variance of Poisson distribution? \hfill 1

\item Is Poisson distribution discrete or continuous? \hfill 1

\item What is the moment generating function of Poisson distribution? \hfill 1

\item If $x_1 \sim Poisson (m_1)$ and  $x_2 \sim Poisson (m_2)$,  $x_1 + x_2 \sim ?$ \hfill 2

\item How is Poisson distribution skewed? \hfill 1

\item What is the kurtosis of Poisson distribution? \hfill 1

\item If a Poisson distribution is $P(x) = \frac{e^{-m}m^x}{x!}, P(x+1) = ?$ Derive in terms of $P(x)$. \hfill 2

\item Prove $\displaystyle \sum_{i=1}^{\infty} P(x) = \sum_{i=1}^{\infty} \frac{e^{-m} m^x}{x!} = 1$ \hfill 2
\end{enumerate}


\chapter{Normal Distribution} 
\section{Creative Questions}


\section{Short Questions}

\chapter{Index Number} 
\section{Creative Questions}


\section{Short Questions}

\chapter{Sampling} 
\section{Creative Questions}


\section{Short Questions}

\chapter{Vital Statistics} 
\section{Creative Questions}

  \begin{enumerate}
  
   \item
	  \textbf{A reseracher uses the following data to know about some demographic characterisics.} 
  
  \begin{enumerate}
    \item
	What is General Fertility Rate? \hfill 1
    \item
	What is the difference between GRR and NRR? \hfill 2
    \item  
	Compute the population density. \hfill 3
    \item
	Are TFR and GRR same for this data? \hfill 4
  \end{enumerate}
  
 \item
  \textbf{For projection of population in a future time period, demographers use simple, \\ geometric or exponential growth technique. Each method has its advantages and \\ disadvantages.}

  \begin{enumerate}
    \item
	What is geometric growth? \hfill 1
    \item
    	In geometric growth method, obtain the formula for time required for the population to get \\ doubled [denote rate as r]. \hfill 2
    \item
    	In exponential method, how much unit of time is required for the population to get tripled?  \hfill 3
     \item
     	For projecting (predicting future values), is geometric growth method better than the \\ exponential method? Justify.  \hfill 4
  \end{enumerate}
  
     \item
	  \textbf{Population of Dhaka and Sylhet by different age groups and areas are given below:} 
	  
	  \begin{table}[h]
	  \begin{center}
\begin{tabular}{lllll}
Division & \multicolumn{3}{c}{Age}         & Area ($km^2$) \\
         & 0-14      & 15-64    & 65+      &               \\
Dhaka    & 10,000,00 & 5,00,000 & 5,80,000 & 1,880         \\
Sylhet   & 7,00,000  & 2,70,000 & 4,70,000 & 2,319        
\end{tabular}
\end{center}
\end{table}
  
  \begin{enumerate}
    \item
	Write down the formula of dependency ratio. \hfill 1
    \item
	What is meant by NRR = 0.983? \hfill 2
    \item  
	Find and compare between the dependency ratios of the cities. \hfill 3
    \item
	Based on data, which city is more comfortable for living? \hfill 4
  \end{enumerate}
  
  \item
  \textbf{Population of New York, Los Angeles, and Chicago by different age groups and areas are given below:} 
  
  \begin{table}[h]
  \begin{center}
  \begin{tabular}{lllll}
  City         & \multicolumn{3}{c}{Age}         & Area (sq. km) \\
               & 0-14      & 15-64    & 65+      &               \\
  New York     & 1,200,000 & 5,000,000 & 700,000  & 789           \\
  Los Angeles  & 1,000,000 & 4,500,000 & 500,000  & 1,302         \\
  Chicago      & 900,000   & 3,800,000 & 600,000  & 606           \\
  \end{tabular}
  \end{center}
  \end{table}
  
  \begin{enumerate}
    \item  
    Find and compare the dependency ratios of New York and Chicago. \hfill 3
    \item
    Based on the data, which city is more comfortable for living? Justify your
    choice. \hfill 4
  \end{enumerate}

  
  
   \item
	  \textbf{As part of an analysis, a researcher collected data on women and live births.} 
	  \begin{table}[h]
	  \centering
\begin{tabular}{c|c|c|c|c|c|c|c}
Age & 15-19 & 20-24 & 25-29 & 30-34 & 35-39 & 40-44 & 45-49 \\ \hline
No. of Women & 540 & 760 & 530 & 495 & 450 & 505 & 430 \\ \hline
No. of live births & 109 & 198 & 86 & 90 & 65 & 76 & 60
\end{tabular}
\end{table}
  
  \begin{enumerate}
    \item
	What is the formula of death rate? \hfill 1
    \item
	Write down the uses of vital statistics. \hfill 2
    \item  
	Find teh Age Specific Birth Rates (ASFR). \hfill 3
    \item
	Find the GFR and compare its concept and value with ASFRs. \hfill 4
  \end{enumerate}
  
  \end{enumerate}

\section{Short Questions}

  \begin{enumerate}
  
  \item What does vital statistics deal with? \hfill 1
  
  \item Mention 4 sources of vital statistics. \hfill 2
  
  \item What are the sources of vital statistics. \hfill 2
  
  \item How is dependency ratio calculated? \hfill 1
  
  \item What is the formula of sex ratio? \hfill 1
  
  \item Write down the formula of population density? \hfill 1
  
  \item What is the formula of crude birth rate \hfill 1
  
  \item How is General Fertility Rate calculated? \hfill 1
  
  \item What is the purpose of Age-Specific Fertility Rate? \hfill 1
  
  \item Two dependency ratios are $d_1=98\%$ and $d_2=104\%$. 
  In which case are there more dependent people \\ per 1000 individuals? \hfill 
  
  \item Ques \hfill 
    
  \item Ques \hfill 

  
    \end{enumerate}

\backmatter
\chapter{Conclusion}
\lipsum[8]

\tableofcontents
\end{document}
