\documentclass[14pt, a4paper,oneside, margin=1.4in]{book}
\usepackage{problem_counter}
\usepackage{lipsum}
\usepackage[hidelinks]{hyperref}
\usepackage{multirow}
\usepackage{titletoc}
\usepackage{amsmath}
\usepackage{geometry}
\usepackage{graphicx}
\graphicspath{ {.} }
\geometry{a4paper, margin=1in}
\usepackage{enumitem} % For word def and ordered list
\newlist{worddefs}{description}{1}
\setlist[worddefs]{font=\sffamily\bfseries, labelindent=\parindent, leftmargin=6em, style=sameline}
\titlecontents*{chapter}
  [0pt]% <left>
  {}
  {\chaptername\ \thecontentslabel\quad}
  {}
  {\bfseries\hfill\contentspage}    

\usepackage{bookmark}
\usepackage{etoolbox}

\makeatletter
\newcommand*{\AddChapterPrefixInBookmarks}{%
  \if@mainmatter
    \ifnum\bookmarkget{level}=0 %
      \preto\bookmark@text{\@chapapp\space}%
    \fi
  \fi
}
\makeatother

\bookmarksetup{
  numbered,
  addtohook=\AddChapterPrefixInBookmarks,
}

% Workaround for numbered sections in unnumbered
% chapter "Introduction" to avoid chapter number
% zero.
\renewcommand*{\thesection}{%
  \ifcase\value{chapter}%
  \else
    \thechapter.%
  \fi
  \arabic{section}%
}

\title{My document}

\begin{document}
\frontmatter

\begin{titlepage}
    \begin{center}
        \vspace*{1cm}
            
        \Huge
        \textbf{Statistics Notes}
            
        \vspace{0.5cm}
        \huge
        First \& Second Paper
            
        \vspace{1.5cm}
            
        \textbf{Abdullah Al Mahmud}

     \vspace{1.5cm}

	\Large 
	Updated on: \today
            
        \vfill
            

            
        \vspace{0.8cm}
            
\includegraphics[width=1cm]{logo.png}
            
        \Large
        www.statmania.info\\
            
    \end{center}
\end{titlepage}


\tableofcontents


\mainmatter
\part {First Paper}

\chapter{Introduction}

\part {Second Paper}
\chapter{Probability} 

\begin{worddefs}
  \item[Trial.] Definition.
  
  \item[Experiment.] An act that can be repeated under some specific condition.
  
    
  \item[Random variable.] A variable whose values are associated with probability..
  
    
  \item[Sample space.] Set of all possible outcomes of a random experiment.
  
    
  \item[Sample point.]  Each outcome of a sample space.
  
    
  \item[Event.] Any subset of a sample space.
  
    
  \item[Simple event.] An event having a single outcome.
  
    
  \item[Compound/Composite event.] An event having more than one outcome.
  
    
  \item[Impossible event.] An event which cannot happen (If P(A) = 0, then A is an impossible event).
  
    
  \item[Certain event.] An event which surely will or will not happen. (P(A) = 0 or 1).
  
    
  \item[Trial.] Definition.
  
    
\end{worddefs}

To be fetched from Rmd file...


\section{Creative Questions}

\begin{enumerate}

     \item
	  \textbf{Events that do not depend on each other are called independent events, and events that cannot occurr simulataneously are called disjoint events.} 
  
  \begin{enumerate}
    \item
	Provide an example of disjoint events, using the set theory. \hfill 1
    \item
	Prove that $P(A\cap \bar B) = P(A) - P(A\cap B)$ \hfill 2
    \item  
	If there are k mutually and exhaustive events, prove $\displaystyle \sum_{i=1}^k P(A_i) = 1$ \hfill 3
    \item
	Prove that two events cannot be simulataneously independent and mutually exclusive. \hfill 4
  \end{enumerate}
    \end{enumerate}
    
    
\chapter{Random Variable and Probability Distribution}

\section{Terms}

\begin{worddefs}

  \item[Random variable.] A variable which is associated with probability.
  
  \item[Probability distribution] A distribution shows how the probability is distributed among the possible values or outcomes. It gives us a pattern of the data.
    
\end{worddefs}

\section{Concepts}

\begin{itemize}
  \item Recall a histogram
  \item We could plot relative frequencies instead of frequencies
  \item Relative frequencies are nothing but probabilities
\end{itemize}

\textbf{Example:}

\subsection{Examples of distribution}

If a biased coin is tossed once, the following may occur:

\begin{table}[h]
\centering
\begin{tabular}{c|c|c}
x    & H   & T   \\ \hline
P(x) & 1/3 & 2/3
\end{tabular}
\end{table}

This is one of the simplest kind of probability distribution. 

**Now**, if we toss a coin twice, we get the following sample space.

% Please add the following required packages to your document preamble:
% \usepackage{multirow}
\begin{table}[h]
\centering
\begin{tabular}{c|c|c|c}
\multicolumn{2}{c}{First Toss →}   & H  & T  \\ \hline
\multirow{2}{*}{Second Toss ↓} & H & HH & HT \\ 
                               & T & TH & TT
\end{tabular}
\end{table}

If we now define

X =  no. of heads

then we can construct the following probability distribution.
\begin{table}[h]
\centering
\begin{tabular}{c|c|c|c}
x    & 0   & 1   & 2   \\ \hline
P(x) & 1/4 & 1/2 & 1/4
\end{tabular}
\end{table}

Since 1 head can appear in two ways (HT, TH), so $P(1H)=\frac24=\frac12$.

Similarly, $P(2H)=\frac14$, and no head (0) can appear in 1 way, so $P(0) = \frac14$

These are tabular distribution. A distribution can also be expressed in a functional form. 

$$P(x) = \frac{x+k}{14}; x= 1,2,3,4$$

is a \textbf{discrete distribution}, since values of x are specific and isolated. The distributions involving a discrete random variable is called a probability (mass) function (pmf), and are denoted by P(x).

The following is a \textbf{continuous distribution}. 

\[f(x) = 6x(1-x); 0\le x \le 1\]

\section{Problems related to distribution}

\Problem A probability density function is given below:

$P(x) = \frac{x+k}{14}; x= 1,2,3,4$

\begin{enumerate}
  \item Find k
  \item $P(X>2)$
  \item $P(X \le 2)$
  \item $P(X \ge 3)$
  \item $P(X=2)$
  \item $P(2 \le X \le 4)$
\end{enumerate}

\TheSolution Class work

\Problem A joint probability density function is given below:

\[f(x) = x+ \frac32 y^2; 0\le x \le 1\, 0\le y \le 1\]



\backmatter
\chapter{Conclusion}
\lipsum[8]

\tableofcontents
\end{document}